
\chapter{Primo Teorema di Incompletezza}
	\hyphenation{lo-gi-ca a-rit-me-ti-ca rap-pre-sen-ta-bi-le ri-cor-si-va-men-te in-de-ci-di-bi-le e-sis-te in-tro-du-cia-mo e-qui-va-len-za ab-bia-mo}

\section{Risultati precedenti su decidibilit\`a effettiva
e decidibilit\`a ricorsiva}

	Pu\`o essere utile, per avere una panoramica concettuale
	completa, anteporre alla dimostrazione del primo teorema di
	incompletezza, risultato che nel framework del corso acquisisce carattere
	conclusivo, un breve riassunto dei risultati fin qui ottenuti.
	
	Una funzione $f$ dai numeri naturali ai numeri naturali si dice
	\textit{\emph{effettivamente} \emph{calcolabile}}
	se esiste una procedura
	effettiva che permette di calcolare $f(x_1,\,\dots,x_n)$ per ogni
	$n$-upla $(x_1,\,\dots,x_n)$. 	
	Ogni funzione ricorsiva totale \`e effettivamente calcolabile. Il viceversa,
	cio\`e che ogni funzione effettivamente calcolabile
	\`e ricorsiva totale, \`e noto
	come \textit{\emph{tesi di Church}}.
	
	Un insieme di numeri naturali \`e detto \emph{\textit{effettivamente
	decidibile}} se esiste una procedura effettiva che, per ogni
	numero, se applicata ad esso, in una quantit\`a
	finita di tempo, risponde correttamente
	alla domanda se tale numero appartenga \textit{o meno} all'insieme.
	Questa propriet\`a \`e equivalente alla propriet\`a per la funzione caratteristica
	dell'insieme di essere effettivamente calcolabile.
	Un insieme \`e detto \emph{\textit{ricorsivo}} se la sua funzione
	caratteristica \`e ricorsiva totale.
	Un insieme ricorsivo \`e effettivamente calcolabile, il viceversa segue dalla
	tesi di Church.
	
	Una relazione \`e \textit{\emph{effettivamente decidibile}} se l'insieme
	delle $n$-uple che la soddisfano lo \`e. \`E \textit{\emph{ricorsiva}}
	se l'insieme delle $n$-uple che la soddisfano lo \`e. Una relazione
	ricorsiva \`e effettivamente decidibile, il viceversa segue dalla tesi di
	Church.
	
	Una funzione $f$ da un sottoinsieme dei 
	numeri naturali ai numeri naturali si dice
	\textit{\emph{effettivamente semicalcolabile}}
	se esiste una procedura effettiva che permette di calcolare
	$f(x_1,\,\dots,x_n)$ per ogni $n$-upla $(x_1,\,\dots,x_n)$ su cui
	$f$ \`e definita.
	Ogni funzione ricorsiva parziale \`e effettivamente semicalcolabile.
	Il viceversa, 	cio\`e che ogni funzione effettivamente semicalcolabile
	\`e ricorsiva parziale, segue dalla tesi di Church.
	
	Un insieme \`e detto \textit{\emph{effettivamente semidecidibile}}
	se esiste una
	procedura effettiva che, per ogni numero, se applicata ad esso, se il numero
	appratiene all'insieme dar\`a risposta affermativa in un intervallo finito di
	tempo, se non vi appartiene, non dar\`a risposta.
	Questa propriet\`a \`e equivalente alla fatto che esiste una funzione
	semicalcolabile di valore costante $1$ il cui dominio \`e l'insieme stesso.
	Questa funzione \`e detta \textit{\emph{semicaratteristica}}.
	Un insieme \`e detto \emph{\textit{ricorsivamente enumerabile}} se
	la sua funzione semicaratteristica \`e ricorsiva.
	Un insieme ricorsivamente enumerabile \`e effettivamente semidecidibile,
	il viceversa segue dalla tesi di Church.
	
	Una relazione \`e \textit{\emph{effettivamente semidecidibile}}
	se l'insieme
	delle $n$-uple che la soddisfano lo \`e. \`E
	\textit{\emph{ricorsivamente enumerabile}}
	se l'insieme delle $n$-uple che la soddisfano lo \`e. Una relazione
	ricorsivamente enumerabile \`e effettivamente semidecidibile,
	il viceversa segue dalla tesi di Church.

\section{Risultati precedenti su rappresentabilit\`a
e ricorsivit\`a}
	Sia $T$ una teoria con uguaglianza nel linguaggio $\mathcal{L}_A$
	dell'aritmetica. Una funzione $f$ si dice \emph{\textit{rappresentabile}}
	in $T$ se esiste una formula $\varphi(x_1,\,\dots,x_n,\,y)$ di $T$
	tale che:\\
	
	\begin{quote}
	Se $f(k_1,\,\dots,\,k_n)=m$ allora $\vdash_T\forall y(\varphi(
	\overline{k_1},\,\dots,\,\overline{k_n},\,y)\leftrightarrow y=\overline{m})$.
	\end{quote}
	Una relazione $R$ si dice \emph{esprimibile} in T se
	esiste una formula $\varphi(x_1,\,\dots,x_n)$ di $T$
	tale che:\\
	
	\begin{quote}
	Se $R(k_1,\,\dots,\,k_n)$ \`e vera allora $\vdash_T\varphi(\overline{k_1},\,
	\dots,\,\overline{k_n})$,\\
	Se $R(k_1,\,\dots,\,k_n)$ \`e falsa allora $\vdash_T\neg\varphi(\overline{k_1},\,
	\dots,\,\overline{k_n})$.\\
	\end{quote}
	Ogni funzione ricorsiva totale \`e rappresentabile in $T$. Ogni relazione
	ricorsiva (complementata) \`e esprimibile in $T$.
	
	Una relazione $R$ si dice \textit{\emph{semirappresentabile}} in $T$
	se esiste una formula $\varphi(x_1,\,\dots,\,x_n)$ tale che:
	
	$$
	R(k_1,\,\dots,\,k_n) \mbox{\:\`e vera sse\:} \vdash_T\varphi(\overline{k_1},\dots,\,
	\overline{k_n})
	$$
	Una relazione \`e semirappresentabile se e solo se \`e ricorsivamente enumerabile.
	
	\textit{La rappresentabilit\`a e l'esprimibilit\`a ci permettono di parlare
	di funzioni e relazioni all'interno del sistema formale stesso}.
	
\section{Decidibilit\`a}
	
	\begin{defi}
		Un insieme di simboli, o espressioni, o oggetti pi\`u complicati \`e detto
		\emph{ricorsivo} sse l'insieme dei numeri di codifica
		dei suoi elementi \`e ricorsivo.
	\end{defi}
	
	\begin{defi}
		Un linguaggio si dice \emph{ricorsivo} se
		l'insieme dei numeri di codifica dei suoi simboli \`e ricorsivo.
	\end{defi}

	\begin{defi}
		Si dice \emph{teoria} un insieme di enunciati\footnote{Con enunciato
		si intende formula della teoria senza variabili libre.} che contiene
		tutti gli enunciati del suo linguaggio che sono dimostrabili a partire
		da essa.
	\end{defi}
	
	\begin{defi}
		Una teoria $T$ si dice \emph{assiomatizzabile} se esiste un
		un insieme decidibile $\Gamma$ di enunciati  tale che $T$ consiste
		di tutte e soli gli enunciati dimostrabili da $\Gamma$. Si dice
		\emph{ricorsivamente assiomatizzabile} se $\Gamma$ \`e ricorsivo.
		Per la tesi di Church le due definizioni sono equivalenti e noi le
		confonderemo.
	\end{defi}
	
	\begin{defi}
		Una teoria si dice \emph{decidibile} se \`e un insieme ricorsivo.
	\end{defi}
	
	\begin{defi}
		Una teoria $T$ si dice \emph{\textit{sintatticamente consistente}} o
		\emph{\textit{coerente}}
		se non esiste un enunciato $\varphi$ in $T$ tale che $\vdash_{T}\varphi$
		e $\vdash_{T}\neg\varphi$; equivalentemente, per la regola \textit{ex falso},
		se non dimostra ogni formula del suo linguaggio.
	\end{defi}
		
	\begin{defi}
		Un enunciato $\varphi$ nel linguaggio $\mathcal{L}_T$ di una teoria $T$
		si dice \emph{refutabile} se $\vdash_T\neg\varphi$.
	\end{defi}
	
	\begin{defi}
		Un enunciato $\varphi$ nel linguaggio $\mathcal{L}_T$ di una teoria $T$
		si dice \emph{indecidibile} se non \`e dimostrabile n\'e refutabile.
	\end{defi}
	
	\begin{defi}
		una teoria $T$ si dice \emph{\textit{completa}} se per ogni enunciato 
		$\varphi$ del suo linguaggio vale $\vdash_{T}\varphi$ oppure
		$\vdash_{T}\neg\varphi$.
	\end{defi} 
	
	\begin{defi}
		Una teoria si dice \textit{\emph{incompleta}} se ammette un enunciato
		indecidibile.
	\end{defi}

\subsection{Premessa}
	Nei prossimi tre paragrafi forniremo quattro versioni della dimostrazione
	del teorema di incompletezza di G\"odel.
	
	Le prime tre sono classiche. La prima di queste:
	
	\begin{itemize}
	\item segue l'approccio di Boolos;
	\item \`e classica\footnote{Quando la dimostrazione \`e
	classica, scriveremo $PA$ anzich\'e $HA$, riferendoci all'
	``aritmetica di Peano''.};
	\item \`e molto elegante;
	\item non suppone l'$\omega$-consistenza;
	\item usa il diagonalization lemma;
	\item non produce esplicitamente un enunciato indecidibile.
	\end{itemize}
	La seconda:
	\begin{itemize}
	\item segue l'approccio di G\"odel;
	\item \`e classica;
	\item \`e pi\`u macchinosa;
	\item suppone l'$\omega$-consistenza;
	\item non usa il diagonalization lemma;
	\item produce esplicitamente un enunciato indecidibile.
	\end{itemize}
	La terza:
	\begin{itemize}
	\item segue l'approccio di Rosser, raffinamento del risultato di G\"odel;
	\item \`e classica;
	\item \`e pi\`u macchinosa;
	\item rilassa l'ipotesi di $\omega$-consistenza;
	\item non usa il diagonalization lemma;
	\item produce esplicitamente un enunciato indecidibile.
	\end{itemize}
	La quarta:
	\begin{itemize}
	\item segue l'approccio di Sambin e Maietti;
	\item \`e intuizionista;
	\item \`e elegante;
	\item non necessita dell'ipotesi di $\omega$-consistenza in quanto
	in $HA$ si dimostra l'existence property;
	\item usa il diagonalization lemma;
	\item produce esplicitamente un enunciato indecidibile.
	\end{itemize}

\section{Primo teorema di incompletezza-prima versione}

	I teoremi di incompletezza di G\"odel esprimono i limiti di un qualunque
	sistema formale che sia abbastanza forte da permetterci di ``riprodurre''
	al suo interno la teoria dei numeri. In realt\`a il lavoro iniziale di G\"odel
	faceva riferimento a un sistema specifico, quello descritto da B. Russell e
	A. N. Whitehead nei \textit{Principia ma\-the\-ma\-ti\-ca}, ma il principio
	\`e in generale applicabile anche a tutti gli altri sistemi formali dotati di
	certe propriet\`a, come, nella nostra trattazione, $PA$.\\
	Prima di procedere stabiliamo che con teoria, impiantata su un qualche
	linguaggio, intendiamo un insieme che contiene tutte le formule senza
	variabili libere di quel linguaggio, tali che siano in essa dimostrate.
		
	La codifica di $MPA$ in $PA$, ottenuta attraverso l'aritmetizzazione,
	render\`a possibile definire in $PA$ formule autoreferenziali ed in
	particolare una formula che nega la propria dimostrabilit\`a. Il primo
	teorema di incompletezza prender\`a il via da queste considerazioni e,
	supposta la consistenza di $PA$, mostrer\`a l'esistenza di una formula
	indecidibile\footnote{E dunque l'incompletezza di $PA$.}.

	\begin{prop}
	Sia $T$ una teoria assiomatizzabile. Se $T$ \`e completa, $T$ \`e decidibile.
	\end{prop}
	
	\textsc{Dimostrazione.}\\
	Dobbiamo mostrare che l'insieme $T^*$ dei codici dei teoremi di $T$
	\`e ricorsivo. Che sia ricorsivamente enumerabile lo sappiamo dai
	capitoli precedenti, pertanto dobbiamo solo mostrare che anche il
	suo complemento \`e ricorsivamente enumerabile. Il complemento di
	$T^*$ \`e l'unione dell'insieme $X$ dei numeri che non sono codici
	di formule e dell'insieme $Y$ dei codici di formule la cui negazione
	sta in $T$, per completezza. Ma $X$ \`e ricorsivo visto che lo \`e
	il suo complemento, e $Y$ \`e ricorsivamente enumerabile: la sua
	funzione semicaratteristica si ottiene facilmente componendo la
	funzione primitiva ricorsiva $neg$ con la funzione semicaratteristica
	di $T*$, operazione che lasciamo al lettore. Dunque per il teorema
	di Kleene possiamo concludere.
	\begin{flushright}$\Box$\end{flushright}

\subsection{Diagonalizzazione di una formula}
	L'apparato formale necessario alla dimostrazione dei teoremi di
	G\"odel \`e finalmente completo. L'obiettivo delle prossime
	righe sar\`a quello di implementare e di studiare il problema
	dell'autoreferenza in $PA$. Cominciamo dalla nozione di
	diagonalizzazione:

	\begin{defi}Data una qualunque formula $\varphi$, la sua
	diagonalizzazione \`e la formula $\exists x(x=\overline
	{\gdnum{\varphi}} \& \varphi)$.
	\end{defi}
	In particolare, se $\varphi(x)$ \`e una formula dotata di una
	sola variabile libera $x$, notiamo l'equivalenza logica $\exists
	x(x=\overline{\gdnum{\varphi}} \& \varphi)\leftrightarrow\varphi
	(\overline{\gdnum{\varphi}})$, che possiamo leggere come: "il
	numerale del G\"odeliano di $\varphi$ la verifica".

	\begin{prop}
	\label{pro:funPriRic}
	Esiste una funzione primitiva ricorsiva $$diag:\mathbb{N}\rightarrow\mathbb{N}$$
	tale che, per ogni formula $\varphi$, $diag(\gdnum{\varphi})=\gdnum{\exists
	x(x=\overline{\gdnum{\varphi}} \& \varphi)}$.
	\end{prop}
	
	\textsc{Dimostrazione.}\\
	La funzione
	
	\begin{eqnarray*}
	num: \mathbb{N} & \longrightarrow & \mathbb{N}\\
	n &\mapsto& \gdnum{\overline{n}}
	\end{eqnarray*}
	\`e definita dallo schema di ricorsione
	
	$$
	\left\{\begin{array}{l}
	num(0)=19\\
	num(n+1)=\gdnum{s} \star \gdnum{(} \star num(n) \star\gdnum{)}
	\end{array}\right.
	$$
	ed \`e dunque primitiva ricorsiva. Applicata a un numero qualunque,
	restituisce il codice del numerale che rappresenta quel numero.
	Grazie ad essa possiamo definire
	
	$$
	diag(n):=\gdnum{\exists x(x=}\star num(n)\star\gdnum{\&}\star n\star\gdnum{)}.
	$$
	Cos\`i costruita, la funzione $diag$ \`e primitiva ricorsiva,
	essendo composizione di funzioni primitive ricorsive. Non resta
	che verificare la condizione definitoria. Data $\varphi$ formula,
	si PA
	
	\begin{eqnarray*}
	diag(\gdnum{\varphi})=\gdnum{\exists x(x=}\star num(\gdnum{\varphi})
	\star\gdnum{\&}\star \gdnum{\varphi}\star\gdnum{)}=\\
	=\gdnum{\exists x(x=}\star \gdnum{\overline{\gdnum{\varphi}}})\star
	\gdnum{\&}\star \gdnum{\varphi}\star\gdnum{)}=\\
	=\gdnum{\exists x(x=\overline{\gdnum{\varphi}}\&\varphi)}
	\end{eqnarray*}
	e dunque la tesi.\begin{flushright}$\Box$\end{flushright}
	
	In altri termini, la funzione $diag$ appena definita \`e primitiva ricorsiva
	(quindi totale) ed associa, in particolare, al codice di una formula il codice
	della corrispondente diagonalizzazione.
	
\subsection{Dimostrazione del primo teorema di incompletezza di G\"odel}
%lemma diagonale
%indefinibilità di theta
%indecidibilità essenziale
%primo teorema di incompletezza
	
	Forti dei risultati a cui siamo arrivati, procediamo verso la prova del
	primo teorema di incompletezza di G\"odel, dal quale ci separano soltanto
	qualche lemma e un teorema. Il primo passo consiste nella dimostrazione del:
	
	\begin{prop}[Lemma diagonale]
	\label{lem:diagLemma}
	Sia $T$ una teoria abbastanza forte da esprimere l'aritmetica, e sia
	$L$ il linguaggio sul quale essa \`e impiantata. Per ogni formula
	$\psi(x)$ di $L$ esiste una formula $\delta_\psi$ tale che
	
	$$
	\vdash_{T} \delta_\psi \leftrightarrow \psi(\overline{\ulcorner
	\delta_\psi \urcorner})
	$$
	\end{prop}
	Dove con  ``teoria abbastanza forte da esprimere l'aritmetica'' si intende
	una teoria che possa rappresentare i naturali, l'uguaglianza e l'ordine dei
	naturali, la somma e il prodotto. Il requisito viene dettato dalla necessit\`a
	di effettuare l'aritmetizzazione per conseguire l'autoriferimento, il che
	risulter\`a chiaro dalla dimostrazione. Evidentemente $PA$ risulta essere
	una teoria abbastanza forte. Prima di dimostrare il lemma, osserviamo che
	lo possiamo leggere come: ``per qualunque propriet\`a si possa descrivere
	attraverso una formula del linguaggio $L$, esiste un'altra formula nello
	stesso linguaggio che viene dimostrata dalla teoria se e solo se il suo
	codice gode della propriet\`a in questione''.
	
	\textsc{Dimostrazione.}\\ 
	Dimostriamo il lemma esibendo la $\delta_\psi$ per una generica formula $\psi(x)$.\\
	Sappiamo che la funzione $diag:\mathbb{N}\rightarrow\mathbb{N}$, che applicata
	al codice di una formula restituisce il codice della relativa diagonalizzazione,
	\`e primitiva ricorsiva. Quindi  \`e totale e soprattutto \`e rappresentabile.
	Ricordiamo che questo significa che esiste una formula $\varphi_{diag}$ tale che
	se $diag(n)=m$ allora $\vdash_{T} \forall y(\varphi_{diag}(\overline{n},y)
	\leftrightarrow y=\overline{m})$.\\
	Definiamo:
	
	$$
	\chi(x)\equiv\exists y(\varphi_{diag}(x,y)\&\psi(y))
	$$
	nella quale notiamo la dipendenza da $\psi$, e chiamiamo il suo numero di G\"odel:
	
	$$
	a\equiv\ulcorner\chi(x)\urcorner
	$$
	quindi costruiamo la $\delta_\psi$ di interesse come diagonalizzazione di $\chi(x):$
	
	$$
	\delta_\psi \equiv \exists x(x=\overline{a} \& \chi(x))
	$$
	e diamo un nome al suo G\"odeliano:
	
	$$
	d\equiv \ulcorner\delta_\psi \urcorner
	$$
	Dal momento che in generale $\exists x(x = t \& A(x))$
	equivale ad $A(t)$, notiamo che nello specifico vale l'equivalenza
	
	$$
	\vdash_{T} \delta_\psi \leftrightarrow \chi(\overline{a})
	$$
	Da questo e dalla definizione di $\chi(x)$, se riusciamo a
	mostrare che $$\vdash_{T} \exists y(\varphi_{diag}(\overline{a},y)
	\&\psi(y)) \leftrightarrow \psi(\overline{d})
	$$
	possiamo concludere.

	Dato che $\delta_\psi$ \`e la diagonalizzazione di $\chi(x)$ si PA che
	$diag(a)=d$ e dunque, per rappresentabilit\`a:
	
	$$
	\vdash_{T} \forall y(\varphi_{diag}(\overline{a},y)\leftrightarrow y=\overline{d})
	$$
	Ma allora, ricordandosi che in generale se vale $A \leftrightarrow B$
	vale anche $(A \& C) \leftrightarrow (B \& C)$, abbiamo che
	
	$$
	\vdash_{T} \forall y((\varphi_{diag}(\overline{a},y)\&\psi(y))\leftrightarrow
	(y=\overline{d}\&\psi(y)))
	$$
	Per cui, visto che se vale una formula quantificata universalmente allora
	anche l'esistenziale vale di sicuro, possiamo affermare che
	
	$$
	\vdash_{T} \exists y((\varphi_{diag}(\overline{a},y)\&\psi(y))\leftrightarrow
	(y=\overline{d}\&\psi(y)))
	$$
	Dalla quale si arriva a
	
	$$
	\vdash_{T} \exists y(y=\overline{d}\&\psi(y)) \leftrightarrow \psi(\overline{d})
	$$
	Dunque la tesi.
	\begin{flushright}$\Box$\end{flushright}

	La dimostrazione appena conclusa fornisce un procedimento esplicito
	per costruire $\delta_{\psi}$ a partire dalla formula $\psi$ data.\\  
	Diamo ora la nozione di definibilit\`a di un insieme di naturali.
	
	\begin{defi} Un insieme $S$ di naturali \`e definibile in una
	teoria consistente $T$ se esiste una formula $D(x)$ tale che
	per ogni naturale $n$ vale $\vdash_{T}D(\overline{n})$ se $n$
	sta in $S$ e $\vdash_{T}\neg D(\overline{n})$ se $n$ non sta in $S$.
	\end{defi}
	
	Osserviamo che:
	
	\begin{prop}
	Sia $S$ un insieme ricorsivo (o decidibile), allora S \`e
	definibile.
	\end{prop}
	
	\textsc{Dimostrazione.}\\
	Se $S$ \`e decidibile allora la sua funzione caratteristica $\chi_S$ \`e
	calcolabile totale, quindi rappresentabile da un predicato a due posti
	$\varPhi(x,y)$. Allora \`e immediato verificare che possiamo definire $S$
	con il predicato $D(x)\equiv \varPhi(x, \overline{1}) $.
	\begin{flushright}$\Box$\end{flushright}
	
	Ora abbiamo il vocabolario per dimostrare che:
	
	\begin{prop}
	Sia $T$ una teoria consistente abbastanza forte da esprimere l'aritmetica,
	l'insieme dei numeri di G\"odel dei teoremi di $T$ non \`e definibile in $T$.
	\end{prop}
	
	\textsc{Dimostrazione.}\\
	Sia $\Theta$ l'insieme dei codici dei teoremi di $T$. Supponiamo che il predicato
	$\theta(y)$ definisca $\Theta$. Per il lemma diagonale sappiamo che esiste una
	formula chiusa $G$ tale che
	
	$$
	\vdash_{T} G \leftrightarrow \neg\theta(\overline{\ulcorner G \urcorner})
	$$
	Quindi se $\vdash_{T} G$ allora $\overline{\ulcorner G \urcorner}$ sta in
	$\Theta$ e $\vdash_{T} \theta(\overline{\ulcorner G \urcorner})$\\
	ma se $\vdash_{T} G$ allora anche $\vdash_{T} \neg\theta(\overline
	{\ulcorner G \urcorner})$,
	contro la consistenza.\\
	Se $\not\vdash_{T} G$ allora $\overline{\ulcorner G \urcorner}$ non sta
	in $\Theta$ e $\vdash_{T} \neg\theta(\overline{\ulcorner G \urcorner})$\\
	ma se $\vdash_{T} \neg\theta(\overline{\ulcorner G \urcorner})$ allora
	anche $\vdash_{T} G$, il che \`e una contraddizione.\\
	L'errore sta nell'aver assunto $\Theta$ definibile, dunque la tesi.
	\begin{flushright}$\Box$\end{flushright}
	
	\begin{thm}[Indecidibilit\`a essenziale]
	Nessuna teoria T, consistente e abbastanza forte da esprimere l'aritmetica,
	\`e decidibile.
	\end{thm}
	
	\textsc{Dimostrazione.}\\
	Dal lemma precedente l'insieme $\Theta$ dei codici dei teoremi di $T$ non
	\`e definibile in $T$. Ma ogni insieme ricorsivo \`e definibile in $T$,
	quindi $\Theta$ non \`e ricorsivo, ovvero $T$ non \`e decidibile.
	\begin{flushright}$\Box$\end{flushright}
	
	\begin{thm}[Primo teorema di incompletezza]
	Non esiste nessuna teoria T, abbastanza forte da esprimere l'aritmetica,
	che sia assiomatizzabile, completa e sintatticamente consistente.
	\end{thm}
	
	\textsc{Dimostrazione.}\\
	Ogni teoria assiomatizzabile e completa \`e decidibile, ma ogni teoria consistente
	in grado di esprimere l'aritmetica \`e indecidibile.
	\begin{flushright}$\Box$\end{flushright}
	
\section{Primo teorema di incompletezza-seconda e terza versione.}
		
\subsection{Enunciati indecidibili}
	
	Sia $T$ una teoria del prim'ordine con gli stessi simboli di $PA$.
	
	\begin{defi}
	$T$ si dice $\omega$-consistente sse, per ogni formula $\varphi(x)$ di $T$,
	se $\vdash_T\varphi(\overline{n})$ per ogni numero naturale $n$, allora
	$\not\vdash\exists x\neg\varphi(x)$.
	\end{defi}
	
	\begin{thm}
	Se $T$ \`e $\omega$-consistente, allora $T$ \`e consistente.
	\end{thm}
	\textsc{Dimostrazione.} Supponiamo $T$ $\omega$-consistente.
	Sia $\varphi(x)$ una formula tale che $\vdash_T
	\varphi(x)$, segue che $\vdash_T\varphi(\overline{n})$ per ogni
	numero naturale $n$. Per $\omega$-consistenza segue che
	$\not\vdash_T\exists x\neg\varphi(x)$. Dunque $K$
	\`e consistente, altrimenti per la regola \textit{ex falso} ogni formula
	sarebbe dimostrabile.\begin{flushright}$\Box$\end{flushright}
	
	Introduciamo le seguenti relazioni
	
	$$
	W_1(u,\,y):=Form(u)\& Fv(u,\,\gdnum{x_1})\& Proof(y,\,Sub(
	\gdnum{\overline{u}},\,\gdnum{x_1},\,u))
	$$
	che dice: $u$ \`e il numero di codifica di una formula $\varphi(x_1)$
	che contiene la variabile libera $x_1$
	e $y$ \`e il numero di codifica di una dimostrazione di $\varphi(\overline{u})$;
	
	$$
	W_2(u,\,y):=Form(u)\& Fv(u,\,\gdnum{x_1})\& Proof(y,\,Sub(
	\gdnum{\overline{u}},\,\gdnum{x_1},\,\gdnum{\neg}\ast u))
	$$
	che dice: $u$ \`e il numero di codifica di una formula $\varphi(x_1)$
	che contiene la variabile libera $x_1$
	e $y$ \`e il numero di codifica di una dimostrazione di
	$\neg\varphi(\overline{u})$.
	
	$W_1(u,\,y)$ \`e primitiva ricorsiva, dunque \`e esprimibile in $PA$ per mezzo
	di una formula $\mathcal{W}_1(x_1,\,x_2)$ con $x_1$ e $x_2$ variabili libere.
	Consideriamo la formula:
	
	$$
	g_{PA}(x_1)\equiv\forall x_2\neg\mathcal{W}_1(x_1,\,x_2)
	$$
	sia $m=\gdnum{g_{PA}}$, consideriamo l'enunciato:
	
	$$
	G_{PA}\equiv\forall x_2\neg\mathcal{W}_1(\overline{m},\,x_2)
	$$
	detto \emph{enunciato di G\"odel}. Si ha che
	(I) $W_1(m,\,y)$ vale se e solo se $y$ \`e il numero di codifica
	di una dimostrazione in $PA$ di $G_{PA}$.
	
	\begin{thm}[Teorema di G\"odel per $PA$, 1931]
	Se $PA$ \`e consistente, allora $G_{PA}$ non \`e
	dimostrabile in PA. Se PA \`e $\omega$-consistente,
	allora $G_{PA}$ non \`e refutabile in PA. In
	particolare, se PA \`e $\omega$-consistente, $G_{PA}$
	\`e indecidibile.
	\end{thm}
	
	\textsc{Dimostrazione.}\\
	Supponiamo $PA$ consistente e $G_{PA}$ dimostrabile in $PA$.\\
	Sia $k$ il numero di codifica di una dimostrazione in $PA$.\\
	Per (I) abbiamo
		$$W_1(m,\,k)$$
	per esprimibilit\`a segue che
		$$\vdash_{PA} \mathcal{W}_1(\overline{m},\,\overline{k}).$$
	Ma poich\'e 
		$$\vdash_{PA}G_{PA}$$
	per pura logica segue che
		$$\vdash_{PA}\neg\mathcal{W}_1(\overline{m},\,\overline{k}).$$
	Da cui segue che 
		$$\vdash_{PA}\mathcal{W}_1(\overline{m},\,\overline{k})$$
	e
		$$\vdash_{PA}\neg\mathcal{W}_1(\overline{m},\,\overline{k})$$
	e ci\`o \`e contro la consistenza di $PA$.
	
	Supponiamo $PA$ $\omega$-consistente e $G_{PA}$ refutabile.\\
	Per consistenza segue che
		$$\not\vdash_{PA} G_{PA}.$$
	Da cui segue che per ogni numero naturale $n$, $n$ non \`e il numero di codifica di
	una dimostrazione in $PA$ di $G_{PA}$.\\
	Per (I) segue che per ogni $n$, $W_1(m,\,n)$ \`e falsa.\\
	Per rappresentabilit\`a segue che per ogni $n$
		$$\vdash_{PA}\neg\mathcal{W}_1(\overline{m},\,\overline{n}).$$
	Per $\omega$-consistenza segue che
		$$\not\vdash_{PA}\exists x_2\neg\neg \mathcal{W}_1(\overline{m},\,x_2)$$
	da cui segue che
	$$\not\vdash_{PA}\exists x_2\mathcal{W}_1(\overline{m},x_2).$$
	Ci\`o \`e contro la refutabilit\`a di $G_{PA}$.
	\begin{flushright}$\Box$\end{flushright}
	
	$W_2(u,\,y)$ \`e primitiva ricorsiva, dunque \`e esprimibile in $PA$ per mezzo
	di una formula $\mathcal{W}_2(x_1,\,x_2)$ con $x_1$ e $x_2$ variabili libere.
	Consideriamo la formula:
	
	$$
	r_{PA}(x_1)\equiv
	\forall x_2(\mathcal{W}_1(x_1,\,x_2)\rightarrow
	\exists x_3(x_3\leq x_2\& \mathcal{W}_2(x_1,\,x_3)))
	$$
	sia $n=\gdnum{r_{PA}}$, consideriamo l'enunciato:
	$$
	R_{PA}\equiv
	\forall x_2(\mathcal{W}_1(\overline{n},\,x_2)\rightarrow
	\exists x_3(x_3\leq x_2\& \mathcal{W}_2(\overline{n},\,x_3)))
	$$
	detto \emph{enunciato di Rosser}. Si ha che
	(II) $W_1(n,\,y)$ vale se e solo se $y$ \`e il numero di codifica
	di una dimostrazione in $PA$ di $R_{PA}$ e (III)
	$W_2(n,\,y)$ vale se e solo se $y$ \`e il numero di codifica
	di una dimostrazione in $PA$ di $\neg R_{PA}$.
	
	\begin{thm}[Teorema di G\"odel-Rosser, 1936]
	Se $PA$ \`e consistente, allora $R_{PA}$ \`e indecidibile.
	\end{thm}
	\textsc{Dimostrazione.} Supponiamo $PA$ consistente e $R_{PA}$ dimostrabile.\\
	Sia $k$ il numero di codifica di una dimostrazione in $PA$.\\
	Per (II) segue 
		$$W_1(n,\,k).$$
	Per esprimibilit\`a segue che
		$$\vdash_{PA}\mathcal{W}_1(\overline{n},\,\overline{k}).$$
	Ma per dimostrabilit\`a di $R_{PA}$ e per pura logica segue che
		$$\vdash_{PA}\mathcal{W}_1(\overline{n},\,\overline{k})
		\rightarrow\exists x_3(x_3\leq\overline{k}\&\mathcal{W}_2
		(\overline{n},\,x_3)).$$
	Per pura logica segue che
		$$\vdash_{PA}\exists x_3(x_3\leq\overline{k}\&\mathcal{W}_2
		(\overline{n},\,x_3)).$$
	Ora, dalle ipotesi segue che
		$$\not\vdash\neg R_{PA}.$$
	Per (III) segue che $W_2(n,\,y)$ \`e falsa per ogni
	numero naturale $y$.\\
	Per esprimibilit\`a segue che $\vdash_{PA}
	\neg\mathcal{W}_2(\overline{n},\,\overline{j})$ per ogni numero naturale
	$j$.\\
	Da cui segue che 
		$$\vdash_{PA}
		\neg\mathcal{W}_2(\overline{n},\,0)\&
		\neg\mathcal{W}_2(\overline{n},\,\overline{1})\&
		\dots\&
		\neg\mathcal{W}_2(\overline{n},\,\overline{k}).$$
	Per proposizioni precedenti su $HA$ e dunque $PA$\footnote{In quanto
	ci\`o che vale in logica intuizionista, vale in logica classica.}
	segue che
		$$\vdash_{PA}\forall x_3(x_3\leq\overline{k}
		\rightarrow\neg\mathcal{W}_2(\overline{n},\,x_3)).$$
	Per pura logica segue che
		$$\vdash_{PA}\neg\exists x_3(x_3\leq\overline{k}
		\&\mathcal{W}_2(\overline{n},\,x_3)).$$
	Ma questa \`e la negazione di una formula derivata precedentemente.
	Ci\`o contraddice la consistenza di $PA$.
	
	Supponiamo $R_{PA}$ refutabile.\\
	Sia $r$ il numero di codifica di una
	dimostrazione di $\neg R_{PA}$.\\
	Per (III) vale $W_2(n,\,r)$.\\
	Per rappresentabilit\`a segue che 
		$$\vdash_{PA}\mathcal{W}_2(\overline{n},\,\overline{r}).$$
	Per consistenza di $PA$ segue che che 
		$$\not\vdash_{PA}R_{PA}.$$
	Per (II) segue che $W_1(n,\,y)$ \`e falsa per ogni
	numero naturale $y$.\\
	Da cui segue che $\vdash_{PA}\neg\mathcal{W}_1
	(\overline{n},\,\overline{j})$ per ogni numero naturale $j$.\\
	In particolare
		$$\vdash_{PA}\neg\mathcal{W}_1(\overline{n},\,0)\&
		\neg\mathcal{W}_1(\overline{n},\,\overline{1})\&\dots
		\&\neg\mathcal{W}_1(\overline{n},\,\overline{r}).$$
	Per proposizioni precedenti su $HA$ segue che:
	
	\begin{quote}
	a) $\vdash_{PA} x_2\leq\overline{r}\rightarrow\neg\mathcal{W}_1
	(\overline{v},\,x_2)$.
	\end{quote}
	D'altra parte, consideriamo la seguente deduzione
	
	$$
	\begin{array}{ll}
	\vdash_{PA}\overline{r}\leq x_2						 								& \mbox{ipotesi}\\
	\vdash_{PA}\mathcal{W}_2(\overline{n},\,\overline{r})								& \mbox{gi\`a dimostrato prima}\\
	\vdash_{PA}\overline{r}\leq x_2\&\mathcal{W}_2(\overline{n},\,\overline{r})		& \mbox{1, 2, pura logica}\\
	\vdash_{PA}\exists x_3(x_3\leq x_2\&\mathcal{W}_2(\overline{n},\,x_3))			& \mbox{3, pura logica}
	\end{array}
	$$
	Per $1-4$ e per pura logica segue che:
	
	\begin{quote}
	b) $\vdash_{PA}\overline{r}\leq x_2\rightarrow
	\exists x_3(x_3\leq x_2\&\mathcal{W}_2(\overline{n},\,x_3)).$
	\end{quote}
	Per proposizioni precedenti su $HA$ segue che:
	
	\begin{quote}
	c) $\vdash_{PA} x_2\leq\overline{r}\vee\overline{r}\leq x_2.$
	\end{quote}
	Per $a-c$ e per pura logica segue che:
	
	$$
	\vdash_{PA}\neg\mathcal{W}_1(\overline{n},\,x_2)\vee\exists x_3
	(x_3\leq x_2\&\mathcal{W}_2(\overline{n},\,x_3)).
	$$
	Per pura logica segue che:
	
	$$
	\vdash_{PA}\forall x_2(\mathcal{W}_1(\overline{n},\,x_2)\rightarrow\exists x_3
	(x_3\leq x_2\&\mathcal{W}_2(\overline{n},\,x_3))).
	$$
	Da cui segue che
		$$\vdash_{PA} G_{PA}.$$
	Per refutabilit\`a ci\`o contraddice la consistenza di
	$PA$.\begin{flushright}$\Box$\end{flushright}
	
	Se si rende $PA$ pi\`u forte, ad esempio aggiungendovi $G_{PA}$ e
	ottenendo $PA_1$, $PA_1$ ammette ancora un
	enunciato indecidibile. Infatti
	qualunque funzione ricorsiva, essendo rappresentabile in $PA$, \`e
	rappresentabile in $PA_1$ e, ovviamente, la relazione $W_{1,PA_1}$,
	scritta per $PA_1$, sar\`a primitiva ricorsiva. Ma questo \`e
	tutto ci\`o di cui abbiamo
	bisogno per ottenere il risultato di G\"odel.
	
	Pi\`u in generale, poich\'e per giungere ai teoremi
	di G\"odel e G\"odel-Rosser abbiamo
	usato la rappresentabilit\`a delle funzioni in $PA$ e la primitiva
	ricorsivit\`a di $Proof$, cio\`e di $Der$, ogni estensione $T$ di $PA$ che
	soddisfa a queste propriet\`a ammetter\`a un enunciato indecidibile.
	
	Poich\'e per ottenere $Der$ in $T$ primitivo ricorsivo \'e sufficiente
	che $T$ sia ricorsivamente assiomatizzabile segue che:
	
	\begin{thm}
	Ogni estensione consistente e ricorsivamente assiomatizzabile di $PA$
	ammette un enunciato indecidibile.
	\end{thm}
	
\subsection{Dimostrabilit\`a e verit\`a}
	
	Poich\'e $\mathcal{W}_1$ esprime $W_1$ in $PA$,
	l'interpretazione standard di $G_{PA}:=\forall x_2\neg
	\mathcal{W}_1(\overline{m},\,x_2)$ afferma che $W_1(m,\,x_2)$ \`e falsa per
	ogni numero naturale $x_2$. Per (I) $\not\vdash_{PA}G_{PA}$.
	Cio\`e \textit{$G_{PA}$ afferma la propria indimostrabilit\`a in
	$PA$}. Per il teorema di G\"odel, se $PA$ \`e consistente,
	$G_{PA}$ \`e indimostrabile. Da questo segue che $G_{PA}$ \textit{\`e vera per
	l'interpretazione standard ma non \`e dimostrabile}. Un ragionamento analogo
	si pu\`o fare per $R_{PA}$.
	
\section{Primo teorema di incompletezza-quarta versione}

	\`E fondamentale la seguente propriet\`a di $HA$:

	\begin{thm}[\textbf{existence property}]
	Data una qualsiasi formula $\phi(x)$ dotata di un'unica variabile libera $x$,
	se vale $\vdash_{HA}\exists x\phi(x)$ allora esiste un numero naturale
	$n$ tale che $\vdash_{HA}\phi(\overline{n})$.
	\end{thm}
	La dimostrazione, che tralasciamo per difficolt\`a e lunghezza,
	\`e basata sull'analisi delle possibili prove.

	Introduciamo un nuovo
	simbolo: per ogni formula $\varphi$ definiamo
	$\Box\varphi\equiv TH(\overline{\ulcorner \varphi \urcorner})$.
	Esprime il fatto che $\varphi$ \`e dimostrabile in $HA$.
	Per diagonalization lemma, precedentemente dimostrato,
	posto $\psi(x)=\neg TH(x)$, otteniamo:

	\begin{prop}
	\label{auto}
	Esiste una formula $G$ tale che $\vdash_{HA} G \leftrightarrow\neg\Box G$.
	$G$ \`e detta \textit{enunciato di G\"odel}.
	\end{prop}

	\begin{thm} Il predicato $\Box$ soddisfa le seguenti condizioni:

	\begin{itemize}
 	\item[\small{HBL.1}:] se $\vdash_{HA}A$ allora $\vdash_{HA}\Box A$;
 	\item[\small{HBL.2}:] $\vdash_{HA} \Box A\rightarrow \Box(\Box A)$;
 	\item[\small{HBL.3}:] $\vdash_{HA} \Box(A\rightarrow B)
 	\rightarrow (\Box A\rightarrow \Box B)$;
	\end{itemize}
	dette \textit{condizioni HBL di derivabilit\`a}, con HBL che
	sta per Hilbert-Bernays-L\"ob.
	\end{thm}

	Da HBL.1 segue:

	\begin{prop}
	\label{hg1}Se $\vdash_{HA} G$ allora $\vdash_{HA}\Box G$.
	\end{prop}

	\begin{oss}
	\label{oss:HBL}
	La condizione $HBL.1$ pu\`o essere rafforzata, valendo anche l'implicazione inversa:
	\begin{eqnarray}
	\vdash_{HA}\Box A\:\:\:\:\Rightarrow\:\:\:\:\vdash_{HA} A.\nonumber
	\end{eqnarray}
	Si supponga infatti che $\vdash_{HA}\Box A$, ossia $\vdash_{HA}\exists
	yPR(\overline{y},\overline{\ulcorner A \urcorner})$.
	Per ``existence property''  esiste un numero naturale $n$ tale
	che $\vdash_{HA} PR(\overline{n},\overline{\ulcorner A \urcorner})$.
	Ci\`o significa che $n$ codifica la prova di $A$, pertanto a
	partire da $n$ si pu\`o costruire la derivazione di $A$ in $HA$.
	Si conclude che $\vdash_{HA} A$.
	\end{oss}

	\begin{thm}
	\label{teo:nonG}
	Se $HA$ \`e consistente allora $\not\vdash_{HA} G$.
	\end{thm}

	\textsc{Dimostrazione.}\\
	Supponiamo $\vdash_{HA} G$. Per proposizione \ref{hg1}
	segue che $\vdash_{HA}\Box G$.
	D'altra parte per proposizione \ref{auto} segue che
	$\vdash_{HA}\neg\Box G$. Ci\`o \`e 
	contro l'ipotesi di consistenza di $HA$.
	\begin{flushright}$\Box$\end{flushright}

	\begin{prop}
	\label{lem:hbl3bis}
	Date $\phi$ e $\psi$ formule, si ha
	$$
	\begin{array}{ll}
	1. & \vdash_{HA}\phi\rightarrow\psi
	\Rightarrow\vdash_{HA}\Box\phi\rightarrow\Box\psi;\\
	2. & \vdash_{HA} \Box\phi\&\Box\psi\rightarrow\Box(\phi\&\psi);\\
	3. & \vdash_{HA} \neg \Box \bot \rightarrow G.
	\footnote{Ricordiamo che $\bot$ \`e
	definita nel linguaggio di $HA$
	come $(0=1)$. Dagli assiomi di $HA$ segue che soddisfa gli
	assiomi standard del falso.}
	\end{array}
	$$
	\end{prop}

	\textsc{Dimostrazione.}\\

	Punto 1.\\
	Per HBL.1 seghe che se

	$$\vdash_{HA} \phi \rightarrow \psi$$
	allora 

	$$\vdash_{HA} \Box(\phi\rightarrow\psi)$$
	per HBL.3 segue la tesi.

	Punto 2.\\
	Poich\'e vale 
	$$\vdash_{HA} \phi \rightarrow (\psi \rightarrow \phi\&\psi)$$
	ed \`e possibile derivare in logica intuizionistica la seguente regola
	$$\prooftree
	\Gamma\vdash A \rightarrow B \justifies \Gamma,A\vdash B\using{\rightarrow ri}
	\endprooftree$$
	segue che:
	$$\prooftree
	\[\[\[\[\[
	\vdash_{HA} \phi \rightarrow (\psi \rightarrow \phi\&\psi)\justifies
 	 \vdash_{HA} \Box\phi\rightarrow\Box(\psi\rightarrow\phi\&\psi)\using{\text{proposizione \ref{lem:hbl3bis}}}\] \justifies
      \Box\phi\vdash_{HA}\Box(\psi\rightarrow\phi\&\psi)\using{\rightarrow ri}\] \justifies
       \Box\phi\vdash_{HA}\Box\psi\rightarrow\Box(\phi\&\psi)\using{\text{\tiny HBL.3}}\]\justifies
        \Box\phi,\Box\psi\vdash_{HA}\Box(\phi\&\psi)\using{\rightarrow ri}\]\justifies
         \Box\phi\&\Box\psi\vdash_{HA}\Box(\phi\&\psi)\using{\& left}\]\justifies
          \vdash_{HA}\Box\phi\&\Box\psi\rightarrow\Box(\phi\&\psi)\using{\rightarrow right}
	\endprooftree$$
	
	Punto 3.\\
	Per proposizione \ref{auto} e per contronominale si ha

	$$\vdash_{HA} \neg\neg\Box G \leftrightarrow \neg G$$
	Poich\'e in generale vale

	$$\vdash_{HA} \Box G \rightarrow \neg\neg\Box G$$
	segue che
	
	$$\vdash_{HA}\Box G \rightarrow \neg G$$
	Per HBL1 segue che
	
	$$\vdash_{HA}\Box(\Box G \rightarrow\neg G)$$
	per HBL3 segue che

	$$\vdash_{HA}\Box\Box G \rightarrow \Box\neg G$$
	per HBL2 segue che
	$$\vdash_{HA} \Box G \rightarrow \Box\Box G$$
	che insieme all'ultimo risultato comporta che

	$$\vdash_{HA} \Box G \rightarrow \Box\neg G$$
	dal cui segue che

	$$\vdash_{HA} \Box G \rightarrow (\Box\neg G \& \Box G)$$
	per proposizione \ref{lem:hbl3bis} segue che
	
	$$\vdash_{HA} \Box G \rightarrow \Box(\neg G \& G)$$
	da cui segue che
	
	$$\vdash_{HA} \Box G \rightarrow \Box\bot$$
	per contronominale segue che
	
	$$\vdash_{HA} \neg\Box\bot \rightarrow \neg\Box G$$
	per proposizione \ref{auto} e per composizione segue che:

	$$\vdash_{HA} \neg\Box\bot \rightarrow G.$$
	Ci\`o \`e quanto volevamo dimostrare.
	\begin{flushright}$\Box$\end{flushright}

	Siamo pronti per dimostrare:

	\begin{thm}[Primo teorema di incompletezza per $HA$]
	\label{teo:inc1}
	Se HA \`e consistente allora $\vdash_{HA}\Box\bot$ \`e indecidibile.
	\end{thm}
	
	\textsc{dimostrazione}\\
 	Assumiamo $\vdash_{HA}\Box\bot$. Per osservazione
 	\ref{oss:HBL} vale $\vdash_{HA}\bot$.
	Poich\`e nel sistema c\`e l'assioma $\vdash_{HA}\neg\bot$,
	segue che che HA \`e inconsistente, contro l'ipotesi.
 	Supponiamo $\vdash_{HA}\neg\Box\bot$.
	Per proposizione \ref{lem:hbl3bis} $\vdash_{HA}\neg\Box\bot\rightarrow G$,
	segue che $\vdash_{HA} G$. Ci\`o \`e contro teorema \ref{teo:nonG}.
	\begin{flushright}$\Box$\end{flushright}
	\\
	Una dimostrazione alternativa abbastanza intuitiva potrebbe essere la seguente:
	
	\begin{thm}[Primo teorema di incompletezza per $HA$] \qquad
	\begin{enumerate}
	\item Se $\vdash_{HA}\phi$ allora $\vdash_{PA}\phi$;
	\item esiste G tale che $\nvdash_{PA} G$ e $\nvdash_{PA} \neg G$ \\
	Allora da 1. e 2. segue che:
	\item esiste G tale che $\nvdash_{HA}$ e $\nvdash_{HA} \neg G$.
	\end{enumerate}
	\end{thm}
	
	\textsc{dimostrazione}\\
 	Assumiamo $\vdash_{HA}G$. Allora per 1. segue che $\vdash_{PA} G$, contro l'ipotesi 2.
	Assumiamo $\vdash_{HA}\neg G$. Allora per 1. segue che $\vdash_{PA} \neg G$, contro l'ipotesi 2.
	\begin{flushright}$\Box$\end{flushright}
	Questa dimostrazione, per\`{o}, risulta non soddisfacente ai nostri scopi, in quanto la caratteristica costruttiva che delinea le nostre dimostrazioni viene a mancare nella seconda ipotesi del teorema. \\
	Per questo motivo confermiamo come quarta versione del primo teorema di completezza quella data da Maietti-Sambin.
