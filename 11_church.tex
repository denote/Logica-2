% ANDREA ZANETTI - MATR 622592 - CDL Magistrale in informatica

% BIBLIOGRAFIA
% Elliot Mendelson, Introduzione alla logica matematica (1972)
% Wikipedia

% Concetti che volendo sarebbero da esplicitare (ma già discussi)
% consistenza
% teorie 1o ordine
% fbf
% thm compl godel
% thm deduzione
% formula logicamente valida
% fintamente assiomatizzabile
% teoria abb forte da expr l'aritmetica

\newcommand{\godel}{G\"{o}del }
\newcommand{\ents}{\textit{Entscheidungsproblem}}

\chapter{Teorema di Church}

\section{Introduzione}

Nell'agosto del 1900, il matematico tedesco David Hilbert partecipò al già citato congresso internazionale di
 matematica di Parigi e, nel suo intervento, presentò una lista di ventitrè problemi che apparivano,
 allo stato della matematica di allora, irrisolti. Il secondo di questi problemi, chiamato \ents \, 
(letteralmente ``problema della decisione'') chiede se si possa definire un algoritmo che possa
 prendere in input la descrizione di un linguaggio formale ed un'espressione del linguaggio e produrre in
output un valore di verità corrispondente al valore di verità dell'espressione data. L'\ents \, si può dunque
identificare con il problema della decisione nella logica del primo ordine (ovvero il verificare algoritmicamente
se un'espressione del primo ordine sia valida), mentre è banalmente risolvibile per la logica proposizionale
(basta costruire la tavola di verità dell'espressione e verificare se si tratti o meno di una tautologia).
Alonzo Church and Alan Turing pubblicarono (rispettivamente nel 1936 e 1937) articoli che indipendentemente
dimostrarono l'impossibilità di decidere algoritmicamente se un'espressione data in un certo linguaggio 
sia vera o falsa, e quindi come una soluzione generale dell'\ents \, sia impossibile. Questo risultato è
conosciuto oggi come teorema di Church o teorema di Church-Turing (da non confondere con la tesi di Church
o tesi di Church-Turing\footnote{Ricordiamo che (informalmente) la tesi di Church è un'ipotesi
 che afferma che se un problema è (intuitivamente) calcolabile, allora esiste una macchina di Turing
 (o un dispositivo equivalente) in grado di di calcolarlo; questo equivale a dire che
 che la classe delle funzioni calcolabili coincide con quella delle funzioni 
calcolabili da una macchina di Turing. }). Qualche ulteriore considerazione riguardo a queste dimostrazioni
 si può trovare nelle conclusioni del capitolo (\ref{chrcon}).

\section{Indecidibilità ricorsiva}
Prima di arrivare alla formulazione del teorema vero e proprio, 
avremo bisogno di chiarire alcuni concetti relativi all'indecidibilità delle teorie.

\begin{defi}
Definiamo una funzione $D(u)$ primitiva ricorsiva tale che, se $u$ è il numero di \godel
 di una formula $\varphi(x_1)$ con variabile libera $x_1$, allora $D(u)$
è il numero di \godel di $\varphi(\bar{u})$. Quindi $D(u)=Sub(\ulcorner \bar{u} \urcorner , \ulcorner x_1 \urcorner , u)$ o, equivalentemente, $D(u)=\ulcorner \varphi(\overline{\ulcorner \varphi(x_1) \urcorner}) \urcorner $
\end{defi}

\begin{defi}
Fissata una teoria $K$, per $T_K$ intendiamo l'insieme dei numeri di \godel
dei teoremi di $K$.
\end{defi}

\begin{prop}
\label{tkne}
Sia $K$ una teoria del primo ordine con identità con gli stessi simboli di $HA$.
Se $K$ è consistente e la funzione $D$ è rappresentabile in $K$, allora $T_K$ non è
esprimibile in $K$.
\end{prop}

% PER LA REGOLA A4
\begin{proof}
Supponiamo che $D$ sia rappresentabile in $K$ e che $T_K$ sia esprimibile in $K$.
Allora esistono formule $\mathcal{D}(x_1,x_2)$ e $\mathcal{T}(x_2)$ tali che:
\begin{enumerate}
 \item Se $D(k)=j$, allora $\vdash \!\! _K \; \mathcal{D}(\bar{k},\bar{j})$
 \item $\vdash \!\! _K \; (\exists!x_2)\mathcal{D}(\bar{k},x_2)$
 \item Se $k$ è in $T_K$, allora  $\vdash \!\! _K \; \mathcal{T}(\bar{k})$
 \item Se $k$ non è in $T_K$, allora  $\vdash \!\! _K \; \neg \mathcal{T}(\bar{k})$.
\end{enumerate}
Consideriamo ora la formula $\varphi(x_1):\forall x_2(\mathcal{D}(x_1,x_2) \rightarrow \neg \mathcal{T}(x_2))$,
e sia $p$ il numero di \godel di tale formula. Costruiamo dunque la formula 
$$\varphi(\bar{p}): \forall x_2(\mathcal{D}(\bar{p},x_2) \rightarrow \neg \mathcal{T}(x_2)).$$
Sia $q$ il numero di \godel di $\varphi(\bar{p})$. Quindi $D(p)=q$. Perciò per (1)
si ha che $\vdash \!\! _K \; \mathcal{D}(\bar{p},\bar{q})$. Ora o vale $\vdash \!\! _K \; \varphi(\bar{p})$
o vale $\nvdash \!\! _K \; \varphi(\bar{p})$.\newline
Se $\nvdash \!\! _K \; \varphi(\bar{p})$ allora $q$ non è in $T_K$ e così, per (4), $\vdash \!\! _K \; 
\neg \mathcal{T}(\bar{q})$. \newline
Se invece $\vdash \!\! _K \; \varphi(\bar{p})$ allora
 $\varphi(\bar{p}): \forall x_2(\mathcal{D}(\bar{p},x_2) \rightarrow \neg \mathcal{T}(x_2))$. \newline
Quindi $\vdash \!\! _K \; \mathcal{D}(\bar{p},\bar{q}) \rightarrow \neg \mathcal{T}(\bar{q})$; ma
$\vdash \!\! _K \; \mathcal{D}(\bar{p},\bar{q})$.
 Quindi $\vdash \!\! _K \; \neg \mathcal{T}(\bar{q})$. \newline
Così in in entrambi i casi $\vdash \!\! _K \; \neg \mathcal{T}(\bar{q})$. \newline
Ora da $\vdash \!\! _K \; \mathcal{D}(\bar{p},\bar{q})$ e da (2) abbiamo che
$\vdash \!\! _K \; \mathcal{D}(\bar{p},x_2) \rightarrow x_2 = \bar{q}$.
Ma poichè $\vdash \!\! _K \; \neg \mathcal{T}(\bar{q})$, allora
$\vdash \!\! _K \; x_2 = \bar{q} \rightarrow \neg \mathcal{T}(x_2)$.
Quindi $\vdash \!\! _K \; \mathcal{D}(\bar{p},x_2) \rightarrow \neg \mathcal{T}(x_2)$, e
per generalizzazione $\vdash \!\! _K \; (x_2)(\mathcal{D}(\bar{p},x_2) \rightarrow \neg \mathcal{T}(x_2))$,
cioè $\vdash \!\! _K \; \varphi(\bar{p})$. Perciò $q$ è in $T_K$ e, per (3),
 $\vdash \!\! _K \; \mathcal{T}(\bar{q})$. Dal momento che abbiamo anche 
$\vdash \!\! _K \; \neg \mathcal{T}(\bar{q})$ $K$ è inconsistente.
\end{proof}

\textbf{Nota:} la dimostrazione precedente viene assunta valida, ma non è costruttiva. In
particolare possiamo notare che è classica quando procedendo per casi asseriamo che 
$\vdash \!\! _K \; \neg \mathcal{T}(\bar{q})$ vale sia quando $\nvdash \!\! _K \; \varphi(\bar{p})$
che quando $\vdash \!\! _K \; \varphi(\bar{p})$, e quindi consideriamo che valga sempre.


\begin{corol}
\label{tkkk}
Se $K$ è consistente e ogni funzione ricorsiva è rappresentabile in $K$, allora $T_K$ non è
esprimibile in $K$. Quindi $T_K$ non è ricorsivo.
\end{corol}

\begin{proof}
$D$ è ricorsiva primitiva, e perciò rappresentabile in $K$. Per la proposizione \ref{tkne} $T_K$
non è esprimibile in $K$ e di conseguenza la sua funzione caratteristica $\chi_{T_K}$ non è
rappresentabile in $K$. Quindi $\chi_{T_K}$ non è ricorsiva, e perciò $T_K$ non è ricorsivo. 
\end{proof}

\begin{defi}
Una teoria $K$ si dice \underline{ricorsivamente indecidibile} se e solo se $T_k$ non è ricorsivo; $K$ si dice \underline{essenzialmente ricorsivamente indecidibile} se e solo se $K$ 
ed ogni estensione consistente di $K$ sono ricorsivamente indecidibili.
\end{defi}

\begin{defi}
Una teoria $K$ si dice \underline{essenzialmente incompleta} se e solo se $K$ ed ogni estensione consistente
di $K$ sono incomplete.
\end{defi}


Se si accetta la tesi di Church, allora l'indecidibilità ricorsiva è equivalente all'indecidibilità
effettiva; chiarito questo punto da qui in avanti il concetto di indecidibilità 
ricorsiva sarà espresso semplicemtente
dalla parola \textit{indecidibile}.

\section{Sistema di Robinson}

\begin{defi}
Chiamiamo \textbf{\emph{aritmetica di Robinson\footnote{La formulazione dell'aritmetica di Robinson mediante 
gli assiomi Q1-Q7 sopra riportati è la più comune, 
sebbene la formulazione originale (Raphael Robinson, 1950) si sviluppi in 13 assiomi dei quali i primi 6
si rendono necessari solo se non consideriamo una teoria con uguaglianza. In alcune trattazioni 
viene aggiunto a Q un ulteriore assioma (unicità del resto) per formare un sistema solitamente denotato
come RR allo scopo di semplificare alcune dimostrazioni.}
}} (o sistema di Robinson) una teoria del primo ordine denotata $Q$ con uguaglianza,
che utilizza il linguaggio dell'aritmetica del primo ordine e per la quale valgono i seguenti assiomi:
\end{defi}
\begin{itemize}
 \item [(Q1)] $\forall x \neg (S(x)=0)$
 \newline (zero non è il successiore di nessun numero)
 \item [(Q2)] $\forall x \forall y (S(x)=S(y) \rightarrow x=y) $
 \newline (numeri diversi hanno successori diversi)
 \item [(Q3)] $\forall x ( \neg(x=0) \rightarrow \exists y(x=S(y))) $
 \newline (ogni numero diverso da zero è il successore di qualche altro numero)
 \item [(Q4)] $\forall x(x+0=x)$
 \item [(Q5)] $\forall x \forall y (x+S(y)=S(x+y))$
 \newline (definizione ricorsiva dell'addizione)
 \item [(Q6)] $\forall x(x\times0=0)$
 \item [(Q7)] $\forall x \forall y (x\times S(y)=(x \times y)+x)$
 \newline (definizione ricorsiva della moltiplicazione)
\end{itemize}

Si noti che per com'è costruita $Q$ rappresenta una versione ridotta di $HA$ finitamente assiomatizzabile
ed in cui è assente
 il principio di induzione, mentre c'è l'aggiunta dell'assioma (Q3) che afferma che ogni numero naturale diverso
 da zero è successore di qualche altro numero (cosa che in $HA$ è dimostrabile appunto per induzione).
Solitamente in letteratura si confronta $Q$ con $PA$ anzichè con $HA$; allo scopo di ciò che andremo a
dimostrare questo non è rilevante, infatti senza perdere di generalità a noi interessa che valga
$$ Q \subseteq S $$
con
\begin{itemize}
 \item [(S)] teoria del primo ordine soggetta al primo teorema di incompletezza (ovvero consistente,
assiomatizzabile ed incompleta);
 \item [(Q)] sottoteoria finitamente assiomatizzabile di S anch'essa soggetta al primo teorema di incompletezza.
\end{itemize}
Mettiamo ora in evidenza alcune proprierà di Q.

\begin{prop}
\label{ofrq}
Ogni funzione ricorsiva è rappresentabile in Q.
\end{prop}
\begin{proof}
Si procede in modo analogo a quanto già fatto parlando di rappresentabilità,
 nella fattispecie dimostrando separatamente la rappresentabilità
per le funzioni di base, la composizione, la ricorsione e la minimalizzazione. Una
formalizzazione precisa della dimostrazione è lasciata come (semplicissimo!) esercizio.
\end{proof}

\begin{prop}
\label{piii}
Q è essenzialmente indecidibile.
\end{prop}
\begin{proof}
Segue dalla proposizione \ref{ofrq} e dal corollario \ref{tkkk}.
\end{proof}

\begin{prop}
\label{piiii}
Q è essenzialmente incompleta.
\end{prop}
\begin{proof}
Sapendo che vale la proposizione \ref{ofrq}, la tesi è diretta conseguenza del 
teorema di Gödel-Rosser. Una dimostrazione rigorosa è lasciata per esercizio.
\end{proof}

Naturalmente i risultati delle proposizioni \ref{piii} e \ref{piiii} sono già stati ottenuti per la
teoria $HA$; la ragione per la quale ci siamo preoccupati di ottenerli ancora per $Q$
è che $Q$ è finitamente assiomatizzabile. \newline
Il nostro interesse nell'aritmetica di Robinson risiede nel fatto che è la teoria più debole
 in cui è possibile rappresentare tutte le funzioni ricorsive e di conseguenza è anche la teoria
 più debole a cui sia applicabile il primo teorema di incompletezza di Gödel. \newline

\begin{defi}
Siano $K_1$ e $K_2$ due qualsiasi teorie del primo ordine con gli stessi simboli. 
$K_2$ si dice \underline{estensione finita} di $K_1$ se e solo se esistono un insieme
 $A$ di formule ed un insieme finito $B$ di formule tali che:
\begin{itemize}
 \item [(a)] i teoremi di $K_1$ sono esattamente le formule derivabili da $A$;
 \item [(b)] i teoremi di $K_2$ sono esattamente le formule derivabili da $A \cup B$.
\end{itemize}
\end{defi}

\begin{defi}
Siano $K_1$ e $K_2$ due qualsiasi teorie del primo ordine con gli stessi simboli.
 $K_1$ e $K_2$ si dicono \underline{compatibili} se e solo se la teoria $A \cup B$,
 ovvero la teoria il cui insieme di assiomi è formato dall'unione degli insiemi di assiomi di $K_1$ e $K_2$, 
è consistente.
\end{defi}

\begin{prop}
\label{p111}
 Siano $K_1$ e $K_2$ teorie del primo ordine con gli stessi simboli di $HA$.
Se $K_2$ è un'estensione finita di $K_1$ ed inoltre $K_2$ è indecidibile, allora anche $K_1$ è indecidibile.
\end{prop}

\begin{proof} 
Sia $A$ un insieme di assiomi per $K_1$, sia inoltre $A \cup \{ \alpha_1,...,\alpha_n \}$ un insieme
di assiomi per $K_2$. Si può supporre che $ \alpha_1,...,\alpha_n $ siano formule chiuse (ovvero
in cui non compaiono variabili libere). \newline
Allora per il teorema di deduzione\footnote{ Il teorema di deduzione (Herbrandt, 1930) afferma che
se $\Gamma$ è un'insieme di formule, $\mathcal{A},\mathcal{B}$ sono formule e $\Gamma,\mathcal{A} \vdash \mathcal{B}$
allora $\Gamma \vdash \mathcal{A} \rightarrow \mathcal{B}$ e viceversa. Lo stesso concetto è già stato presentato
parlando di HA con la regola $\rightarrow$-right.} 
una formula $\varphi$ è dimostrabile in $K_2$ se e solo se
$(\alpha_1\wedge...\wedge\alpha_n) \rightarrow \varphi $ è dimostrabile in $K_1$, ovvero:
$$ \vdash _{K_{2}} \varphi \;\;\; sse \;\; \vdash \!\! _{K_{1}} (\alpha_1\wedge...\wedge\alpha_n) \rightarrow \varphi. $$
Sia ora $c$ il numero di \godel di $(\alpha_1\wedge...\wedge\alpha_n)$, allora $b$ sarà il numero di \godel di un teorema di $K_2$ se e solo se
$ \ulcorner ( \urcorner * c * \ulcorner \rightarrow \urcorner * b * \ulcorner ) \urcorner $ rappresenta il numero di \godel 
di un teorema di $K_1$; cioè $b$ è in ${T_K}_2$ se e solo se
 $ \ulcorner ( \urcorner * c * \ulcorner \rightarrow \urcorner * b * \ulcorner ) \urcorner $ è in ${T_K}_1$. \newline
Quindi se ${T_K}_1$ fosse ricorsivo allora per sostituzione anche ${T_K}_2$ sarebbe ricorsivo, il che è in contraddizione con
l'ipotesi; di conseguenza $K_1$ è indecidibile.
\end{proof}

\begin{prop}
\label{p112}
Sia $K$ una teoria del primo ordine con gli stessi simboli di $HA$. Se K è compatibile con $Q$,
 allora $K$ è indecidibile. 
\end{prop}

\begin{proof}
Essendo per ipotesi $K$ compatibile con $Q$, allora la teoria $K \cup Q$ è un'estensione consistente di $Q$.
Quindi dato che assumiamo $Q$ indecidibile per la proposizione \ref{piii},
 allora $K \cup Q$ sarà anch'essa indecidibile.
Ma dato che $K \cup Q$ è un'estensione finita di $K$ allora per la proposizione \ref{p111}
anche $K$ è indecidibile.

\end{proof}

\begin{prop}
\label{p113}
Sia $P_{HA}$ il calcolo dei predicati con gli stessi simboli di HA. Allora $P_{HA}$ è indecidibile.
\end{prop}

\begin{proof}
Si osserva che $P_{HA} \cup Q = Q$. Quindi $P_{HA}$ è compatibile con $Q$; di conseguenza dalla proposizione \ref{p112}
si deduce che $P_{HA}$ è indecidibile.

\end{proof}

\section{Teorema di Church}

\begin{defi}
Denotiamo con $PF$ il calcolo predicativo totale del primo ordine, che contiene:
\begin{itemize}
 \item [(a)] tutte le lettere predicative $A^n_j$;
 \item [(b)] tutte le lettere funzionali $f^n_j$;
 \item [(c)] tutte le costanti individuali $a_j$.
\end{itemize}
\end{defi}

\begin{defi}
 Denotiamo con $PP$ il calcolo predicativo puro del primo ordine che contiene tutte le lettere
predicative, ma non lettere funzionali o costanti individuali.
\end{defi}

\begin{lem}
\label{lmmm}
Esiste una funzione ricorsiva $h$ tale che, per ogni formula $\varphi$ di $PF$ con numero di \godel
$u$, esiste una formula $\varphi'$ di $PP$ con numero di \godel $h(u)$ tale che $\varphi$ è dimostrabile
in $PF$ se e solo se  $\varphi'$ è dimostrabile in $PP$.
\end{lem}

%facile verificare
\begin{proof}
Sia $\varphi$ una formula di $PF$. Si procede associando lettere predicative distinte
che non occorrono in $\varphi$ alle lettere funzionali ed alle costanti individuali 
che vi compaiono, più precisamente:
\begin{itemize}
 \item [(1)] a lettere funzionali distinte $f^n_j$ in $\varphi$ associamo lettere predicative
distinte $A^{n+1}_r$ che non occorrono in $\varphi$;
 \item [(2)] a costanti individuali distinte $a_j$ in $\varphi$ associamo lettere predicative
distinte $A^1_k$ che non occorrono in $\varphi$.
\end{itemize}
\begin{enumerate}
 \item Sia $z$ la prima variabile non in $\varphi$; otteniamo allora $\varphi^{\ast}$ da $\varphi$ rimpiazzando
tutte le occorrenze di $a_j$ in $\varphi$ con $z$. Si costruisca ora la formula
$$\varphi_1: \; (\exists z) A^1_k(z) \rightarrow (\exists z) (A^1_k(z)\wedge \varphi^{\ast}) $$
dove $A^1_k$ è la lettera predicativa associata ad $a_j$. \newline
A questo punto è facile verificare che $\varphi$ è logicamente valida se e solo se
$\varphi_1$ è logicamente valida. Si prosegua con simili sostituzioni fino ad 
ottenere una formula $\psi$ senza costanti individuali e tale che $\varphi$ è
logicamente valida se e solo se $\psi$ è logicamente valida.
\item Prendiamo ora il termine $f^n_l(t_1,...,t_n)$ più a sinistra in $\psi$, dove $t_1,...,t_n$
non contengono lettere funzionali. Sia $w$ la prima variabile non in $\psi$, otteniamo allora
 $\psi^{\ast}$ da $\psi$ rimpiazzando $f^n_l(t_1,...,t_n)$ con $w$; ora similmente a 
quanto fatto al passo precedente costruiamo la formula
$$\psi_1: \; (\exists w) A^{n+1}_r(w,t_1,...,t_n) \rightarrow (\exists w) (A^{n+1}_r(w,t_1,....t_n)\wedge \psi^{\ast})$$
dove $A^{n+1}_r$ è la lettera predicativa corrispondente a $f^n_l$. Si può dunque verificare facilmente che
$\psi$ è logicamente valida se e solo se $\psi_1$ è logicamente valida. Si ripetano ora
le stesse trasformazioni finchè non si raggiunga una formula $\varphi'$ che non contenga 
lettere funzionali; a questo punto $\varphi'$ è una formula di $PP$ ed è logicamente valida
se e solo se $\varphi$ è logicamente valida.
\end{enumerate}
Per il teorema di completezza di \godel\footnote{Il teorema di completazza di \godel
afferma che in ogni calcolo predicativo del primo ordine i teoremi sono esattamente
le formule logicamente valide.} abbiamo che $\varphi$ è logicamente valida 
se e solo se $\vdash \!\! _{PF} \; \varphi$, e $\varphi'$  logicamente valida 
se e solo se $\vdash \!\! _{PP} \; \varphi'$. Possiamo finalmente definire
$$h(u)=\left\{ \begin{array}{ll}
  \ulcorner \varphi' \urcorner & se\ u\ \grave{e}\ il\ numero\ di\ G\ddot{o}del\ di\ una\ formula\ di\ PF;\\
  0 & altrimenti.\end{array}\right.$$
Si osserva che $h$ è effettivamente computabile, ricorsiva, e tale che per ogni formula
 $\varphi$ di $PF$ con numero di \godel
$u$ essa restituisce $h(u)$  numero di \godel di una formula $\varphi'$ di $PP$  tale che $\varphi$ è dimostrabile
in $PF$ se e solo se  $\varphi'$ è dimostrabile in $PP$.
\end{proof}

\begin{thm}
\label{thmch}
(Church, 1936) $PF$ (1) e $PP$ (2) sono indecidibili.
\end{thm}

\begin{proof}
Procediamo analizzando una parte per volta:
 \begin{itemize}
  \item [(1)] Per il teorema di completezza di \godel una qualsiasi formula $\varphi$ di $P_{HA}$
è dimostrabile in $P_{HA}$ se e solo se essa è logicamente valida, e $\varphi$ è dimostrabile in 
$PF$ se e solo se essa è logicamente valida, ovvero:
$$ \vdash \!\! _{P_{HA}}  \varphi \;\;\; sse \;\; \vdash \!\! _{PF} \; \varphi. $$
Questo significa che se la formula $\varphi$ è dimostrabile in $P_{HA}$ essa a maggior ragione
sarà dimostrabile in $PF$; d'altra parte la dimostrazione di $\varphi$ in $PF$ non avrà
bisogno di più simboli di quelli utilizzati in $P_{HA}.$ \newline
Osserviamo ora che l'insieme $Frm_{P_{HA}}$ dei numeri di \godel di formule di $P_{HA}$ è ricorsivo,
perciò vale
$$T_{P_{HA}} = T_{PF} \cap  Frm_{P_{HA}}$$
in cui $T_{P_{HA}}$ è l'insieme dei numeri di \godel dei teoremi di $P_{HA}$ e
$T_{PF}$ è l'insieme dei numeri di \godel dei teoremi di $PF$. Se $T_{PF}$
fosse ricorsivo allora anche $T_{P_{HA}}$ sarebbe ricorsivo, in contraddizione a quanto 
dimostrato dalla proposizione \ref{p113}. \newline 
Si conclude quindi che $PF$ è indecidibile.
  \item [(2)] Per il lemma \ref{lmmm} $u$ è in $T_{PF}$ se e solo se $h(u)$ è in $T_{PP}$.
Dal momento che $h$ è ricorsiva, la ricorsività di $T_{PP}$ implicherebbe la ricorsività 
di $T_{PF}$, in contraddizione di quanto appena dimostrato in (1). Perciò $T_{PP}$ non è
ricorsivo, cioè, $PP$ è indecidibile. 
 \end{itemize}
\end{proof}

\section{Conclusioni}
\label{chrcon}

In letteratura esistono diverse varianti della dimosrazione del teorema di Church, delle quali
menzioniamo qui le più fondamentali come appendice della nostra trattazione. La prima versione
del teorema,
sviluppata e pubblicata appunto da Alonzo Church nel 1935-36 fu sviluppata utilizzando il formalismo del 
$\lambda$-calcolo e si basa fortemente sul precedente lavoro di Stephen Kleene. 
Nel 1937 Alan Turing dimostrò
indipendentemente lo stesso risultato utilizzando il processo di riduzione per ridurre l'halting
problem (o problema della fermata) per una macchina di Turing all'\ents. Il lavoro di entrambi è
in ogni caso fortemente influenzato dai lavori di \godel relativi al processo di aritmetizzazione
ed ai teoremi di incompletezza.
$$$$

Come già accennato, se si accetta la tesi di Church allora dire \textit{``ricorsivamente indecidibile''} 
(espresso semplicemente da \textit{``indecidibile''} nella nostra trattazione) equivale a dire \textit{``effettivamente indecidibile''}.
In particolare quindi il teorema di Church (\ref{thmch}) afferma che non esistono procedimenti di
decisione ne per il calcolo puro dei predicati $PP$ e neppure per il calcolo predicativo totale $PF$.
Questo implica che non esiste nessun metodo effettivo per determinare quando una data formula
è logicamente valida. \newline
Contrariamente a quanto ci si potrebbe aspettare però si può dimostrare che il calcolo predicativo
monadico puro (costituito da quelle formule del calcolo predicativo puro che contengono esclusivamente
lettere predicative a un argomento) è effettivamente decidibile. In un certo senso questo è il
miglior risultato ottenibile, in quanto si può dimostrare anche (Kalmàr, 1936; Church, 1956) che non esiste
alcun procedimento di decisione per la validità (o dimostrabilità) di formule che contengono solo lettere predicative binarie.



