%\chapter{Conseguenze}

\begin{abstract}
Queste note sono divise in tre momenti: dopo un inquadramento
storico del problema dei fondamenti della matematica prenderemo in
con\-si\-de\-ra\-zio\-ne alcuni aspetti del pensiero di David Hilbert esponendo
in particolare il suo programma fondazionale, quindi parleremo delle
conseguenze fon\-da\-zio\-na\-li dei Teoremi di Incompletezza di G\"odel sottolineando le difficoltà in cui si imbattè il programma di Hilbert a seguito di questi.
\end{abstract}



\section{Introduzione all'epoca}

Il diciannovesimo secolo fu un'epoca di grandi trasformazioni e un'età durante la quale la matematica subì dei cambiamenti così profondi che non è esagerato parlare di una vera e propria seconda nascita della materia così come nell'età greca c'era stata la prima. Analisi, geometria, algebra e logica furono completamente rivoluzionate.

L'analisi, dopo un secolo di travolgenti successi, attraversò la fase cosiddetta di \emph{rigorizzazione dell'analisi}, durante la quale i concetti di funzione, serie, integrale e continuità vennero affinati da parte di Cauchy, Fourier e Riemann, fino a li\-be\-rar\-li dall'uso degli infinitesimi ed infiniti attuali, tramite il concetto di li\-mi\-te. Un passo importante di tale processo fu la definizione rigorosa, nel 1972, dei numeri reali ad opera di Weierstrass, Dedekind e Cantor.

In geometria, nel tentativo di dimostrare la dipendenza del quinto postulato della geometria euclidea dagli altri quattro, si assistette alla nascita delle geometrie non-euclidee con i lavori di Gauss, Lobacewskij e Bolyai. La sco\-per\-ta di tali geometrie metteva così in crisi la nozione di intuizione geometrica e l'idea che gli assiomi debbano necessariamente codificare un aspetto univoco ben determinato della nostra concezione dello spazio.

L'algebra assunse una forma completamente astratta: importanti furono gli sviluppi in teoria delle equazioni ad opera di Galois e il lavoro di Grassman sulle algebre che generalizzano i quaternioni di Hamilton.\\
Si assistette inoltre alla nascita della scuola britannica di algebristi, tra i quali ricordiamo Hamilton, Cayley, Boole e Sylvester.\\
Di notevole portata furono, infine, le interazioni tra l'algebra e le altre discipline della matematica: in particolare, i moderni metodi algebrici sostituirono i metodi sintetici in geometria e permisero lo sviluppo dello studio delle funzioni ellittiche (con Kummer, Dedekind a Abel) e dell'analisi (con Dirichlet, Jacobi, Riemann e Hamilton).

La logica subì una rigorosa definizione dei suoi linguaggi formali, in particolare Frege e Peano provvidero alla definizione del linguaggio della logica dei predicati del primo ordine.\\
Si diede avvio inoltre allo studio algebrico della logica, con Boole e De Morgan, e si assistette alla nascita della teoria dei modelli.

Questa rivoluzione delle scienze matematiche portò ad un acceso dibattito sulla definizione dell'oggetto proprio degli studi matematici: la tradizionale definizione della matematica quale scienza della grandezza, della misura e della quantità non era più adeguata!

Strettamente legata a questa esigenza di caratterizzazione della scienza ma\-te\-ma\-ti\-ca, vi fu un'esigenza di unificazione del sapere matematico, ormai fra\-zio\-na\-to in moltissime teorie interdipendenti.

Tale obiettivo fu raggiunto attraverso quella che ancor oggi è una delle teorie più discusse e interessanti di tutta la matematica, ossia la teoria degli insiemi, ad opera del matematico tedesco Georg Cantor.

La potenza del linguaggio insiemistico portò alla nascita di nuove teorie (topologia, analisi funzionale, teoria della misura di Lebesgue, geometria algebrica,...) e al rinnovamento di teorie già affermate (analisi reale, algebra astratta,...).

D'altra parte, la scoperta a cavallo del 1900, da parte di Cantor e Russell, dell'esistenza di paradossi all'interno della teoria degli insiemi, aprì il periodo di \emph{crisi dei fondamenti}.
Visto il ruolo fondazionale attribuito alla teoria degli insiemi, la comparsa dei paradossi rendeva dunque pressante il bisogno di una sicura fondazione della matematica.

Tre furono i principali indirizzi circa le proposte fondazionali: il \emph{logicismo} con Frege e Russell, l'\emph{intuizionismo} con Brouwer, il \emph{formalismo} con Hilbert.\\
\bigskip


\section{Il Formalismo}

Per l'importanza dei suoi risultati nelle discipline più svariate, per la profondità delle sue intuizioni e per l'influenza che esercitò sui suoi contemporanei, David Hilbert è indubbiamente una delle figure chiave della storia della matematica, ideologo e principale esponente del Formalismo.

Hilbert venne a contatto con numerosi matematici illustri della sua epoca, come Karl Weierstrass, Richard Dedekind e Georg Cantor, e a G\"ottingen rappresentò per quattro decenni la figura guida di un'intera scuola: tra i suoi studenti vi furono matematici del calibro di Herman Weyl, Felix Bernstein, Richard Courant, Ernst Zermelo; John Von Neumann e Paul Bernays erano i suoi assistenti, e figure importanti come Emmy Noether, Edmund Landau, Alonzo Church frequentavano lo stesso ambiente.

Oltre a questo ruolo di indirizzo, Hilbert fornì in prima persona significativi contributi ai campi di ricerca più disparati, dalla teoria degli anelli a quella dei numeri, dalla geometria alla fisica matematica e all'analisi funzionale, oltre che alla logica e al problema dei fondamenti.
Gli interessi matematici di Hilbert furono vastissimi, di modo che è difficile trovare un settore della matematica nel quale egli non sia intervenuto con contributi di tale peso ed importanza da indurre in essi profonde trasformazioni tematiche e metodiche.

In virtù della sua conoscenza enciclopedica della matematica, nel 1900, venne incaricato di tenere il discorso di apertura del Congresso internazionale dei ma\-te\-ma\-ti\-ci tenutosi a Parigi. In tale occasione propose alla comunità ma\-te\-ma\-ti\-ca un elenco di ventitré problemi aperti, da lui ritenuti i più rilevanti da affrontare nel secolo incipiente: tale lista ebbe una grossa influenza sulle direttrici della ricerca matematica nel primo Novecento, e i ventitré problemi vennero tutti affrontati in modo sistematico, e molti risolti almeno in parte.

Il lavoro di Hilbert illustra gran parte dei temi e degli interessi della ma\-te\-ma\-ti\-ca del diciannovesimo secolo: l'enfasi per la rappresentazione simbolica e per la caratterizzazione astratta, l'uso di metodi infinitari e non costruttivi e la ricerca di un'unità fondazionale. E proprio la ricerca di questa unità e di nuovi metodi matematici, lo renderà uno degli artefici della insiemizzazione della matematica e della conseguente rivalutazione della teoria degli insiemi sviluppata da Cantor.

In virtù di tutto ciò egli vide nei paradossi della teoria degli insiemi un pro\-ble\-ma drammatico e di cui era necessaria una chiara e precisa soluzione, al fine di restituire alla matematica quella certezza e inoppugnabilità che, a suo giudizio, le era propria. La sua risposta alla crisi dei fondamenti fu la proposizione di una nuova forma di metodo assiomatico: il \emph{metodo assiomatico formale}.
\newpage



\subsection{Il metodo assiomatico formale}

Il metodo assiomatico è quel modo di sviluppare una teoria, che consiste nel fissare certe proposizioni iniziali (dette \emph{assiomi} o \emph{postulati}) e da queste, procedendo per deduzione, ottenere nuove proposizioni. Questo approccio è stato utilizzato fin dall'antichità e l'esempio più illustre è rappresentato dagli \emph{Elementi} di Euclide.

Fino a tutto l'Ottocento, l'idea soggiacente a questo metodo era la seguente: prendendo come assiomi proposizioni la cui verità sia evidente, e prestando attenzione ad usare solo modi di inferenza che preservino la verità, si ottengono proposizioni la cui verità è implicita nella verità degli assiomi, ed è pertanto garantita anche quando essa non è immediatamente evidente. In altri termini, il metodo assiomatico consisteva nel dimostrare la verità di certe proposizioni riducendola alla verità di altre proposizioni prefissate considerate evidenti.

Questa visione degli assiomi come proposizioni ``evidentemente vere'' per\-si\-stet\-te fino alla fine del XIX secolo. Kant, per esempio, riteneva che gli assiomi della geo\-me\-tria euclidea fossero proposizioni vere \emph{a priori}, cioè esprimessero verità inerenti alla nostra modalità di percezione dei fenomeni.

Nell'Ottocento, però, si comprese che negando il quinto postulato di Euclide (il celebre \emph{postulato delle parallele}) non si otteneva alcuna contraddizione, in quanto gli assiomi ottenuti sarebbero risultati veri semplicemente modificando l'interpretazione intuitiva dei termini `punto' e `retta', come dimostrato da Beltrami e da molti altri dopo di lui.\\
Questo importantissimo esempio concreto portò a prendere pienamente coscienza della dipendenza della nozione di verità dall'interpretazione intesa dei concetti coinvolti nella proposizione.

\`{E} in questo contesto che si inserisce nel 1899 la pubblicazione dei \emph{Grundlagen der Geometrie} (\emph{Fondamenti della geometria}) di Hilbert, un'opera che ha avuto una profonda influenza sulla matematica del Novecento.

Hilbert mostra di essere pienamente cosciente del fatto che, sebbene la nozione di verità di una proposizione dipenda dall'interpretazione intesa, è possibile concepire la relazione di consequenzialità logica in modo tale che essa risulti indipendente da tale interpretazione. In altre parole, le inferenze ammissibili devono basarsi esclusivamente sulla forma logica delle relazioni tra concetti, e non devono dipendere in nessun modo dal significato intuitivo attribuito a tali concetti (ciò che Hilbert chiama `intuizione geometrica'). All'interno della teoria, le nozioni si intendono implicitamente definite dagli assiomi, e le proprietà che è lecito usare nelle inferenze sono tutte e sole quelle espresse dagli assiomi.

Procedendo in questo modo, si ottengono dimostrazioni che stabiliscono non più la \emph{verità} della conclusione, ma il fatto che, data una \emph{qualunque} interpretazione che renda veri gli assiomi della teoria, questa deve anche rendere vera la conclusione. Che questo fosse ciò che Hilbert aveva in mente è chiarissimo nella sua celebre affermazione che ``si deve sempre poter dire al posto di `punti, rette, piani', `tavoli, sedie, boccali di birra' ''.

La visione della matematica che emerge da quest'opera è quella del `se... allora': la matematica consiste nell'esplorazione delle conseguenze logiche di certe assunzioni. Se la matematica deve essere assolutamente certa, la validità dei suoi teoremi non può in alcun modo dipendere dalla configurazione contingente del mondo. Il problema della verità delle proposizioni (relativamente ad una data interpretazione) è invece empirico e pertanto giace al di fuori del dominio della matematica.
\bigskip

Il metodo assiomatico era sostenuto da Hilbert come `una procedura ge\-ne\-ra\-le per il pensiero scientifico', applicabile ad ogni teoria la cui costruzione logica possa essere basata su un numero limitato di proposizioni fondamentali.\\
\`{E} in questo senso che Hilbert riconosce alla matematica, identificata con il metodo assiomatico, un valore universale, affermando che:

\emph{nella loro parte teorica [le scienze] si dispiegano direttamente all'interno della matematica. [...] Tutto ciò che può essere oggetto del pensiero scientifico, non appena è maturo per la formazione di una teoria, cade sotto il metodo assiomatico e per suo tramite sotto la matematica. Progredendo verso livelli sempre più profondi di assiomi otteniamo anche illuminazioni sempre più profonde sulla natura del pensiero scientifico e diveniamo sempre più consapevoli dell'unità del nostro sapere. Nel segno del metodo assiomatico la matematica sembra essere chiamata ad un ruolo guida in tutto ciò che è scienza.}
\bigskip

\subsection{Il finitismo}

Attraverso il metodo assiomatico formale, Hilbert ritiene di essere in grado di dare alla matematica una fondazione capace di completa chiarezza e certezza.  La necessità di una fondazione, per Hilbert, non era affatto dettata dal bisogno di ``rinforzare'' i risultati della matematica stessa, quanto dal desiderio di spiegare una volta per tutte cosa sia la matematica e come si origina la sua certezza.

Condizione necessaria affinché si possa affermare la certezza delle proposizioni matematiche è che esse riguardino oggetti che sono completamente conoscibili, dunque finiti e concretamente presentabili, e le cui relazioni rilevanti al discorso siano, diremmo oggi, decidibili in senso intuitivo. Ora, i segni che si usano in teoria dei numeri soddisfano questi requisiti, e viceversa è ra\-gio\-ne\-vo\-le assumere che un insieme di oggetti della tipologia descritta possano essere legittimamente chiamati un \emph{sistema di segni}.

Hilbert nega che oggetti infiniti, quali i numeri reali, siano immediatamente rappresentabili alla nostra conoscenza e quindi ammissibili come oggetto di pensiero contenutistico. Infatti, egli afferma che ``l'infinito non è rea\-liz\-za\-to in nessun luogo; non esiste in natura, né è ammissibile come fondamento del nostro pensiero razionale [\dots] Pertanto, le operazioni con l'infinito devono essere rese certe all'interno del finito''.

Saranno dunque quegli oggetti finiti e concretamente esibibili chiamati \emph{segni} ad essere oggetto di quell'attività che Hilbert chiama \emph{matematica contenutistica}, \emph{finitistica} o \emph{metamatematica}. Nel contesto di un sistema di segni, le singole proposizioni matematiche hanno un valore di verità.
Inoltre tali valori di ve\-ri\-tà sono determinabili con procedure finite mediante un semplice confronto di simboli.
\bigskip

\subsection{La proposta formalista}

Hilbert ritiene perciò che la completa certezza e verificabilità degli enunciati matematici si possano ottenere solo all'interno di un contesto in cui gli oggetti siano rigorosamente finiti e supervisionabili, e le loro relazioni immediatamente verificabili. D'altra parte, egli vuole giustificare \emph{tutta la matematica}, anche quella transfinita, ponendola su basi certe.

Ora, come è possibile conciliare questi due \emph{desiderata}? Certo i numeri reali \emph{non} sono in generale oggetti finitamente presentabili, né le loro relazioni (ad esempio $x<y$) sono direttamente verificabili, cioè decidibili. Dunque come è possibile ottenere assoluto rigore nel campo dell'analisi?

La proposta di Hilbert è di utilizzare il metodo assiomatico.\\
Supponiamo infatti di aver fondato una teoria transfinita, quale ad esempio l'ana\-li\-si, in modo assiomatico: ora, sebbene gli oggetti di cui la teoria stessa parla non siano oggetti finiti, tanto le proposizioni della teoria stessa (in particolare gli assiomi) che le dimostrazioni lo sono.\\
Inoltre, e questa è una delle profondissime intuizioni di Hilbert, i modi di inferenza disponibili nella teoria potranno sempre essere assegnati in modo chiaro come parte della teoria stessa, sottoforma di regole.\\
Pertanto, in un contesto assiomatico di questo tipo, sarà possibile stabilire in modo effettivo se una data sequenza finita di proposizioni costituisca o meno una dimostrazione di una determinata proposizione.

\`{E} dunque possibile sviluppare l'analisi (e analogamente le altre teorie infinitarie) in maniera finitistica prendendo come oggetto del nostro ragionamento non gli ``enti infiniti'' di cui la teoria parla, ma gli enti finiti che costituiscono la teoria stessa, cioè gli assiomi, le dimostrazioni, i teoremi. 

Hilbert si serve dunque dei metodi della logica matematica per tradurre le componenti di una teoria assiomatica in oggetti della matematica finitaria, cioè in segni. In questo modo, specificata una teoria formale (che consiste di assiomi e regole) è possibile verificare effettivamente, cioè con un numero finito di operazioni se una certa figura sia o meno una dimostrazione di una formula.

%Hilbert chiama quindi \emph{matematica} ciò che avviene all'interno di un si\-ste\-ma formale. I sistemi formali stessi, che consistono di oggetti finiti che altro non sono che segni, sono quindi oggetto della matematica finitistica, che viene pertanto detta \emph{metamatematica}.

L'intento di Hilbert non è quello di convincere i matematici a sviluppare le loro teorie all'interno di un sistema formale, ma solo quello di mostrare che la matematica transfinita può essere riformulata all'interno di un tale sistema ed è dunque possibile comprenderla completamente all'interno del finito, senza invocare l'esistenza di oggetti infiniti. In questo modo si ottiene inoltre un metodo che permette di controllare con assoluta certezza la correttezza di un risultato ottenuto anche con metodi transfiniti.

In definitiva, il \emph{programma formalista} prevede la completa formalizzazione dell'aritmetica, spogliando segni logici e matematici del loro significato, attraverso la costruzione di un {\em sistema assiomatico formale}, cioè un sistema costituito di soli segni "`"`senza significato"'"' e regole sintattiche per legare i segni tra loro.\\
In tal modo la metamatematica è un linguaggio con cui si parla della matematica dall'esterno, cioè parlando dei segni che determinano formule o espressioni ma\-te\-ma\-ti\-che: la \emph{matematica}, infatti, tratta delle formule e delle espressioni considerando il loro significato e l'interpretazione che noi gli diamo, mentre la \emph{metamatematica} tratta formule ed espressioni considerandole come stringhe di simboli privi di significato.

Questa operazione di formalizzazione ci permette di affrontare lo studio della matematica in maniera diversa.\\
Infatti, è possibile operare direttamente su queste stringhe di segni con una serie di manipolazioni di simboli prefissate e ricavare, a partire da alcune formule iniziali, delle nuove espressioni, senza preoccuparci di quale sia il significato di ciò che stiamo ottenendo o se si tratti di un'affermazione vera. Questa serie di espressioni potrà poi essere interpretato in una dimostrazione.\\
In questo caso, le formule iniziali vengono interpretate come gli assiomi o le ipotesi da cui si parte per la dimostrazione, le manipolazioni di simboli sono interpretate come le regole logiche di deduzione, mentre la stringa finale a cui si arriva è il teorema che si è dimostrato. 



\section{Il problema della coerenza}

Si è visto come la possibilità di sviluppare teorie di strutture transfinite con metodi finitari sia ottenuta a scapito della capacità della matematica di costruire esplicitamente una qualche rappresentazione degli oggetti del proprio discorso.\\
Hilbert mostra dunque come l'astrazione ci permetta di sviluppare teorie anche in assenza di un dominio del discorso e di procedere secondo quella che chiama una filosofia del `come se', nel senso che si ragiona `come se' oggetti del tipo descritto dagli assiomi fossero dati, prescindendo dalla possibilità concreta di costruire tali oggetti.\\
Attraverso i sistemi formali, la matematica si configura come ``teoria delle forme'' astratte, cioè dei sistemi di connessioni logiche tra concetti.

Questa prospettiva sulla matematica apre due tipi di problemi.
\begin{itemize}
\item Quando dovremmo studiare un certo sistema formale? Cioè, quali sistemi formali sono matematicamente significativi e perché?
\item Quando possiamo studiare un certo sistema formale? Cioè quali sistemi formali rappresentano un sistema di connessioni logiche che sia possibile almeno in principio, cioè che possa essere concepito senza generare contraddizioni?
\end{itemize}

La prima di queste domande non sembra aver mai preoccupato Hilbert e Bernays, ma la mancanza di un'elaborazione su questo tema è stato spesso oggetto di critiche alle posizioni formaliste.

Il secondo problema, quello di garantire una volta per tutte la coerenza di certe teorie -in particolare dell'aritmetica e dell'analisi- è invece considerato da Hilbert e Bernays il problema cruciale della metamatematica.

L'idea di Hilbert è che ogniqualvolta si voglia studiare un certo sistema di connessioni logiche tra concetti ci si può porre in un sistema formale, il che permette di sviluppare una teoria pienamente rigorosa, ma è importante garantire che tale teoria ``stia in piedi'', cioè non si riveli a posteriori insensata anche dal punto di vista meramente combinatorio, permettendo di derivare qualunque conclusione.

Ovviamente -sostiene Bernays- c'è una schiacciante evidenza empirica della coerenza dell'aritmetica e dell'analisi, che risiede nell'ampio uso che si è fatto di tali teorie senza che mai sorgesse l'ombra di una contraddizione. Tuttavia, grazie alla traduzione di tali teorie in sistemi formali, cioè oggetti della ma\-te\-ma\-ti\-ca finistica, l'enunciato di coerenza della teoria diviene una proposizione della metamatematica.

Ora, qualora gli assiomi di una teoria siano soddisfatti da una certa struttura finitisticamente presentabile e le regole di derivazione preservino la verità rispetto a tale struttura, la coerenza della teoria è garantita: infatti, tutti i teoremi de\-ri\-va\-bi\-li nella teoria saranno veri quando interpretati in tale struttura, e poiché una contraddizione non può mai essere vera per definizione, la teoria non può derivare contraddizioni.
Una struttura come quella descritta è detta un \emph{modello finito} della teoria, e ciò che si è detto è che una teoria che ammetta modelli finiti è coerente.

Il problema è che teorie quali l'aritmetica, l'analisi e la teoria degli insiemi non possono per loro stessa natura ammettere modelli finiti. L'unico modo per affrontare il problema della coerenza è quindi quello diretto: esaminare gli assiomi e le regole della teoria e mostrare, con ragionamenti sintattici rigorosamente finitistici, che essi non permettono di derivare la formula $1\neq 1$ (o un'altra qualunque contraddizione fissata).

Certo è possibile dare dimostrazioni di coerenza relativa: ad esempio, nei \emph{Grundlagen der Geometrie} la coerenza degli assiomi della geometria euclidea è dimostrata costruendo un modello per tali assiomi basato sui numeri reali, ed è quindi ridotta alla coerenza dell'analisi.

Tuttavia, affinché il tutto stia in piedi si deve dimostrare in modo diretto la coerenza di almeno una teoria.
Hilbert si propone di dimostrare me\-ta\-ma\-te\-ma\-ti\-ca\-men\-te la coerenza dell'a\-ritme\-ti\-ca formalizzata, utilizzando metodi finitari. Cioè vuole dimostrare l'impossibilità di ottenere formule formalmente contraddittorie dagli assiomi, utilizzando però procedimenti che non fanno uso di un numero infinito di operazioni o proprietà delle formule, e che sono detti perciò \emph{finitari}.

Nel 1900, al secondo Congresso Internazionale dei Matematici, Hilbert presenta una lista di problemi aperti: il secondo di questi è, appunto, la dimostrazione della coerenza dell'aritmetica!
\`{E} così che Hilbert, nel corso degli anni Venti, insieme ai giovani Ackermann e Von Neumann, inizia a lavorare al problema della coerenza dell'aritmetica: qualora fosse riuscito a dimostrarla, ciò avrebbe segnato il trionfo del piano fondazionale formalista, noto come \emph{programma di Hilbert}.
\bigskip

\section{G\"odel}

Kurt G\"{o}del nacque a Br\"unn, in Moravia, il 28 aprile 1906. Conseguita la licenza liceale, si trasferì a Vienna per iniziare i suoi studi universitari, inizialmente con l'intenzione di studiare fisica teorica, ma poi dedicandosi alla matematica e alla filosofia, e concentrandosi infine sulla logica matematica.

Il decennio 1929-1939 fu un periodo di lavoro intenso, che produsse i principali risultati di G\"odel nel campo della logica matematica.\\
Nel 1930 egli cominciò ad approfondire il programma di Hilbert di stabilire con mezzi finitari la non contraddittorietà dei sistemi assiomatici formali per la ma\-te\-ma\-ti\-ca ed  è proprio cercando la soluzione a questo problema che arrivò, invece, a dimostrare che questa dimostrazione di coerenza è impossibile, nella forma in cui la immaginava Hilbert. 

Pubblicò, ancora venticinquenne, il suo risultato nel 1931 su una ri\-vi\-sta scien\-ti\-fi\-ca\footnote{Monatshefte f\"{u}r Mathematik und Physik}, con un articolo dal titolo "`"`Uber formal unentscheidbare Satze der \emph{Principia Mathematica} und verwandter systeme"'"' ("`"`Sulle proposizioni formalmente indecidibili dei \emph{Principia Mathematica} e dei sistemi affini"'"').

Le conclusioni e i caratteri del tutto originali di questi teoremi attirarono presto l'attenzione di molti intellettuali. Uno dei primi a riconoscere il potenziale significato dei risultati di incompletezza e ad incoraggiarlo a proseguire verso un loro più ampio sviluppo fu John von Neumann, che era tre anni più vecchio di G\"odel ma che si era già distinto per i brillanti risultati in teoria degli insiemi, teoria della dimostrazione, analisi e fisica matematica.
\bigskip



\section{Il Formalismo dopo i Teoremi di G\"odel}

Hilbert aveva concepito un ambizioso e articolato piano per mettere al sicuro tutta la matematica esistente, sviluppandola in modo perfettamente su\-per\-vi\-sio\-na\-bi\-le e dimostrandola libera da contraddizioni.

Ma Hilbert si era spinto anche oltre, congetturando che il metodo assiomatico fornisca i mezzi non solo per formulare, ma anche per risolvere ogni problema matematico. Tale pretesa può essere giustificata solo ritenendo che le teorie formali con cui la matematica ha a che fare siano \emph{complete}, cioè che dimostrino sempre o una formula o la sua negazione. Ciò è esplicitamente congetturato da Hilbert e Bernays per quanto riguarda l'aritmetica e l'analisi e sembra ci si aspettasse lo stesso anche per la teoria degli insiemi.

Ma le cose non stanno come Hilbert pensava: nel 1931 G\"odel dimostrò infatti che ogni teoria formale che estenda la teoria standard PA dell'aritmetica non può essere completa, dato che esistono nella teoria delle \emph{proposizioni indecidibili} che il sistema non è in grado né di dimostrare né di refutare.

Ma c'è di più: G\"odel mostrò che ogni proposizione della me\-ta\-ma\-te\-ma\-ti\-ca su un dato sistema formale è equivalente ad una proposizione aritmetica, traducibile in una formula nel linguaggio dell'aritmetica. In particolare, alla proposizione di consistenza (o coerenza) di una teoria $T$ cor\-ri\-spon\-de una formula aritmetica $\mathrm{Con}(T)$.\\ G\"odel dimostrò che ogni teoria $T$ che estenda la teoria standard dell'aritmetica non dimostra la formula $\mathrm{Con}(T)$.

Ne segue l'impossibilità di dimostrare con metodi finitistici la consistenza dell'aritmetica e, \emph{a fortiori}, la consistenza di una teoria all'interno della quale è possibile sviluppare l'aritmetica, quale quella degli insiemi.

Da ciò segue che una dimostrazione di coerenza di un tale sistema deve ne\-ces\-sa\-ria\-men\-te fare ricorso a qualche principio non contenuto in esso.\\
Quindi se si cerca una dimostrazione finitaria della coerenza di tale teoria, si deve estendere l'ambito della matematica finitaria oltre ciò che è esprimibile nell'aritmetica formale e questo effettivamente è stato fatto da Gerhard Gentzen che trovò nel 1936 una dimostrazione di coerenza per l'a\-ritme\-ti\-ca formale che fa uso dell'induzione fino ad un particolare numero ordinale transfinito, chiamato epsilon-zero.\\
Se poi dall'aritmetica si passa all'analisi matematica, cioè ad una teoria formale per i numeri reali, la dimostrazione di coerenza deve attendere il 1970 e richiede l'induzione fino ad un ordinale infinitamente più grande di epsilon-zero. 

Inoltre, come si è dedotto sopra, la teoria ZF non è in grado di dimostrare la propria coerenza, quindi per farlo abbiamo bisogno di un principio non dimostrabile in ZF stessa. Ora, da un lato è del tutto irragionevole ritenere che la matematica finitaria non sia contenuta nella teoria degli insiemi nella quale è di fatto contenuta tutta la matematica d'oggi e dall'altro, fatto ancora più rilevante, è attualmente del tutto inimmaginabile un principio matematico che trascenda la teoria degli insiemi e quindi risulta proprio impossibile allo stato attuale fornire una prova di coerenza di tale teoria.\footnote{Sambin, \emph{Alla ricerca della matematica perduta}.}
\bigskip

Il programma di Hilbert per mettere al sicuro la matematica dai paradossi garantendone la consistenza con mezzi ``al di sopra di ogni sospetto'' era in questo modo distrutto.

Nonostante questo duro colpo alla sua filosofia, il Formalismo si impose nella pratica matematica come fondazione dominante: oggi, gran parte della comunità matematica utilizza una teoria assiomatica, la teoria degli insiemi di Zermelo-Fraenkel con assioma di scelta (ZFC), come fondamento per il proprio lavoro.

Il successo del metodo assiomatico formale è probabilmente da attribuirsi al fatto che esso rimane ad oggi l'unica maniera in cui è possibile sviluppare in modo completamente chiaro e finitistico teorie transfinite, quali la teoria degli insiemi e l'analisi, preservandone tutti i risultati (anzi, favorendo salti nell'infinito sempre più arditi e slegati da ogni possibilità di interpretazione costruttiva).

A prescindere comunque dalla teoria degli insiemi e dal suo attuale successo, il Formalismo ha dato un grosso contributo alla matematica e alla sua filosofia per una ragione più generale.\\
Grazie al metodo assiomatico formale, infatti, fissate le premesse e le regole logiche, il procedimento di deduzione diviene un processo combinatorio e pertanto non controverso e anzi addirittura automatizzabile. In questo modo si garantisce una volta per tutte la possibilità di risolvere ogni disputa sulla correttezza dei teoremi formalizzati, e di ricondurre i disaccordi a una diversa (ma libera) scelta di princìpi (assiomi o regole logiche). 

