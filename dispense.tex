\documentclass[10pt,a4paper]{amsbook}

\usepackage{ucs}
\usepackage[utf8x]{inputenc}
\usepackage{amsmath}
\usepackage{amsfonts}
\usepackage{amssymb}
\usepackage{amsthm}
\usepackage[italian]{babel}
\usepackage{fontenc}
\usepackage[all]{xy}
\usepackage{graphicx}
\usepackage{prooftree}

\usepackage{hyperref}
\hypersetup{%
    colorlinks=true, linktocpage=true, pdfstartpage=3, pdfstartview=FitV,%
    breaklinks=true, pdfpagemode=UseNone, pageanchor=true,
    pdfpagemode=UseOutlines, plainpages=false, bookmarksnumbered,
    bookmarksopen=true, bookmarksopenlevel=1, hypertexnames=true,
    pdfhighlight=/O,%hyperfootnotes=true,%nesting=true,%frenchlinks,%
    %urlcolor=webbrown, linkcolor=RoyalBlue, citecolor=webgreen,
    %pagecolor=RoyalBlue,%
    % uncomment the following line if you want to have black links (e.g., for
    % printing)
    %urlcolor=Black, linkcolor=Black, citecolor=Black, pagecolor=Black,%
    pdftitle={Dispense di Logica 2},%
    pdfauthor={},%
    pdfsubject={},%
    pdfkeywords={},%
    pdfcreator={pdfLaTeX},%
    pdfproducer={LaTeX with hyperref and classicthesis}%
}

\usepackage{tikz}
\usetikzlibrary{shapes,arrows}
% Define block styles
\tikzstyle{decision} = [diamond, draw, fill=white, 
    text width=3.5em, text badly centered, node distance=2cm, inner sep=0pt]
\tikzstyle{block} = [rectangle, draw, fill=white, 
    text width=5em, text centered, minimum height=2em]
\tikzstyle{line} = [draw, -latex']
\tikzstyle{noblock}=[rectangle,
    text width=5em, text centered, minimum height=2em]

\newcommand{\roundentry}[1]{*++[o][F-] {#1}} 
\newtheorem{esempio}{Esempio}[chapter]
\newtheorem{oss}{Osservazione}[chapter]
\newtheorem{prop}{Proposizione}[chapter]
\newtheorem{defi}{Definizione}[chapter]
\newtheorem{lem}{Lemma}[chapter]
\newtheorem{thm}{Teorema}[chapter]
\newtheorem{extra}{Esercizio}
\newtheorem{nota}{Nota}
\newtheorem{programmi}{Definizione}[section]
\newtheorem{progabaco}[programmi]{Teorema}
\newtheorem{corol}{Corollario}[chapter]


\newenvironment{mylisting}
{\begin{list}{}{\setlength{\leftmargin}{1em}}\item\bfseries}
{\end{list}}

\newenvironment{mytinylisting}
{\begin{list}{}{\setlength{\leftmargin}{1em}}\item\tiny\bfseries}
{\end{list}}

\title{Dispense di Logica 2}
\author{}
\date{}

\begin{document}

\maketitle

\tableofcontents

\include{01_turing_registri}
\chapter{Funzioni Ricorsive}
\label{lemma diagonalizzazione}

Lo scopo di questo capitolo \`e quello di dimostrare l'equivalenza tra
i concetti di calcolabilit\`a con registri e calcolabilit\`a con
programmi.  Iniziamo dunque a definire i programmi.

\section{I programmi}
Nella sezione precedente abbiamo visto come il contenuto dei registri,
nelle macchine a registri, possa essere modificato attraverso
determinate \emph{istruzioni} che la macchina \`e in grado di
riconoscere. Una sequenza di queste semplici istruzioni costituisce un
\emph{programma}.


\begin{programmi}
Chiamiamo programma P una sequenza finita non nulla di istruzioni etichettate
\end{programmi}

Chiameremo $x_i$, $x_j$ e $x_k$ variabili dove $i,j,k \in \mathbb{N}$. Assumeremo come convenzione,
che l'output di $P$ deve essere messo (dal programmatore) nella variabile $x_0$.
Inoltre l'ennupla $a_1, a_2, ..., a_n \in \mathbb{N}^k $ fornita come input al nostro programma verrà memorizzato nelle prime n variabili $x_1, ..., x_n$.
Assumiamo anche che tutte le variabili utilizzare nel nostro programma siano automaticamente inizializzate a zero e scriveremo $P(a_1, a_2, ..., a_n)$ per dire che eseguiamo $P$ con l'input $a_1, a_2, ..., a_n$.

Le istruzioni del programma possono essere:

\begin{enumerate}
\item $x_j \leftarrow 0$ assegna ad $x_j$ il valore zero. (azzeramento)
\item $x_j \leftarrow x_i $ assegna il contenuto di $x_i$ a $x_j$ (trasferimento)
\item $x_j \leftarrow x_j+1$ aumenta il contenuto di $x_j$ di uno e 
  riassegna il risultato come valore a $x_j$ (successsore)
\item $\textrm{if }(condizione) $ then goto $\alpha$ else goto $\beta$ con
  $\alpha$ e $\beta$ etichette di istruzioni del programma. (salto)
\end{enumerate}
	
Esempi di condizioni possono essere il confronto fra due variabili, il confronto di
una variabile con un numero naturale ecc..

\begin{nota}
Nella (\emph{4}) come condizione \`e possibile utilizzare 
una qualunque formula atomica di un linguaggio che usi le relazioni
$=,\geq,<$ e le operazioni $+,-,successore,\ldots $ e le loro
composizioni.
\end{nota}



\begin{extra}
Si scriva un programma in cui sia presente un'istruzione del tipo
(\emph{4}) avente per condizione $x_i\neq x_j$. (\emph{Suggerimento}:
si pensi ad una funzione in $(x_i,\ldots ,x_j)$ il cui valore sia
diverso da $0$ quando $x_i\neq x_j$)
\end{extra}	

\begin{extra}
Mostrare, con qualche esempio, che \`e possibile sostituire alla
condizione della (\emph{4}) una qualsiasi espressione di
calcolo \newline booleano classico.
\end{extra}	

Per eseguire una \emph{computazione} la macchina deve essere
fornita di un \emph{programma} $P$ e di una \emph{configurazione
  iniziale} (input), che pu\`o essere, ad esempio, una sequenza $a_1, a_2,
...$ di numeri naturali.
Supponiamo che $P$ sia composto da $s$ istruzioni $I_1, I_2,...,
I_s$.

Allora la macchina inizia la computazione eseguendo $I_1$, poi $I_2$,
e cos\`i via, a meno che non si incontri un'istruzione di \emph{goto}
che fa saltare la computazione in una delle $s$ istruzioni. La
computazione si ferma quando, e solo quando non c'\`e una prossima
istruzione oppure incontriamo la parola chiave \textbf{stop}.
Possiamo illustrare questo attraverso un esempio.
	
\begin{esempio} Consideriamo il seguente programma con $s=6$ istruzioni:
\begin{mylisting}
$I_1$: $\textrm{if }x_1 > x_2$ then goto $I_2$ else goto $I_5$
  \\ $I_2$: $x_2 \leftarrow x_2+1$\\ $I_3$: $x_3 \leftarrow
  x_3+1$\\ $I_4$: $\textrm{if }x_1 \neq x_2$ then goto $I_2$ else goto
  $I_5$\\ $I_5$: $x_0 \leftarrow x_3$\\
\end{mylisting}
E con la seguente configurazione iniziale:
\begin{mylisting}
$x_0 = 0, x_1 = 9, x_2 = 7, x_3 = 0, ...$
\end{mylisting}
Durante il calcolo le variabili vengono modificate come segue:
\begin{mylisting}	
$x_0 = 0, x_1 = 9, x_2 = 7, x_3 = 0, ...$ ($I_1: x_1 > x_2$)\\ $x_0 =
  0, x_1 = 9, x_2 = 8, x_3 = 0, ...$ ($I_2$)\\ $x_0 = 0, x_1 = 9, x_2
  = 8, x_3 = 1, ...$ ($I_3$)\\ $x_0 = 0, x_1 = 9, x_2 = 8, x_3 = 1,
  ...$ ($I_4: x_1 \neq x_2$)\\ $x_0 = 0, x_1 = 9, x_2 = 9, x_3 = 1,
  ...$ ($I_2$)\\ $x_0 = 0, x_1 = 9, x_2 = 9, x_3 = 2, ...$
  ($I_3$)\\ $x_0 = 0, x_1 = 9, x_2 = 9, x_3 = 2, ...$ ($I_4: x_1 =
  x_2$)\\ $x_0 = 2, x_1 = 9, x_2 = 9, x_3 = 2, ...$ ($I_5$)\\
\end{mylisting}	
\end{esempio}

\begin{nota}		
Al momento non poniamo la nostra concentrazione su quale funzione
calcola questo programma, ma ci limitiamo ad illustrare in quale modo
opera un programma in senso puramente meccanico.
\end{nota}

\begin{extra}
Dare un programma che memorizza in $x_0$ 1 se $x_1 > x_2$ e 0
altrimenti.
\end{extra}	

Per comprendere il significato del programma e l'andamento della
computazione \`e spesso conveniente desciverlo in modo informale
attraveso un \emph{Flow Chart}, per esempio il flow chart
rappresentante il programma dell'Esempio 1.1 \`e dato in Figura
1. Notare che i test contenuti nei rombi rappresentano le istruzioni
\emph{if then else} (4) che hanno quindi due prosecuzioni in base al
risultato del test; mentre i rettangoli sono le istruzioni di
successore o assegnazione che continuano sempre con la prossima
istruzione.
		
\begin{figure}[h] 
\begin{tikzpicture}[node distance = 1.6 cm, auto]
 % Place nodes
   \node[noblock] (start) {START};
    \node [decision, below of=start] (I1) {$x_1 > x_2$};
    \node [block, below of=I1] (I2) {$x_2 \leftarrow x_2 + 1$};
    \node [block, below of=I2] (I3) {$x_3 \leftarrow x_3 + 1$};
    \node [decision, below of=I3] (I4) {$x_1 \neq x_2$};
    \node [block, below of=I4] (I5) {$x_0 \leftarrow x_3$};
   \node[noblock,below of=I5] (stop) {STOP};

    
    % Draw edges 
   \path [line] (start) -- (I1); \path [line] (I1) --
   node{yes}(I2); \path [line] (I1.east) -- ++(1.0,0) -- ++(0,-1) |-
   node [near start] {no}(I5.east); \path [line] (I2) -- (I3); \path
   [line] (I3) -- (I4); \path [line] (I4.west) -- ++(-.8,0) --
   ++(-.8,0) |- node [near start] {yes}(I2.west); \path [line] (I4)
   --node{no} (I5); \path [line] (I5) -- (stop);
\end{tikzpicture}
\caption{Flow Chart}   
	
\end{figure}
	
Dall'esempio 1 notiamo inoltre come il comando \emph{goto} nella
definizione dell'\emph{if} renda possibile l'esecuzione di cicli,
infatti l'istruzione $I_4$ riporta l'esecuzione all'istruzione $I_2$
tante volte fino a quando la condizione $x_1 \neq x_2$ non viene
soddisfatta.
	
Pi\`u in generale possiamo dire che il comando \emph{goto} pu\`o
essere utilizzato per simulare quello che nei linguaggi ad alto livello conosciamo come 
ciclo \emph{for}. 
Infatti con $n$ fissato il ciclo:\\
$$ \left[
\begin{array}{l}
for \; $i = 1$ \; to \; $n$\\ \ \ \ \ \left[P\;
programma]\right. \\ next \; $i$\\
\end{array} \right.
\
$$ 
viene implementato dal programma nell'Esempio 1.2 dove $n$ è un opportuna variabile
che riceve in input il valore di $n$.

\newpage

\begin{esempio}

\begin{minipage}[c]{.50\textwidth}
\begin{mylisting}
 $I_1: x_i \leftarrow 0 $\\ 
 $I_2: x_i \leftarrow x_i + 1 $ \\
 $I_3: $ if $ n<x_i $ then goto stop else goto $I_4$ \\
 $I_4:  \left[ P\; programma \right ] $ \\
 $I_5: x_i \leftarrow x_i+1 $ \\
 $I_6 $ if $n < x_i $ then \\ \hspace{\stretch{1}} goto stop  else goto $I_4$\\

\end{mylisting}
\end{minipage}
\hspace{5mm}%
\begin{minipage}[c]{.50\textwidth}
\begin{tikzpicture}[node distance = 1.6 cm, auto]
    % Place nodes 
\node[noblock] (start) {START}; 
\node [block, below of=start] (I1) {$x_i \leftarrow 0$};
\node [block, below of=I1] (I2) {$x_i \leftarrow x_i+ 1$};
\node [decision, below of=I2] (I3) {$n < x_i$};
\node [block, below of=I3] (I4) {$ \left [ P \right] $};
\node [block, below of=I4] (I5) {$x_i \leftarrow x_i+ 1$};
\node [decision, below of=I5] (I6) {$n < x_i$};
\node[noblock,below of=I6] (stop) {STOP};

    % Draw edges
\path [line] (start) -- (I1);
\path [line] (I1) -- (I2);
\path [line] (I2) -- (I3);
\path [line] (I3.west) -- ++(-.8,0) -- ++(-.8,0) |- node [near start] {yes}(stop.west);
\path [line] (I3) -- node{no} (I4);
\path [line] (I4) -- (I5);
\path [line] (I5) -- (I6);
\path [line] (I6.east) -- ++(+.10,0) -- ++(+.8,0) |- node [near start] {no}(I4.east);
\path [line] (I6) --node{yes} (stop);

\end{tikzpicture}
\end{minipage}

\end{esempio}

\vspace{5mm}% 
	
\newpage

Ci chiediamo adesso, le quattro istruzioni che abbiamo introdotto sono tutte necessarie? 
E' possibile toglierne qualcuna e continuare a calcolare le stesse funzioni?
E immediato verificare che l'istruzione di trasferimento $x_j \leftarrow x_i$ può
essere scritta in funzione delle altre. Vediamo l'esempio:

\begin{esempio} utilizziamo le altre tre istruzioni per simulare il comportamento dell' istruzione di trasferimento:
\begin{mylisting}
$I_1$: $\textrm{if }x_j = x_i$ then goto stop else goto $I_2$\\
$I_2$: $x_j \leftarrow 0$\\
$I_3$: $x_j \leftarrow x_j +1$\\ 
$I_4$: $\textrm{if }x_j = x_i$ then goto stop else goto $I_2$\\
\end{mylisting}
\end{esempio}

Abbiamo quindi introdotto l'istruzione di trasferimento al solo fine di semplificare 
la scrittura dei nostri programmi.

Un altra domanda che sorge è: "senza l'istruzione di salto
riusciamo a calcolare le stesse funzioni di quante ne calcoleremo con tutte e quattro? Ebbene si può dimostrare che 
le uniche funzioni che possiamo calcolare senza l'istruzione di salto sono del tipo

\begin{enumerate}
\item $f(x) = m \, \, m \in  \mathbb{N} $ 
\item $f(x) = x+ m \, \,  \forall x ,  m \in  \mathbb{N} $ 
\end{enumerate}

\begin{extra}
Dimostrare che senza l'istruzione di if .. goto calcoliamo solo funzioni del tipo (1) e (2).
\end{extra}

\begin{extra}
Dimostrare che senza l'istruzione di if .. goto calcoliamo solo funzioni che sono totali.
\end{extra}

Questo fatto ci mostra che con tutta la sua semplicità l'istruzione di salto 
è importantantissima per calcolare quanto c'e di calcolabile e ciò si traduce nel fatto che
se il nostro linguaggio di programmazione preferito non disponesse dell' istruzione di salto 
esso servirebbe a ben poco.

Notiamo anche il fatto che non tutti i moderni linguaggi di programmazione 
dispongono dell'istruzione goto dovuto al fatto che l'uso questa istruzione 
è generalmente considerato indice di cattiva programmazione
ma ciò è stato rimpiazzato dall'uso delle cosiddette strutture di controllo iterative (for e while)
che implicitamente permettono di saltare in maniera controllata da un punto 
all'altro del programma.


\subsection{Funzioni calcolabili con un programma}

Abbiamo visto dei capitoli precedenti come in maniera naturale "leghiamo" il
concetto di macchina al concetto di funzione, così come in questo 
capitolo abbiamo legato il concetto di programma a quello di funzione.
Cioè è naturale pensare il calcolo del valore di una qualsiasi funzione $f(a_1, a_2, ... , a_n)$ 
per mezzo di un programma $P$ con configurazione iniziale $a_1, a_2, ... , a_n$.

In generale, la funzione associata ad un programma è parziale, anziché totale, in quanto tra le caratteristiche dei nostri programmi, rientra la
posibilità che, per certi input non venga prodotto alcun output.

Si consideri ad esempio il seguente programma:
	
\begin{mylisting}
$I_1$: if $x_i = x_i $ then goto $I_1$ else goto stop
\end{mylisting}	

che non termina mai. La condizione è sempre vera è ciò causa l'esecuzione
continua della stessa istruzione. La funzione calcolata
da questo programma la chiameremo funzione sempre indefinita.

Diamo adesso una definizione di funzione calcolabile con un programma.

\begin{programmi}
Una funzione $f$ si dice \emph{calcolabile da un programma} se,
$\exists$ un programma $P$ tale che per ogni $(a_1,...,a_n) \in \mathbb{N}^k$ 
$P(a_1,...,a_n)$ termina sse $(a_1,...,a_n) \in dom(f) $ e  $x_0=a$ se 
$f(a_1,...,a_n) = a$
\end{programmi}

\subsection{Equivalenza tra programmi e macchine a registri}

E' facile verificare che con una macchina a registri si possono
eseguire le stesse operazioni viste all'inizio della sezione (e
viceversa), e in effetti vale il seguente teorema.
\begin{progabaco}
Ogni funzione \'e computabile con un programma se e solo se \'e
computabile con una macchina a registri.
\end{progabaco}
\textsc{Dimostrazione} $(\Rightarrow)$ Vediamo come le macchine a
registri sono in grado di calcolare le quattro istruzioni che
costituiscono un programma:\\

\begin{enumerate}
\item 
 $x_j \leftarrow 0$ \hspace{10mm} \xymatrix{ \roundentry{j-}
  \ar@(ul,ul)[] \ar@(ul,dl)[] \ar[r] & stop }
 \vspace{10mm}

\item $x_j \leftarrow x_i $ \hspace{10mm} \xymatrix{ \roundentry{j-}  
  \ar@(ul,ul)[] \ar@(ul,dl)[] \ar[r] & \roundentry{i-} \ar[d]  \ar[r] &  
  \roundentry{42-} \ar@(dr,d)[d] \ar[r] & stop \\ & \roundentry{j+} \ar[d] & \roundentry{i+} \ar[u]
  \\ & \roundentry{42+} \ar@(dl,dl)[uu]}
\vspace{10mm}

\item $x_j \leftarrow x_j+1$ \hspace{10mm} \xymatrix{ \roundentry{j+}
  \ar@(ul,ul)[] \ar[d] \\ ...  }
\vspace{10mm}

\item $\textrm{if }x_i \neq0$ then goto $\alpha$ else goto
  $\beta$ \hspace{10mm} \xymatrix{ \roundentry{i-} \ar@(ul,ul)[]
  \ar[d] \ar[r] & \beta\\ \roundentry{i+} \ar[d] \\ \alpha }

\end{enumerate}
	
$(\Leftarrow)$ Si tratta di simulare con (1)$\rightarrow$(4) le
operazioni di una macchina a registri:
\begin{enumerate}
\item esegue l'incremento di uno, la prima operazione elementare di
  una macchina a registri;\\ \\
\hspace{10mm} \xymatrix{\roundentry{j+} \ar@(ul,ul)[] \ar[d] \\ ...}
\hspace{10mm} $x_j \leftarrow x_j+1$

\item la seconda operazione elementare sulle macchine a registri si
  ottiene usando (4) e ponendo in $\alpha$ un'istruzione che sottragga
  1 al registro se non \'e vuoto, altrimenti il programma esegue
  l'istruzione $else$ $\beta$, che corrisponde all'operazione di
  procedere a destra.\\ \\ \xymatrix{ \roundentry{i-} \ar@(ul,ul)[]
    \ar[d] \ar[r] & \beta\\ \alpha }
 \hspace{10mm} $\textrm{if }x_i \neq 0$ then goto $\alpha$ else goto
 $\beta$ \\ con $\alpha = x \stackrel{\centerdot}{-} 1$ e $\beta=$
 'procedi a destra'.
\end{enumerate}

\hspace{\stretch{1}} $\Box$\\
\begin{esempio}[Differenza]
\ Come esempio riportiamo il calcolo della funzione differenza
definita come segue:\\
\begin{center}
$n \stackrel{\centerdot}{-} m= \left\{ \begin{array}{ll} n-m &
    \textrm{se } m \leq n\\ 0 & \textrm{altrimenti}
\end{array} \right.$
\end{center} 

\begin{figure}[h]
\hspace{0cm} \xymatrix{ \roundentry{1-} \ar[r] \ar@(dr,d)[d] & stop
  \\ \roundentry{2-} \ar@(ul,ul)[] \ar[u] \ar[r] & \roundentry{1-}
  \ar[d] \ar[r] & stop \\ & \roundentry{42+} \ar@(d,dl)[u] }
\caption{Macchina a registri che calcola la differenza definita sui
  naturali $n \stackrel{\centerdot}{-} m$}
\end{figure}
	
\ \\ Tale funzione viene calcolata dalla macchina a registri in Figura
2 e dal programma o che abbiamo esaminato nell'Esempio 1.1.
\end{esempio}
	
	
Il teorema appena enunciato stabilisce la completa equivalenza tra i
concetti di calcolabilità con registri e calcolabilità con
programmi. Si noti come attraverso queste equivalenze si giunga ad un
livello di astrazione sempre maggiore, e come proprio questa
progressione ci fornisca strumenti via via pi\'u versatili attraverso
i passaggi per la dimostrazione dei teoremi di incompletezza.
				

\section{Funzioni primitive ricorsive}
Vogliamo definire ora una classe che comprenda tutte e sole le
funzioni calcolabili. Seguiremo a tale scopo Post, il quale si serve
di una definizione ricorsiva. Una definizione \`e detta
\emph{ricorsiva} quando ci\`o che \`e da definirsi viene definito
facendo ricorso a istanze pi\`u elementari dello stesso problema.
Tale metodo consiste nel:
\begin{itemize}
 \item fissare un insieme di funzioni iniziali immediatamente
   calcolabili quale base della procedura di definizione
 \item indicare alcune regole per derivare altre funzioni ricorsive da
   quelle date in partenza (regole che ovviamente preservino la
   calcolabilità delle funzioni derivate)
 \item escludere dalla classe delle funzioni ricorsive quelle funzioni
   che non siano le iniziali o da queste derivabili.
\end{itemize}
Diamo dunque la seguente definizione:

\begin{programmi}
Si dice \emph{funzione primitiva ricorsiva} un elemento della
classe definita induttivamente a partire dalle seguenti funzioni:
\begin{itemize}
\item [a.] la funzione nulla $z$ tale che $z(x)=0$;
\item [b.] la funzione proiezione $p^n_i$ tale che $p_i^n(x_1,\ldots
  ,x_n)=x_i$ $\forall i\leq n$
\item [c.]la funzione successore $s$ tale che $s(x) = x+1$;
\end{itemize}
\end{programmi}
	
Le regole per produrre nuove funzioni sono:
\begin{enumerate}
\item le operazioni di \emph{\emph{composizione}}, che date le
  funzioni primitive ricorsive $f: \mathbb{N}^k \rightarrow
  \mathbb{N}$ e $g_i : \mathbb{N}^n \rightarrow \mathbb{N}$ per $i= 1,
  ... , k$ permette di ottenere una funzione $h : \mathbb{N}^n
  \rightarrow \mathbb{N}$ per cui:\\ $
  h(x_1,.....,x_n)=f(g1(x_1,.....,x_n),....., g_k (x_1,.....,x_n))$
  \\ anch'essa primitiva ricorsiva.
\item lo schema di \emph{\emph{ricorsione primitiva}}, che date
  le funzioni primitive ricorsive $f: \mathbb{N}^k \rightarrow
  \mathbb{N}$ e $g : \mathbb{N}^{k+2} \rightarrow \mathbb{N}$, allora
  \`e primitiva ricorsiva anche la funzione $h : \mathbb{N}^{k+1}
  \rightarrow \mathbb{N}$ che soddisfi il sistema di equazioni:
\begin{center}
$\left\{ \begin{array}{ll} h(x_1,.....,x_n,0)= f(x_1,.....,x_n)
    \\ h(x_1,.....,x_n,s(y))=g(x_1,.....,x_n, y, h(x_1,.....,x_n,y))
				\end{array} \right.$
\end{center}
\end{enumerate}
\vspace{5mm}%
	
La classe delle funzioni primitive ricorsive \`e dunque \emph{chiusa
  rispetto alle operazioni di composizione e ricorsione
  primitiva}. Ora vedremo che questa classe di funzioni, che abbiamo
appena definito in termini matematici, \`e calcolabile dai programmi.


\begin{thm}
\emph{Ogni funzione primitiva ricorsiva è eseguibile da un programma
(e quindi calcolabile).}\end{thm}
\begin{proof}
Dobbiamo fare un programma per le tre funzioni di base, per la
composizione generalizzata e la ricorsione. Come sappiamo l' output è
per convenzione in $x_{0}$. Assumiamo anche in questa dimostrazione che 
se un programma P utilizza una variabile per la prima volta prima di usarla
azzeri il suo contenuto. Questa assunzione deve essere fatta perché
quando parleremo di ricorsione e composizione delle funzioni 
i programmi verranno eseguiti uno dopo l'altro e tra le loro possibilità rientra quella 
di utilizzare le stesse variabili usate dai programmi precedenti. Ciò assicura
che azzerando la variabile prima dell'uso non si trovino valori `sporchi`.

\begin{itemize}
\item[a.] Facciamo un programma per la funzione zero:
\begin{mylisting}
$I_1$: $x_{0}\leftarrow0$
\end{mylisting}

\item[b.] Il programma per il successore:\begin{mylisting}$I_1$:
  $x_{1}\leftarrow x_{1}+1$ \\$I_2:x_{0}\leftarrow x_{1} $\end{mylisting}

\item[c.] La proiezione:\begin{mylisting}$I_1$: $x_{0}\longleftarrow
  x_{i}$\end{mylisting}
\end{itemize}

\begin{enumerate}
\item La composizione: siano $F, G_1, G_2, \dots, G_k$ programmi che calcolano
  le funzioni $f,g_{1},...g_{k}$ rispettivamente. Dobbiamo far vedere che
  riusciamo ad esibire un programma $H$ che calcola la funzione $h$.
  L'output di ogni
  funzione viene salvato in $x_{0}$; quindi, poich\'e vogliamo salvare
  tutti i risultati intermedi del calcolo di $g_{1},\dots,g_{k}$
  dobbiamo copiare di volta in volta $x_0$ in una variabile che siamo
  sicuri che non sia usata da uno dei programmi. Inoltre dobbiamo
  riservarci dello spazio per le variabili di input poich\'e qualche
  programma potrebbe modificarle durante la sua esecuzione.  Per
  sapere dove possiamo salvare tutti questi valori dobbiamo sapere
  quali variabili vengono utilizzate dai programmi.
  Sia quindi
  $$\rho(Q)=max\left\{ i|x_{i}\leftarrow...\in Q\right\}$$
  il massimo indice di una variabile assegnata in $Q$ e
  $$\varepsilon=max\left\{
  \rho(G_{1}),....,\rho(G_{k}),\rho(F),k,n\right\}$$
  il massimo indice utilizzato nei programmi $F, G_1, \dots, G_k$.

Costruiamo il programma $H$ che calcola la composizione come segue:
salviamo l'input, che si trova in $x_{1},\dots,x_{n}$ in
$x_{\varepsilon+1},\dots,x_{\varepsilon+n}$, eseguiamo il programma
$G_1$ e copiamo l'output in $x_{\varepsilon+n+1}$:

\begin{mylisting}
$x_{\varepsilon+1}\leftarrow x_{1}$\\
$\vdots$\\
$x_{\varepsilon+n}\leftarrow x_{n}$\\
$[G_{1}]$\\
$x_{\varepsilon+n+1}\leftarrow x_{0}$
\end{mylisting}

Copiamo l'input:
\begin{mylisting}
$x_{1}\leftarrow x_{\varepsilon+1}$\\
$\vdots$\\
$x_{n}\leftarrow x_{\varepsilon+n}$
\end{mylisting}

Eseguiamo $G_{2}$ e copiamo l'output in $x_{\varepsilon+n+2}$;
ripetiamo questa operazione per tutti gli altri programmi fino ad
applicare il programma $G_{k}$ e copiare il suo output in
$x_{\varepsilon+n+k}$.

A questo punto tutti gli input per F si trovano in
$x_{\varepsilon+n+1},\dots,x_{\varepsilon+n+k}$.  Copiamo questi
valori in $x_{1},\dots,x_{k}$ e applichiamo il programma F:

\begin{mylisting}
$x_{1}\leftarrow x_{\varepsilon+n+1}$\\
$\vdots$\\
$x_{k}\leftarrow x_{\varepsilon+n+k}$\\
$[F]$
\end{mylisting}

Dopo questa sequenza di operazioni il risultato della composizione si
trova in $x_{0}$.

\item La ricorsione: sia $F$ un programma che calcola $f$ e $G$ un
  programma per $g$. Per realizzare un programma $H$ che calcola $h$
  dobbiamo innanzitutto salvare le variabili di input e i risultati
  parziali di $f$ e $g$ in variabili non utilizzate dai due programmi
  $F$ e $G$.

  Sia $\rho(Q)=max\left\{ i|x_{i}\leftarrow\dots \in Q\right\}$ il
  massimo indice utilizzato nel programma Q; definiamo
  $\varepsilon=max\left\{ \rho(F),\rho(G),k+2\right\}$ il massimo
  indice utilizzato per entrambi i programmi.

Allora, prima copiamo l'input e mettiamo 0 in $x_{\varepsilon+k+2}$
per cominziare la ricorsione (per calcolare $h(x,0)$). Adesso possiamo
applicare il programma $F$.

Poi dobbiamo guardare se l'$y$ che ci è stato dato all'inizio che copiamo
in $x_{\varepsilon+k+3}$ è uguale
a zero; se è cosi, il programma deve terminare. Se non è cosi, il
programma deve calcolare $h(x,s(y))$ (la prima volta $s(y)$ sarà
1). Il programma G utilizza come input $x_{1},\dots,x_{k}$ per
$x_{1},\dots,x_{k}$, $x_{k+1}$ per $y$ e $x_{k+2}$ per il risultato
parziale $h(x,y)$. Quindi dobbiamo mettere l'input iniziale in
$x_{1},\dots,x_{k}$, il risultato al passo precedente che si trova in
$x_{\varepsilon+k+2}$ dobbiamo copiarlo in $x_{k+1}$ e, dopo aver
eseguito $G$, dobbiamo copiare il suo output da $x_{0}$ in $x_{k+2}$.
Infine calcoliamo il successore di $y$ e controlliamo se equivale
all'$y$ iniziale: in caso affermativo dobbiamo terminare, altrimenti
torniamo ad eseguire il ciclo. Allora, il programa sarà:

\begin{mylisting}
$x_{\varepsilon+1}\leftarrow x_{1}$\\
$\vdots$\\
$x_{\varepsilon+k}\leftarrow x_{k}$\\
$x_{\varepsilon+k+3}\leftarrow x_{k+1}$ \\
$[F]$\\
$x_{\varepsilon+k+2}\leftarrow x_{0}$\\
$x_{\varepsilon+k+1}\leftarrow 0$\\
$I_t:\; $if$\; x_{\varepsilon+k+3}=x_{\varepsilon+k+1}\; $then goto$\;stop$ else goto $I_{t+1}$\\
$I_{t+1}:x_{1}\leftarrow x_{\varepsilon+1}$\\
$\vdots$\\
$x_{k}\leftarrow x_{\varepsilon+k}$\\
$x_{k+1}\leftarrow x_{\varepsilon+k+1}$\\
$x_{k+2} \leftarrow x_{\varepsilon+k+2}$\\
$[G]$\\
$x_{\varepsilon+k+2} \leftarrow x_0$\\
$x_{\varepsilon+k+1}\leftarrow x_{\varepsilon+k+1}+1$\\
$if \; x_0=x_0\; $ then goto $I_t$ else ...
\end{mylisting}
\end{enumerate}
\end{proof}

% Controllare
\subsection{Esempi di funzioni primitive ricorsive}

\begin{esempio}[somma]
$h:\mathbb{N}^2 \to \mathbb{N}$, $h(x,y)=x+y$.
Possiamo usare lo schema semplificato
$\left\{ \begin{array}{ll}
	x+0=x\\
	x+s(y)=s(x+y)
\end{array}\right.$.\\
Nel nostro schema di ricorsione la stessa funzione si ottiene con $h$
tale:\newline
$$\begin{array}{ll}
	h(x,0)=f(x)=p_1^1(x)=x\\
	h(x,s(y))=g(x,y,h(x,y))=s(p_3^3(x,y,h(x,y)))=s(h(x,y)).
\end{array}$$\newline
\end{esempio}
%
\begin{esempio}[prodotto]

 $h:\mathbb{N}^2 \to \mathbb{N}$, $h(x,y)=x \cdot y$ che possiamo scrivere
come
$\left\{ \begin{array}{ll}
	x \cdot 0=0\\
	x \cdot s(y)= x \cdot y + x
\end{array}\right.$ e quindi: \newline
$$\begin{array}{ll}
	h(x,0)=f(x)=z(x)\\
	h(x,s(y))=g(x,y,h(x,y))=p_3^1(x,y,h(x,y))+p_3^3(x,y,h(x,y)).
\end{array}$$\newline
\end{esempio}
%
\begin{esempio}[fattoriale] Per adattare la definizione della funzione
fattoriale
$h:\mathbb{N} \to \mathbb{N}$, $h(x)=x!$ allo schema generale si considera
$\left\{ \begin{array}{ll}
	0!=1\\
	s(y)! = y! \cdot s(y)
\end{array}\right.$ e quindi si pu\`o prendere $h$ come segue: \newline
$$\begin{array}{ll}
	h(x,0)=f(x)=s(z(x))=1\\
	h(x,s(y))=g(x,y,h(x,y))= s(p_3^2(x,y,h(x,y)))) \cdot p_3^3(x,y,h(x,y))=
\\
	\hspace{1.8cm} = s(y)\cdot h(x,y).
	\end{array}$$\newline
\end{esempio}
%
%
\begin{esempio}[elevamento a potenza] $h:\mathbb{N}^2 \to \mathbb{N}$,
$h(x,y)=x^y$:
$\left\{ \begin{array}{ll}
	x^0=1\\
	x^{s(y)} = x^y \cdot x
\end{array}\right.$ e quindi: \newline
$$\begin{array}{ll}
	h(x,0)=f(x)=s(z(x))=1\\
	h(x,s(y))=g(x,y,h(x,y))= p_3^1(x,y,h(x,y)) \cdot p_3^3(x,y,h(x,y)) =
x\cdot h(x,y).
	\end{array}$$\newline
\end{esempio}
%
%
\begin{esempio}[predecessore] $p:\mathbb{N} \to \mathbb{N}$ tale che
$\left\{ \begin{array}{ll}
	p(0)=0\\
	p(s(y)) = y
\end{array}\right.$ e quindi: \newline
$$\begin{array}{ll}
	p(0)=f=0\\
	p(s(y))=g(y,p(y))= p_2^1(y,p(y)).
	\end{array}$$\newline
\end{esempio}
%
%
\begin{esempio}[sottrazione] $h:\mathbb{N}^2 \to \mathbb{N}$,
$ h(x,y) = \left\{ \begin{array}{ll}
	x \stackrel{\centerdot}{-} y \ se \ y \leq x\\
	0 \ altrimenti
\end{array}\right.$ dunque: \newline
$$\begin{array}{ll}
	h(x,0)=f(x)=x\\
	h(x,s(y))=p(h(x,y)).
	\end{array}$$\newline
\end{esempio}
%
%
\begin{esempio}[segno] $sgn:\mathbb{N} \to \left\{0,1\right\}$,
$ sgn(y) = \left\{ \begin{array}{ll}
	0 \ se \ y = 0\\
	1 \ se \ y > 0
\end{array}\right.$ che si pu\`o scrivere come:
 \[ sgn(y) = y \stackrel{\centerdot}{-} p(y) \]
che \`e primitiva ricorsiva per quanto visto negli esempi precedenti.
\end{esempio}
%
%
\begin{esempio}[controsegno] $\overline{sgn}:\mathbb{N} \to
\left\{0,1\right\}$,
$ \overline{sgn}(y) = \left\{ \begin{array}{ll}
	1 \ se \ y = 0\\
	0 \ se \ y > 0
\end{array}\right.$ che si pu\`o pensare come:
 \[ \overline{sgn}(y) = 1 \stackrel{\centerdot}{-} sgn(y) \]
dunque \`e primitiva ricorsiva.
\end{esempio}
%
%
\begin{esempio}[] $f(\overrightarrow{x},y)= \sum_{i=0}^{y} g(\overrightarrow{x},
i)$ con $g$ primitiva ricorsiva, allora:\\
 $ f(\overrightarrow{x}, 0) = g(\overrightarrow{x}, 0)  $ \\
 $ f(\overrightarrow{x}, s(y)) = f(\overrightarrow{x}, y) +
g(\overrightarrow{x}, s(y)). $
\end{esempio}
%
%
\begin{esempio}[] $f(\overrightarrow{x},y)= \prod_{i=0}^{y}
g(\overrightarrow{x}, i)$ con $g$ primitiva ricorsiva, allora:\\
$	f(\overrightarrow{x}, 0) = g(\overrightarrow{x}, 0) $ \\
$ 	f(\overrightarrow{x}, s(y)) = f(\overrightarrow{x}, y) \cdot
g(\overrightarrow{x}, s(y)). $
\end{esempio}
%
%
\begin{esempio}[] $ \chi_{\geq}(x,y) = \left\{ \begin{array}{ll}
	1 \ se \ x \geq y\\
	0 \ altrimenti
\end{array}\right.$
\[ \chi_{\geq}(x,y) = sgn (s(x) \stackrel{\centerdot}{-} y) . \]
\end{esempio}
%
%
\begin{esempio}[valore assoluto] Basta scriverlo come
\[ \left|x - y \right| = (x \stackrel{\centerdot}{-} y) + ( y
\stackrel{\centerdot}{-} x) \]
oppure
\[ \left|x - y \right| = (x \stackrel{\centerdot}{-} y) \chi_{\geq}(x,y)  +
( y \stackrel{\centerdot}{-} x) (1 - \chi_{\geq}(x,y)) .\]
\end{esempio}
%
%
\begin{esempio}[] $f(\overrightarrow{x}) = \left\{ \begin{array}{ll}
	g_1(\overrightarrow{x})\ se\ vale\  R(\overrightarrow{x})\\
	g_2(\overrightarrow{x})\ altrimenti
\end{array}\right.$ con \\$\chi_R(\overrightarrow{x}) = \left\{
\begin{array}{ll}
	1\ se\ R(\overrightarrow{x})\ vale\\
	0\ altrimenti
\end{array}\right.$ con $g_1(\overrightarrow{x})$, $g_2(\overrightarrow{x})$
e $\chi_R(\overrightarrow{x})$ primitive ricorsive.\\
Allora
\[ f(\overrightarrow{x}) = g_1(\overrightarrow{x}) \chi_R(\overrightarrow{x}) +
 g_2(\overrightarrow{x}) (1 - \chi_R(\overrightarrow{x})). \]
\end{esempio}
%

% Alessandro Onnivello
\section{Dalle funzioni primitive ricorsive alle funzioni ricorsive}

Nel paragrafo precedente abbiamo definito l'insieme delle funzioni
primitive ricorsive; possiamo dire che queste siano sufficienti per
rappresentare tutte le funzioni calcolabili da una macchina di Turing?
La risposta è negativa, come andremo ora a dimostrare.

\begin{prop}\label{PRsonoTotali}
Le funzioni primitive ricorsive sono totali.
\end{prop}
\begin{proof}
Si dimostra per induzione sulla struttura delle funzioni primitive
ricorsive.  Poichè le funzioni di base sono funzioni totali dobbiamo
verificare che la composizione e la ricorsione primitiva preservano la
totalità delle funzioni composte.

Per la composizione dobbiamo verificare che la funzione composta
$f(g_1, \dots, g_k)$ è totale. Per ipotesi induttiva
(strutturale) $f,g_1,\dots,g_k$ sono totali. Per la totalità di
$g_1,\dots,g_k$ tutti gli argomenti di $f$ sono definiti e, poichè $f$
è totale, anche la composizione è definita.

Per la ricorsione primitiva dobbiamo dimostrare che
$h:\mathbb{N}^{k+1} \to \mathbb{N}$ è definita $\forall y \in
\mathbb{N}$ e $\forall \vec{x} \in \mathbb{N}^k$.

$$h(\vec{x},y)=\left\{
\begin{array}{ll} h(\vec{x}, 0) = f(\vec{x})\\
                  h(\vec{x}, s(y)) = g(\vec{x},y,h(\vec{x},y))
\end{array} \right.$$

Si dimostra per induzione su $y$.

$y = 0$: $h(\vec{x}, 0) = f(\vec{x})$\\ $f$ è totale per ipotesi
induttiva quindi $\forall \vec{x} \in \mathbb{N}^k (\vec{x}, 0) \in
dom(h)$ come atteso

$y + 1$: $h(\vec{x}, y + 1) = g(\vec{x}, y, h(\vec{x}, y))$\\
$h(\vec{x}, y)$ è totale per ipotesi induttiva su $y$ quindi
$h(\vec{x}, y)$ è definita. Poichè $g$ totale per ipotesi induttiva e
$h(\vec{x}, y)$ è definita allora $\forall \vec{x} \in \mathbb{N}^k \;
(\vec{x}, y + 1) \in dom(h)$.
\end{proof}

Ora, per quanto detto nella precedente proposizione, l'insieme delle funzioni
primitive ricorsive permette di rappresentare solo funzioni calcolabili totali
ma, come abbiamo visto nel primo capitolo, le macchine di Turing possono
calcolare anche funzioni parziali. Quindi possiamo affermare che le
funzioni primitive ricorsive non rappresentano tutte le funzioni Turing
calcolabili.

Ma limitandoci al caso delle funzioni totali, possiamo dire che queste
siano tutte primitive ricorsive? Anche in questo caso la risposta è
negativa come vedremo nel prossimo paragrafo.

\section{Metodo diagonale di Cantor}
Per poter dimostrare che l'insieme delle funzioni primitive ricorsive
non contiene tutte le funzioni totali calcolabili andremo ora a
introdurre il metodo diagonale (o diagonalizzazione) di Cantor.  Il
metodo di diagonalizzazione è una tecnica dimostrativa, ideata da
Georg Cantor per provare la non numerabilità dei numeri reali, che
risulta molto utile anche nell'ambito della logica matematica e della
teoria della computabilità come vedremo in seguito.

Il metodo diagonale di Cantor consiste nel costruire una funzione $\chi$
che differisce da un'insieme infinito di funzioni $\chi_0,\chi_1,\dots$
effettivamente enumerabile. Si costruisce quindi la seguente tabella
dove ogni colonna contiene una funzione unaria e le righe contengono
la sequenza dei naturali, ovvero tutti i possibili argomenti.
\begin{table}[!h]
\begin{center}
\begin{tabular}{|c|c|c|c|c}

\hline
 & $\chi_0$ & $\chi_1$ & $\chi_2$ & $\ldots$\\
\hline
0 & $\chi_0(0)$ & $\chi_1(0)$  & $\chi_2(0)$  & $\ldots$\\
\hline
1 & $\chi_0(1)$ & $\chi_1(1)$  & $\chi_2(1)$  & $\ldots$\\
\hline
2 &  $\chi_0(2)$ & $\chi_1(2)$  & $\chi_2(2)$  & $\ldots$\\
\hline
$\vdots$ & $\vdots$ &       $\vdots$  & $\vdots$  & $\ddots$\\

\end{tabular}
\end{center}
\caption{Metodo diagonale di Cantor}
\end{table}

A questo punto possiamo costruire una funzione $\chi$ in modo tale che
differisca da ogni funzione enumerata nella tabella; vogliamo quindi
costruire una funzione tale che $\forall i \; \chi(i) \neq
\chi_{i}(i)$, ovvero che differisce da ogni altra funzione elencata
nella tabella almeno sulla diagonale. Questa nuova funzione, creata
tramite una procedura effettiva, non può appartenere all'insieme di
partenza, poichè altrimenti differirebbe da sè stessa sulla diagonale.

\begin{prop}\label{TotCalcNonNum}
Data una qualunque lista \emph{effettiva} di tutte le funzioni totali
calcolabili, unarie per semplicit\`a, $f_{0},\; f_{1},\; f_{2},
\cdots$ esiste sempre una funzione $g$ totale calcolabile che non
compare nella lista.
\end{prop}
\begin{proof}
Sia $f_{0},\; f_{1},\; f_{2}, \cdots$ una lista effettiva di funzioni
totali calcolabili, allora costruiamo la funzione $g$ nel seguente
modo:

\begin{center}
$g: \mathbb{N} \to \mathbb{N}$\\
$g(x) = f_{x}(x) + 1$
\end{center}
\begin{table}[!h]
\begin{center}
\begin{tabular}{|c|c|c|c|c}

\hline
 & $f_0$ & $f_1$ & $f_2$ & $\ldots$\\
\hline
0 & $f_0(0) \mathbf{+1}$ & $f_1(0)$  & $f_2(0)$  & $\ldots$\\
\hline
1 & $f_0(1)$ & $f_1(1) \mathbf{+1}$  & $f_2(1)$  & $\ldots$\\
\hline
2 &  $f_0(2)$ & $f_1(2)$  & $f_2(2) \mathbf{+1}$  & $\ldots$\\
\hline
$\vdots$ & $\vdots$ &       $\vdots$  & $\vdots$  & $\ddots$\\
\end{tabular}
\end{center}
\caption{Funzione di diagonalizzazione $\mathbf{g}$}
\label{diagG}
\end{table}

$g$ è sicuramente totale calcolabile poichè, essendo la lista di
funzioni effettiva è quindi possibile, dato un qualsiasi $n \in
\mathbb{N}$, calcolare l'ennesima funzione, valutarla in $n$ ed
aggiungerci 1. È chiaro che $g$ non può appartenere alla lista, poichè
presa una qualsiasi colonna $n$ della tabella~\ref{diagG} che
rappresenta l'immagine dell'ennesima funzione, alla riga ennesima
$g(n) \neq f_{n}(n)$ poichè per definizione $g(n)=f_{n}(n)+1$. Perciò
non esiste alcun $n$ tale che $f_{n}(x)=g(x)$ per ogni $x \in
\mathbb{N}$ e quindi $g$ non è contenuta nella nostra lista.
\end{proof}

Fatto questo ragionamento, ne consegue che non è possibile enumerare
in modo effettivo tutte le funzioni totali calcolabili.


La proposizione appena dimostrata potrebbe indurre a pensare che le funzioni totali calcolabili non siano numerabili, del resto è lo stesso metodo che usa Cantor per dimostrare la non numerabilità di $\mathbb{R}$. Ma non  così, la differenza è nella parola "effettiva" presente nella proposizione. Quello su cui lavoriamo è una lista effettiva di tutte le funzioni calcolabili, una lista costruibile ossia generabile da un algoritmo e quindi ne possiamo conoscere ogni elemento ($f_x$). 
Cantor invece non richiedeva che la sua lista di numeri reali fosse effettiva, probabilmente all' epoca non esisteva ancora questo concetto, ma soprattutto sarebbe impossibile averla e memorizzarla. Infatti un qualunque numero reale è composto da infinite cifre decimali e non basterebbero nemmeno tutti gli atomi dell'universo per conterlo, figuriamoci la memoria di un computer (infatti usa un'approssimazione finita)!\\
La cosa che può sembrare paradossale è che estendendendo l'insieme delle funzioni totali ricorsive a quelle parziali ricorsive si riesce a ottenere tale lista effettiva.



Possiamo usare ora la proposizone~\ref{TotCalcNonNum} per dimostrare il seguente

\begin{thm}\label{diagRic} Esiste una funzione totale calcolabile che non
è primitiva ricorsiva.
\end{thm}

\begin{proof}
Dobbiamo innanzitutto costruire una lista \emph{effettiva} di tutte le
funzioni primitive ricorsive (per semplicit\`a, supponiamole ad un
argomento).

Cerchiamo di scrivere informalmente un algoritmo che ci permetta di
listare tutte queste funzioni (in seguito vedremo come si pu\`o fare
in modo rigoroso).  Una tale enumerazione si pu\`o ottenere nello
stesso modo in cui si ottiene una lista di teoremi a partire dagli
assiomi di una teoria. Si prende prima una funzione base, poi si
``contano'' tutte le funzioni ottenute da questa mediante una sola
applicazione di composizione e/o ricorsione. Poi si prende la seconda
funzione e tutte le funzioni derivate mediante una applicazione delle
due regole da quest'ultima e/o dalle funzioni gi\`a ottenute al passo
precedente, e cos\`i via. In questa maniera non si tralascia alcuna
funzione e ad ogni passo la lista \`e finita. Abbiamo cos\`i ottenuto
una enumerazione effettiva per tutte le funzioni dell'insieme
$\mathcal{PR}$.\\ Ma per quanto detto nella
proposizone~\ref{TotCalcNonNum} sappiamo che le funzioni totali non
sono enumerabili effettivamente quindi questo implica che esiste una
funzione totale calcolabile che non appartiene l'enumerazione che
abbiamo appena definito e che quindi non appartiene all'insieme
$\mathcal{PR}$.
\end{proof}

A questo punto ci chiediamo: cosa ci manca per avere tutte le funzioni
calcolabili?
Ripensando a quanto visto fino ad ora, il problema pu\`o stare o nel pretendere
di avere una lista effettiva delle funzioni calcolabili (cosa che per\`o \`e
ragionevole a farsi: lo abbiamo visto poco fa), o nell'imporre la propriet\`a di
totalit\`a alle funzioni. Dunque vediamo se lasciando cadere questa assunzione
riusciamo a classificare tutte le funzioni calcolabili.


\section{Le funzioni ricorsive}
Avendo osservato che le funzioni primitive ricorsive non
esauriscono tutte le funzioni calcolabili, il passo successivo \`e  quello di
estenderle ulteriormente. In particolare ci sar\`a bisogno di trovare
anche funzioni parziali nella nostra definizione di ``ricorsivit\`a''. Perci\'o
introduciamo
la seguente funzione: \\

f. \textbf{minimizzazione}: sia $f: \mathbb{N}^{d+1} \to \mathbb{N}$ (anche
parziale)\\
definiamo la funzione di minimizzazione $h: \mathbb{N}^{d} \to \mathbb{N}$ come
$$ h(\vec{x}):= \mu_{y}(f(\vec{x},y)) $$
dove

\[\mu_{y}(f(\vec{x},y)) = 
\begin{cases}
il \; minimo \; y \; tale \; che: \\
\qquad (i) \; f(\vec{x},z)\downarrow \; \forall z \leq y \; e \\
\qquad (ii) \; f(\vec{x},y)=0 & \text{se y esiste,} \\
non \; definita \; & \text{se $\exists z < y \; f(\vec{x},z) \uparrow$}\\
 & \text{o se $f(\vec{x},y)\neq 0 \; \forall y \in \mathbb{N} $}
\end{cases} \]

con $\mu$ che prende il nome di operatore di minimizzazione.
\begin{thm} Le funzioni ricorsive sono b-programma-calcolabili.
\end{thm}
\begin{proof} Poich\`e abbiamo gi\`a dimostrato che le tre funzioni di base
(funzione zero, successore e proiezioni), la composizione generalizzata e la
ricorsione primitiva sono calcolabili da un b-programma ci resta solo da far
vedere che questo vale anche per la minimizzazione. Infatti se
$f:\mathbb{N}^{d+1} \to \mathbb{N}$ \`e computabile con un b-programma $\alpha$
allora possiamo eseguire la seguente assegnazione:
$$x_i \leftarrow f(\vec{x_{j}}, x_k)$$
(dove con $\vec{x_{j}}$ vogliamo rappresentare in modo compatto una lista di d
variabili di input)che significa "`dai come input $\vec{x_{j}}$ e $x_k$ al
b-programma $\alpha$ che calcola $f(\vec{x_{j}}, x_k)$ e dai output risultante
come valore a $x_i$"'. Dunque il b-programma che calcola la funzione di
minimizzazione su $\vec{x}$ \`e il seguente:\\
   \begin{mylisting}
       $x_0 \leftarrow 0$ \\
       $loop$: $\vec{x_{1}} \leftarrow \vec{x}$\\
       $x_{d+2} \leftarrow f(\vec{x_{1}}, x_0)$\\
       if $x_{d+2} = 0$ then goto $stop$\\
       $x_0 \leftarrow x_0 + 1$\\
       goto $loop$
  \end{mylisting}
\end{proof}

Una volta dimostrato che la minimizzazione è calcolabile dai
b-programmi possiamo dare la seguente
\begin{defi} Una funzione $f:\mathbb{N}^{d} \to \mathbb{N}$ (totale o
parziale) si dice \emph{ricorsiva} se si ottiene dalle funzioni
iniziali (a.--c.) applicando la composizione, la ricorsione e la
minimizzazione (d.--f.).
\end{defi}

Una cosa importante da notare \`e che l'utilizzo dell'operatore $\mu$ consente
di ottenere funzioni parziali anche a partire funzioni totali, come vedremo nei
seguenti esempi.
\begin{esempio} data $f:\mathbb{N} \to \mathbb{N}$ definita nel seguente modo:
$$f(x)= \left\{ \begin{array}{ll}
il \; min \; y \; t. \, c. \; x+y=0 & se \; \exists \; y\\
non \; definita & se \; x+y \neq 0 \; \forall y \in \mathbb{N}\\
\end{array} \right.$$
utilizzando la minimizzazione pu\`o essere scritta nel seguente modo:
$$f(x)= \mu_{y}(x+y)$$

Si noti che, essendo $x \in \mathbb{N}$, se $x \neq 0 \; f$ non \`e definita
anche se la funzione somma su cui viene applicato l'operatore di minimizzazione
\`e totale.
\end{esempio}

\begin{esempio} data $f:\mathbb{N}^{2} \to \mathbb{N}$ definita nel seguente
modo:
$$f(x,y)= \left\{ \begin{array}{ll}
x/y & se \; y \vert x\\
non \; definita & se \; y \nmid x\\
\end{array} \right.$$
pu\`o essere scritta nel seguente modo:
$$f(x,y) = \mu_{z}(\vert (z*y) - x \vert) $$
anche in questo caso la funzione $g(x,y,z)=\vert (z*y) - x \vert$ \`e totale, ma
combinata con l'operatore di minimizzazione consente di ottenere la funzione
parziale $f$.
\end{esempio}

\section{Relazioni ricorsive}
\begin{defi} $R \subseteq \mathbb{N}^{n}$ è una \emph{relazione
ricorsiva} se esiste una funzione ricorsiva $\chi_R$ che assume solo valori 0 e
1 e che soddisfa
$$\chi_R(x_1, \ldots, x_n):= \left \{ \begin{array}{ll}
                                      1 & (x_{1}, x_{2}, \cdots, x_{n}) \in R\\
                                      0 & (x_{1}, x_{2}, \cdots, x_{n}) \not \in
 R\\
                                      \end{array} \right. $$
\end{defi}
\begin{prop} Se $R, S \subseteq \mathbb{N}^{n}$ sono relazioni
ricorsive allora anche $R\land S$, $R \vee S$, $\neg R$ sono relazioni
ricorsive. Inoltre, se $w \in \mathbb{N}$, anche $\forall y \leq
w. R(x_1, \ldots, x_{n-1}, y)$ e $\exists y \leq w. R(x_1, \ldots,
x_{n-1}, y)$ sono relazioni ricorsive.
\end{prop}
\begin{proof} Si pone
\begin{itemize}
\item $\chi_{R\land S} = \chi_R \chi_S$;
\item $\chi_{R\vee S} = \chi_R + \chi_S - \chi_R  \chi_S$;
\item $\chi_{\neg R} = 1 - \chi_R$;
\item $\chi_{\forall y
\leq w. R(x_1, \ldots, x_{n-1}, y)} = \prod_{i=0}^{w} \chi_R(x_1, \ldots,
x_{n-1},i)$;
\item $\chi_{\exists y
\leq w. R(x_1, \ldots, x_{n-1}, y)} = sgn(\sum_{i=0}^{w} \chi_R(x_1, \ldots,
x_{n-1},i))$
\end{itemize}
\end{proof}

Grazie alla definizione data sopra, ci \`e possibile giustificare la
\emph{definizione per casi} delle funzioni ricorsive.

Supponiamo di avere una funzione $f: \mathbb{N}^{n} \to n$ t. c.
$$f(\vec{x}):= \left \{ \begin{array}{ll}
                    g_1(\vec{x}) & $se $ \vec{x} \in R_{1}\\
                    \vdots \\
                   g_k(\vec{x}) & $se $ \vec{x} \in R_{k}\\
                    \end{array} \right.$$
con $g_{1}, g_{2}, \cdots, g_{k}:\mathbb{N}^{n} \to \mathbb{N}$ funzioni
ricorsive e dove $R_{1}, R_{2}, \cdots, R_{k} \subseteq \mathbb{N}^{n}$ sono
relazioni ricorsive mutualmente esclusive ed esaustive, ossia abbiamo che:
\begin{itemize}
\item $\sum_{i=1}^{k} \chi_{R_{i}}(\vec{x})=1$ e che
\item se $R_{i}(\vec{x})$ vale allora $\forall j \neq i \; \chi_{R_{i}}(\vec{x})
\cdot \chi_{R_{j}}(\vec{x})=0$.
\end{itemize} 
Possiamo quindi prendere le funzioni caratteristiche $\chi_{R_{1}},
\chi_{R_{2}}, \cdots, \chi_{R_{k}}$ e definire la funzione $f$ come:
$$f(\vec{x})= \sum_{i=1}^{k} g_i(\vec{x})\chi_{R_{i}}(\vec{x})$$ e
quindi essendo $f$ ottenuta per composizione di funzioni ricorsive \`e
anch'essa ricorsiva.  Quindi se vogliamo descrivere una funzione che
assume valori diversi in casi diversi lo possiamo fare mantenendoci
all'interno delle funzioni primitive ricorsive. Questo può permetterci
di definire funzioni come quella del prossimo esempio.

\begin{esempio} Data la seguente funzione:
\[ f: \mathbb{N}^{2} \to \mathbb{N} \]
\[ f(x,y)= \begin{cases}
x-y	& \text{se $x \geq y$}\\
x+y & \text{se $x<y$}
\end{cases}\]
la sua definizione per casi \`e la seguente:\\
$f(x,y) = \sum_{i=1}^{2} g_{i} \cdot \chi_{R_{\geq}}(x,y) = (x-y)\cdot
\chi_{R_{\geq}}(x,y) + (x+y)\cdot \neg \chi_{R_{\geq}}(x,y) $
Si lascia come esercizio al lettore il trovare $\chi_{R_{\geq}}$.
\end{esempio}

% Alessandro Bruni

\section{Equivalenze}

Per riassumere il percorso seguito finora presentiamo il seguente schema.
Ad ogni passaggio abbiamo introdotto un linguaggio pi\`u
raffinato senza per\`o aggiungere nuove funzioni:
 ogni volta abbiamo aggiunto qualcosa che fosse definibile in
termini del precedente e quindi in particolare di macchine di Turing.
\begin{align*}
{\rm Macchine\ }&{\rm di\ Turing\ }\\
\bigcup &|\\
{\rm Macchine\ }&{\rm a\ registri\ }\\
\bigcup &|\\
{\rm Progra}&{\rm mmi}\\
\bigcup &|\\
{\rm Funzioni\ }&{\rm ricorsive\ }\\
\bigcup &| \; {\rm?}\\
{\rm Macchine\ }&{\rm di\ Turing\ }\\
\end{align*}

È valida valida l'ultima inclusione? in caso affermativo tutti i
concetti di funzioni calcolabili descritti fino ad ora sono tra loro
equivalenti. Il seguente teorema dimostra che quest'inclusione è
effettivamente valida.

\begin{thm}[Le funzioni Turing-calcolabili sono ricorsive]
Ovvero, per ogni MdT possiamo costruire una funzione ricorsiva che la calcola.
\end{thm}

\begin{proof}
Dobbiamo innanzitutto trovare un modo per rappresentare il nastro (ovvero la
memoria) della MdT. Per semplicità di esposizione poniamoci nel caso $\sigma =
\{ B, 1 \}$, poichè ogni altro alfabeto finito può avere una rappresentazione
simile nei naturali, scegliendo una rappresentazione in base $|\sigma|$
dell'input.

Una MdT computa una funzione $f: \mathbb{N}^k \rightarrow \mathbb{N}$,
possibilmente parziale, quindi lo stato del nastro all'inizio della
computazione è una sequenza di caratteri che contiene le variabili in
input:
$$\dots BB\overline{x_1+1}B\overline{x_2+1}B...B\overline{x_k+1}BB\dots $$
mentre alla fine del calcolo la macchina dovrebbe lasciare nel nastro
il risultato della funzione che calcola:
$$\dots BB\overline{f(x_1,\dots,x_k)+1}BB\dots $$

Diamo ora le seguenti definizioni, per meglio comprendere la nostra codifica
del nastro in $\mathbb{N}$.

\begin{defi}[Numerale sinistro]
Si definisce numerale sinistro la sequenza di caratteri, letti da sinistra a
destra a partire dal primo 1, che si trovano strettamente a sinistra della
testina della MdT, dove 1 rappresenta la cifra 1 e $B$ rappresenta la cifra 0.
\end{defi}
\begin{defi}[Numero sinistro]
Si dice numero sinistro l'interpretazione nei naturali del numerale sinistro,
ovvero quel numero $n \in \mathbb{N}$ la cui rappresentazione in base 2 è
esattamente il numerale sinistro.
\end{defi}
\begin{defi}[Numerale destro]
È la sequenza di caratteri a destra della testina (compreso quello sopra la
testina stessa) lette da destra verso sinistra, a partire dal primo 1 che
compare sul nastro.
\end{defi}
\begin{defi}[Numero destro]
È l'interpretazione nei naturali del numerale destro.
\end{defi}

È facile comprendere da queste definizioni che il nastro e la testina della
macchina di Turing possono essere rappresentati da una coppia di interi, ovvero
il numero sinistro e il numero destro. Inoltre il carattere esattamente sotto
la testina è 1 se il numero destro è dispari, 0 se pari.

\begin{esempio}
Per questa macchina di Turing
\begin{align*}
\dots BB11B&111B1111BB \dots\\
&\,\uparrow
\end{align*}
il numerale sinistro è 1101, il numero sinistro è 13; il numerale destro è
1111011, il numero destro è 123.
\end{esempio}

Ora dobbiamo poter simulare su questa coppia di numeri le operazioni che la
macchina di Turing svolge sul nastro. Queste sono:
\begin{enumerate}
 \item scrivere un 1 nella casella sulla quale sta la testina;
 \item scrivere uno 0 sulla casella;
 \item spostare la testina a destra (R);
 \item spostare la testina a sinistra (L);
\end{enumerate}

Per le prime due è facile trovare una funzione ricorsiva che modifica il
numero destro in modo da rappresentare la scrittura di un 1 o di uno 0: nel
primo caso basta sommare 1 se il numero è pari (ovvero nella posizione
sottostante alla testina della MdT simulata vi è uno 0), nel secondo caso è
sufficente sottrarre 1 se il numero è dispari.

Definiamo quindi le funzioni che svolgono queste due operazioni:
$$w_1(x) = \left\{
\begin{array}{ll}
x+1 & \text{se } 2 \mid x\\
x & \text{se } 2 \nmid x
\end{array}
\right.$$
che simula l'operazione di scrivere un 1, e
$$w_B(x) = \left\{
\begin{array}{ll}
x & \text{se } 2 \mid x\\
x-1 & \text{se } 2 \nmid x
\end{array}
\right.$$
che simula l'operazione di scrivere uno 0.

Per simulare lo spostamento della testina sul nastro bisogna osservare come
cambiano i numerali quando la testina si muove a destra o a sinistra. Siano $l$
e $r$ rispettivamente il numero sinistro e il numero destro prima dello
spostamento della testina, $l'$ ed $r'$ il numero sinistro ed il numero destro
dopo lo spostamento della testina.
\begin{enumerate}
 \item Se la testina si muove verso sinistra ed $l$ è pari allora vuol
   dire che a sinistra della testina c'è uno 0, che passa da cifra
   meno significativa del numerale sinistro ad essere la cifra meno
   significativa del numerale destro.  Quindi $l' = l/2$ ed $r' = r
   \cdot 2$.
 \item Se la testina si muove verso sinistra ed $l$ è pari allora vuol
   dire che a sinistra della testina c'è un 1, che passa dal numerale
   sinistro al numerale destro; quindi $l' = (l-1)/2$ ed $r' = r \cdot
   2 + 1$.
 \item Se la testina si muove verso destra e la cifra meno
   significativa del numerale destro è un 1, questa passa al numerale
   sinistro per cui: $l' = l \cdot 2 + 1$, $r' = (r-1)/2$.
 \item Se la testina si muove verso destra e la cifra meno
   significativa del numerale destro è uno 0, $l' = l \cdot 2$ e $r' =
   r/2$.
\end{enumerate}

Vediamo ora come codificare le quintuple $\langle q_i, S_i, S_j,
\{L,R\}, q_j \rangle$ che descrivono una macchina di Turing.

Queste definiscono una funzione di transizione $\tau : \mathbb{N}^2
\rightarrow \mathbb{N}^3$ e, poichè le nostre funzioni ricorsive sono
tutte della forma $f:\mathbb{N}^k \rightarrow \mathbb{N}$, dobbiamo
trovare una funzione di codifica delle ennuple $\nu: \mathbb{N}^n
\rightarrow \mathbb{N}$ ed una classe di funzioni di proiezione
$\nu_i: \mathbb{N} \rightarrow \mathbb{N}$ che estrapolano la
proiezione dell'i-esima componente dall'ennupla codificata nei
naturali.

Delle funzioni adatte a questo scopo sono le seguenti:
$$\nu(\vec{x}) = \prod_{i=1}^n p_i^{x_i}$$ per la codifica
e $$\nu_i(w) = \max_x.p_i^x \mid w$$ per la decodifica dell'i-esima
componente, dove $p_i$ indica l'i-esimo numero primo. Queste funzioni
sfruttano l'unicità della fattorizzazione in numeri primi dei
naturali.

La funzione di codifica è chiaramente primitiva ricorsiva, in quanto
composizione di funzioni primitive ricorsive, ovvero il prodotto e
l'elevamento a potenza.

Per verificare che anche la funzione di proiezione è primitiva
ricorsiva dobbiamo introdurre gli operatori di minimizzazione e
massimizzazione limitata, rispettivamente $\mu_{w<n}[f](\vec{x})$ e
$\max_{w<n}[f](\vec{x})$, che calcolano il minimo e il massimo $w \in
[0,n]$ tale per cui $f(\vec{x}, w) = 0$.

$$\mu_{w<n}[f](\vec{x}) = \sum_{i=0}^n sgn\left(\prod_{j=0}^i
f(\vec{x},w)\right)$$

Sia $h(\vec{x},y) = f(\vec{x},n-y)$, definiamo la funzione di
massimizzazione in questo modo:

$$\max_{w<n}[f](\vec{x}) = n-\mu_{w<n}[h](\vec{x})$$

Chiaramente sia la minimizzazione che la massimizzazione limitate sono
funzioni primitive ricorsive, in quanto definite in termini di altre
funzioni primitive ricorsive.

Ora definiamo la funzione caratteristica della divisione:

\begin{align*}
div(x,y) =& \left\{
\begin{array}{ll}
1 & \text{se } x \mid y\\
0 & \text{se } x \nmid y
\end{array}
\right. \\
=& \, sgn\left( \left| \mu_{w<y+1}.\left|x \cdot w - y \right| - (y+1)\right|
\right)
\end{align*}

Infine possiamo definire la funzione di proiezione nel seguente modo:
$$\nu_i(w) = \max_{x<w}.\left| div(p_i^x, w) - 1\right|$$ e quindi
abbiamo dimostrato che anche la massimizzazione è primitiva rcorsiva.

Con gli strumenti che ci siamo appena procurati possiamo ora costruire
la codifica della funzione di transizione. Intanto ci serve una mappa
dei simboli che usiamo per descrivere la macchina di Turing nei
naturali:
\begin{align*}
q_i &\rightsquigarrow i+1\\
B &\rightsquigarrow 0\\
1 &\rightsquigarrow 1\\
L &\rightsquigarrow 1\\
R &\rightsquigarrow 2\\
\end{align*}

La funzione di transizione, dato un insieme $S$ di regole per la
macchina di Turing, è la seguente:
$$\tau(x,y) = \left\{
\begin{array}{ll}
\nu(u,v,w) & \text{se $\langle q_i, S_i, S_j, D, q_j \rangle \in
  S$}\\ & \text{e $q_i \rightsquigarrow x$, $S_i \rightsquigarrow y$,
  $S_j \rightsquigarrow u$, $D \rightsquigarrow v$, $q_j
  \rightsquigarrow w$}\\ \nu(0,0,0) & \text{se non si applica nessuna
  regola}
\end{array}
\right.$$ è una funzione definita per casi su un insieme finito di
regole $S$, per cui, come dimostrato nei paragrafi precedenti, è
primitiva ricorsiva.

Codifichiamo ora la funzione che esegue sul nastro la tripla ottenuta
dalla funzione di transizione. Siano $left(l,r)$ e $right(l,r)$ le
funzioni che simulano lo spostamento della testina rispettivamente a
sinistra ed a destra secondo le regole date in precedenza. Un passo
della MdT è codificato dalla seguente funzione:

$$step(l,r,x) = \left\{
\begin{array}{ll}
left(l, w_0(r)) \cdot p_3^{\nu_3(x)} & \text{se }\nu_1(x) = 0, \nu_2(x) = 1\\
left(l, w_1(r)) \cdot p_3^{\nu_3(x)} & \text{se }\nu_1(x) = 1, \nu_2(x) = 1\\
right(l, w_0(r)) \cdot p_3^{\nu_3(x)} & \text{se }\nu_1(x) = 0, \nu_2(x) = 2\\
right(l, w_1(r)) \cdot p_3^{\nu_3(x)} & \text{se }\nu_1(x) = 1, \nu_2(x) = 2\\
\end{array}
\right.$$

Questa funzione, definita per ricorsione primitiva, ci da lo stato
della macchina dopo $t$ passi:
\begin{align*}
f(l,r,0) &= \nu(l,r,1)\\
f(l,r,t+1) &= step(\nu_1(n), \nu_2(n), \tau(\nu_3(n), \nu_2(n) \mod 2) )
\end{align*}
con $n = f(l,r,t)$.

Vediamo come codificare una k-tupla di una funzione ricorsiva nel valore $r$.
Intanto codifichiamo un singolo elemento:
\begin{align*}
g(r,0) &= 2\cdot r + 1\\ 
g(r,x+1) &= 2\cdot f(r, x) + 1\\
\end{align*}
usando questa funzione codifichiamo la tupla:
$$cod(x_1, \dots, x_n) = g(\dots 2\cdot g(0, x_n), x_1)$$
e decodifichiamo il risultato:
$$dec(r) = \mu_x.r+1\dot{-}2^x$$

Ora non ci resta che costruire la funzione che esegue la macchina di
Turing sull'input e che ritorna la codifica del nastro alla
terminazione: $$MdT(\vec{x}) = dec(\nu_2(f(0,cod(\vec{x}),\mu_t .
|\nu_3(f(0,cod(\vec{x}),t)) = 0|)))$$

In questo modo abbiamo definito una funzione ricorsiva che svolge
l'intera computazione di una macchina di Turing, se questa termina, ed
è indefinita altrimenti. In effetti si può dire qualcosa in più: fino
alla fine della nostra costruzione abbiamo usato solo funzioni
primitive ricorsive, quindi totali per la
Proposizione~\ref{PRsonoTotali}. Ovvero è sempre possibile sapere qual
è lo stato della computazione della macchina di Turing (rappresentato
dai numerali sinistro e destro) ad un dato tempo $t$, la vera
incognita sta nella terminazione della macchina, com'era giusto
aspettarsi.

È interessante inoltre notare che la costruzione di questa funzione è
quasi del tutto indipendente dalla macchina di Turing che andiamo a
simulare con la nostra funzione ricorsiva, se non per la funzione di
transizione $\tau$ che non è altro che la rappresentazione
dell'insieme di regole date in pasto alla macchina. Potremmo dunque
benissimo pensare sostituire $\tau$ con un'altra funzione, più
sofisticata, che prende in input le quintuple codificate, lo stato
iniziale e il carattere letto e ritorna la tripla che codifica il
carattere da scrivere, la direzione in cui spostare la testina e lo
stato finale; questa funzione è in grado di prendere una qualsiasi
macchina di Turing codificata ed eseguirla su un qualsiasi
input. Inserendo questa particolare funzione di transizione e
sostituendola nel nostra prova otteniamo la macchina di Turing
universale, in grado cioè di eseguire una qualunque macchina di
Turing.

Abbiamo quindi dimostrato che per ogni macchina di Turing esiste una
funzione ricorsiva che la computa ed abbiamo visto come costruirla.
\end{proof}


\subsection{Tesi di Church-Turing}
Poichè tutti questi (ed altri) tentativi di definire tutto ciò che è
effettivamente calcolabile portano alla stessa classe di funzioni, si
può pensare che la nostra nozione intuitiva di calcolabilità coincida
esattamente con ognuna di queste definizioni.


% Giuseppe Ferri

\section{Esempio di funzione non ricorsiva}
La definizione di funzione non ricorsiva che si intende dare è basata sul processo di definizione delle funzioni ricorsive. In particolare, si definisce un \emph{sistema di nomi di arietà} per le funzioni ricorsive.

\begin{itemize}
 \item La funzione unaria \emph{zero} che ad ogni input associa il valore 0 viene denotata con il nome \emph{$Z^1$}, dove l'indice 1 ricorda che si tratta di una funzione ad un argomento.
\item La funzione unaria \emph{succ} che ad ogni input \emph{x} associa il valore \emph{x}+1 viene denotata con il nome \emph{$S^1$}, dove l'indice 1 ricorda che si tratta di una funzione ad un argomento.
\item La funzione \emph{n}-aria \emph{$proj^n_i$}, con \emph{$1 \leq i\leq n$}, che ad ogni ennupla (\emph{$x_1,...,x_n$}) associa il valore \emph{$x_i$} viene denotata con il nome \emph{$P^n_i$}, dove l'indice \emph{n} ricorda che si tratta di una funzione a \emph{n} argomenti.
\item Se \emph{$F^m$},  \emph{$G^n_1$},...,\emph{$G^n_m$} sono i nomi delle \emph{m} + 1 funzioni \emph{f}, \emph{$g_1$},...,\emph{$g_m$} rispettivamente la prima di arietà \emph{m} e, tutte le altre, di arietà \emph{n}, allora \emph{$C^n$[$F^m$,$G^n_1$,..., $G^n_m$]} è il nome della funzione \emph{h} di arietà \emph{n} ottenuta componendo \emph{f} con le funzioni \emph{$g_1,...,g_m$}, ossia \emph{$h(\vec{x}) = f(g_1(\vec{x}),...,g_m(\vec{x}))$}.
\item Se \emph{$K^n$} e \emph{$G^{n+2}$} sono i nomi delle funzioni \emph{k} e \emph{g} di arietà rispettivamente \emph{n} e \emph{n+2} allora \emph{$R^{n+1}$[$K^n$, $G^{n+2}$]} è il nome della funzione \emph{f} di arietà \emph{n+1} ottenuta per ricorsione primitiva da \emph{k} e \emph{g}, ossia \emph{$f(0, \vec{y}) =k(\vec{y})$} e \emph{$f(x+1, \vec{y}) =g(x, \vec{y}, f(x, \vec{y}))$}.
\item Se \emph{$G^{n+1}$} è il nome della funzione \emph{g} di arietà \emph{n+1} allora \emph{$M^n$[$G^{n+1}$]} è il nome della funzione \emph{f} di arietà \emph{n} ottenuta da \emph{g} per minimizzazione, ossia \emph{$f(\vec{x}) := \mu y.g(\vec{x}, y) = 0$}.
\end{itemize}


Possiamo quindi dare la seguente definizione.

\begin{defi}[Funzione ricorsiva]
Una funzione è \emph{ricorsiva} se e solo se ha un nome.
\end{defi}
 
In effetti una funzione ricorsiva può avere molti nomi, tuttavia un nome \emph{F} definisce esattamente una funzione ricorsiva che indicheremo con \emph{$f_F$} quando vorremmo mettere l'accento sul fatto che la definizione della funzione dipende dal suo nome.
\begin{esempio}
[somma]
La funzione somma si può definire induttivamente come:
$$\left\{
\begin{array}{ll} y + 0 = y\\
                 y + succ(x)= succ(x + y)
\end{array} \right.$$
Si pone il segno di addizione + in forma più esplicita, ottenendo:
$$\left\{
\begin{array}{ll} somma(0, y) = y\\
                 somma(succ(x), y) = succ(somma(x, y))
\end{array} \right.$$
Ed infine mettiamo queste equazioni nel formato ufficiale per la ricorsione:

$$\left\{
\begin{array}{ll} somma(0, y) = P^1_1(y)\\
                 somma(succ(x), y) = C^3[S^1, P^3_3](x, y, somma(x, y))
\end{array} \right.$$

ed in breve si ottiene il nome da dare alla somma:
$$Sum^2 := R^2[P^1_1, C^3[S^1, P^3_3]]$$	
\end{esempio}

\begin{esempio}
[prodotto]

$$\left\{
\begin{array}{ll} prod(x, 0) = zero(x) = Z^1(x)\\
                 prod(x, succ(y)) = somma(x, prod(x, y)) = C^3[Sum^2, P^3_1, P^3_3]](x, y, prod(x, y))
\end{array} \right.$$
Quindi il nome del prodotto è: 
$$Prod^2 := R^2[Z^1, C^3[Sum^2, P^3_1, P^3_3]]$$	
\end{esempio}




\begin{defi}[Grado di un nome]
Il \emph{grado del nome F di una funzione ricorsiva}, indicato con \emph{$\delta$(F)}, è il numero di simboli che appaiono nella scrittura di F, cioè il numero di Z, S, $P^n_i$, C[-,-,...,-], R[-,-] e M[-].
\end{defi}

Notiamo che ci sono una infinità numerabile di funzioni ricorsive di qualsiasi grado visto che possiamo costruire infiniti nomi utilizzando questi segni. Tuttavia, se poniamo un vincolo al numero di segni utilizzabili vale il seguente teorema. 

\begin{thm}[Finitezza]
 Il numero di funzioni ricorsive di arietà 1 a cui si può dare un nome utilizzando al più n simboli è finito.
\end{thm}

\begin{proof}
Se per costruire il nome posso usare al massimo \emph{n} segni, allora posso al più utilizzare \emph{Z, S, C, R, M} e $P^k_i$ con \emph{$k \leq n$} e con questi posso solo scrivere nomi di lunghezza finita; perciò il numero di nomi diversi possibile è finito.
 \end{proof}

Vediamo ora come definire una funzione e dimostrare che non è ricorsiva.

\begin{defi}[Produttività di una funzione ricorsiva]
La \emph{produttività $p_F$} della funzione ricorsiva $f_F$ di nome F è il valore di $f_F$ su 0 se tale valore esiste ($f_F$ potrebbe essere una funzione parziale), 0 altrimenti.
\end{defi}

Possiamo allora definire la funzione di produttività \emph{p} nel modo seguente: $$p(n)=max_{\delta(F)\leq 2n + 1} p_F$$

Si nota che la funzione di produttività \emph{p} è ben definita visto che, in conseguenza del teorema di finitezza, per ogni numero naturale \emph{n}, c'è un numero finito di funzioni ricorsive il cui nome si scrive con al massimo \emph{2n} + 1 simboli e quindi è possibile calcolare il massimo tra questo numero finito di valori.\\
Per dimostrare che la funzione di produttività non è ricorsiva ne studiamo alcune proprietà molto elementari.

\begin{prop}Per ogni n, $p(n) \geq n$.
\end{prop}
\begin{proof} 
Consideriamo il nome da dare alla funzione unaria che ad ogni input associa il valore n:
$$N_n := C[S, C[S,...C[S, Z]...]]$$
dove ci sono \emph{n} coppie \emph{C} e \emph{S}. 
Osserviamo che $\delta(N_n) = 2n + 1$ e quindi con \emph{2n} + 1 segni possiamo definire una funzione ricorsiva in grado di produre almeno \emph{n} come output quando applicata a 0.
\end{proof}

\begin{prop}Per ogni n, $p(n + 1) > p(n)$.
\end{prop}
\begin{proof} 
Consideriamo una funzione ricorsiva \emph{f} il cui nome \emph{F} richiede al massimo 2\emph{n} + 1 simboli e che produca come output \emph{p(n)} quando applicata a 0. Allora la funzione di nome \emph{C[S, F]} richiede al massimo 2 + 2\emph{n} + 1 = 2(\emph{n} + 1) + 1 simboli e produce su 0 un output pari a \emph{$p(n) + 1 > p(n)$} e quindi \emph{$p(n + 1) \geq p(n) + 1 > p(n)$}. 
\end{proof}

La proposizione precedente mostra chiaramente che la funzione \emph{p} è strettamente monotona e quindi per ogni \emph{i} e \emph{j}, se \emph{$p(i) \leq p(j)$} allora \emph{$i \leq j$}.

\begin{prop}Per ogni n, $p(n + 5) \geq 2n$.
\end{prop}
\begin{proof} 
Vediamo come definire una funzione ricorsiva che raddoppi il suo input. Definiamo per ricorsione una funzione \emph{d} così definita:
$$\left\{
\begin{array}{ll} d(0) = 0\\
                d(x + 1) = succ(succ(d(x))
\end{array} \right.$$
 Allora è immediato verificare che un nome per questa funzione è 
$$D := R[M[Z], C[S, C[S, P^2_2]]]$$ 
che richiede 8 segni.
Se ora noi componiamo la funzione di nome \emph{$N_n$} che abbiamo definito in precedenza con la funzione di nome \emph{D} appena introdotta otteniamo il nome \emph{C[D, $N_n$]} che richiede $1 + 8 + 2n + 1 \leq 2(n + 5) + 1$ segni e definisce una funzione unaria che su input 0 produce output 2\emph{n}. 
\end{proof}

Siamo ora in grado di dimostrare che la funzione \emph{p} non può essere ricorsiva.


\begin{thm}La funzione di produttività non è ricorsiva.
\end{thm}
\begin{proof}
Supponiamo che \emph{p} sia una funzione ricorsiva. \\ Allora dovrebbe avere un nome \emph{P} e tale nome avrebbe un certo numero \emph{$k_p$} di simboli. Consideriamo ora il nome \emph{C[P, C[P, $N_n$]]}. Si tratta di un nome che richiede 1+ \emph{$k_p$} + 1 + \emph{$k_p$} + 2\emph{n} + 1 = 2(\emph{n} + 1 + \emph{$k_p$}) + 1 segni e che descrive una funzione ricorsiva il cui valore su 0 è uguale a \emph{p(p(n))}. Quindi \emph{$p(n + 1 + k_p) \geq p(p(n))$}. Ma allora in virtù della stretta monotonia della funzione \emph{p} otteniamo che, per ogni \emph{n}, \emph{$n + 1 + k_p \geq p(n)$}. \\Visto che tale disequazione vale per ogni \emph{n}, il risultato deve valere anche quando sostituiamo a \emph{n} il valore \emph{n + 5} ottenendo \emph{$n + 6 + k_p \geq p(n + 5)$}. Ma ora, in virtù della proposizione precedente, possiamo ricavare \emph{$n + 6 + k_p \geq p(n + 5) \geq 2n$} da cui si deduce che, per ogni \emph{n},  \emph{$k_p \geq n - 6$} che è assurdo visto che \emph{$k_p$} è il numero, fissato in partenza per ipotesi, dei simboli necessari per scrivere il nome della funzione di produttività. 
 \end{proof}

%\include{03_insiemi_re}
\include{04_ha}
\include{04_a_linguaggi}
\chapter{Rappresentabilità}


\section{Introduzione e definizioni di base}

Nel capitolo precedente abbiamo introdotto la teoria formale $HA$ (Aritmetica di Heyting) come equivalente costruttivo della teoria $PA$ basata sugli assiomi di Peano. Lo scopo di questo capitolo \`e vedere cosa tale teoria $HA$ \`e in grado di dimostrare.\\
Infatti quando costruiamo una teoria formale, vogliamo che il sistema deduttivo all'interno di tale teoria sia in grado di provare tutto ci\`o che noi siamo in grado di fare informalmente (e quindi al di fuori di questa teoria).\\
Per arrivare a questo risultato diamo qualche definizione necessaria alla comprensione della trattazione successiva.\\

\begin{defi}
Sia $T$ una teoria nel linguaggio $\mathcal{L}_{A}$ dell'aritmetica. Diciamo che la relazione\footnote{Quando parliamo di relazioni (o funzioni) ci riferiamo sempre a relazioni (o funzioni) che hanno come argomenti numeri naturali.} $R$ ad $n$ argomenti \`e \underline{esprimibile} in $T$ sse esiste una formula $\phi(x_{1},\ldots,x_{n})$ di $T$, con $x_{1}, \ldots, x_{n}$ variabili libere, tale che, per ogni numero naturale $k_{1}, \ldots,k_{n}$, valgono le seguenti:
\begin{enumerate}
  \item se $R(k_{1}, \ldots,k_{n})$ \`e vera, allora $\vdash_{T} \phi(\overline{k}_{1}, \ldots, \overline{k}_{n})$\footnote{Dove $\overline{k}_{1}, \ldots,\overline{k}_{n}$ sono, come abbiamo visto, numerali.};
  \item se $R(k_{1}, \ldots,k_{n})$ \`e falsa, allora $\vdash_{T} \neg \phi(\overline{k}_{1}, \ldots, \overline{k}_{n})$.
\end{enumerate}
\end{defi}
\vspace{0.1cm}
\underline{Esempio}: la relazione di uguaglianza \`e esprimibile in $HA$ dalla formula $\phi(x_{1},x_{2})\equiv x_{1}=x_{2}$. Infatti, per ogni numero naturale $k_{1}, k_{2}$ valgono:\\
 \begin{enumerate}
  \item se $k_{1}=k_{2}$ (ovvero $R(k_{1},k_{2})$ \'e vera), ricordando che in $HA$ se $m=n$ allora $\vdash_{HA} \overline{m}=\overline{n}$\footnote{Come ᅵ stato dimostrato nel precedente capitolo.}, si ottiene $\vdash_{HA} \overline{k}_{1}=\overline{k}_{2}$ (ovvero $\vdash_{HA}\phi(\overline{k}_{1},\overline{k}_{2}))$);
  \item se $k_{1}\neq k_{2}$ (ovvero $R(k_{1},k_{2})$ \'e falsa), ricordando che in $HA$ se $m\neq n$ allora $\vdash_{HA} \overline{m}\neq \overline{n}$\footnote{Come ᅵ stato dimostrato nel precedente capitolo.}, si ottiene $\vdash_{HA} \overline{k}_{1}\neq \overline{k}_{2}$ (ovvero $\vdash_{HA} \neg \phi(\overline{k}_{1},\overline{k}_{2})$).
\end{enumerate}
\vspace{0.5cm}
\begin{defi}
Sia $T$ una teoria con uguaglianza nel linguaggio $\mathcal{L}_{A}$ dell'aritmetica. Diciamo che una funzione $f$ ad $n$ argomenti \`e \underline{rappresentabile} in $T$ sse esiste una formula $\phi(x_{1}, \ldots,x_{n}, y)$ di $T$, con $x_{1}, \ldots, x_{n}, y$ variabili libere, tale che, per ogni numero naturale $k_{1}, \ldots, k_{n}, m$, valgono le seguenti:
\begin{enumerate}
  \item se $f(k_{1}, \ldots,k_{n})=m$, allora $\vdash_{T} \phi(\overline{k}_{1}, \ldots, \overline{k}_{n}, \overline{m})$;
  \item $\vdash_{T} (\exists ! y) \phi(\overline{k}_{1}, \ldots, \overline{k}_{n}, y)$.
\end{enumerate}
La condizione $(2)$, in presenza della $(1)$, pu\`o essere sostituita dalla seguente:
\begin{enumerate}
\item[(2')] $\vdash_{T} \forall\ y (\phi(\overline{k_1},\ldots,\overline{k_n}, y)\rightarrow y = \overline{f(k_1,\ldots,k_n)})$.
\end{enumerate}
\end{defi}
\vspace{0.1cm}
\underline{Esempio}: la funzione somma \'e rappresentabile in $HA$ dalla formula $\phi(x_{1},x_{2},y)\equiv x_{1}+x_{2}=y$. Infatti, per ogni numero naturale $k_{1}, k_{2}, m$ valgono:\\
 \begin{enumerate}
  \item se $k_{1}+k_{2}=m$ (ovvero $f(k_{1}, k_{2})=m$) allora ricordando che in $HA$ se $a+b=c$ allora $\vdash_{HA} \overline{a}+\overline{b}=\overline{c}$\footnote{Come si puᅵ ricavare dalle proprietᅵ dimostrate nel precedente capitolo.}, si ha che $\vdash_{HA} \overline{k}_{1}+\overline{k}_{2}=\overline{m}$ (ovvero $\vdash_{HA} \phi(\overline{k}_{1}, \overline{k}_{2}, \overline{m})$);
  \item $\vdash_{HA}(\exists ! y) (\overline{k}_{1} + \overline{k}_{2}= y)$ vale banalmente.
\end{enumerate}
\vspace{0.5cm}
\begin{defi}
Una relazione $R$ si dice \underline{complementata} se per essa vale il principio del terzo escluso, ovvero se, per ogni $k_1,\ldots,k_n$, vale che $R(k_1,\ldots,k_n)$ \`e vera oppure $\neg R(k_1,\ldots,k_n)$ \`e vera.
\end{defi}
\vspace{0.7cm}

\section{Risultati}
La seguente proposizione esprime una definizione equivalente a quella giᅵ data di rappresentabilit\`a.

\begin{prop}
Sia $T$ una teoria nel linguaggio $\mathcal{L}_{A}$ dell'aritmetica. Una funzione $f$ a $n$ argomenti \`e rappresentabile con una formula $\phi (\vec{x}$\footnote{D'ora in poi adotteremo la notazione vettoriale per indicare in generale $n$ argomenti, ovvero $\vec{x}=(x_{1},\ldots, x_{n})$.}$, m)$ sse per ogni naturale $k_{1},\ldots k_{n}, m$ vale la condizione:
$$se \ \ f(\vec{k}) = m \ \ allora \vdash_{T} \forall\ y\ (\phi(\vec{\overline{k}}, y)\leftrightarrow y = \overline{m}).$$
\end{prop}
\vspace{0.2cm}
\textsc{\textbf{Dim:}} ($\Rightarrow$) L'ipotesi che $f$ sia rappresentabile con la formula $\phi(\vec{x},m)$ porge le seguenti condizioni per ogni naturale $k_{1},\ldots k_{n}, m$:
\begin{enumerate}
\item se $f(\vec{k})=m$ allora $\vdash_{T} \phi(\vec{\overline{k}}, \overline{m})$;
\item [(2')]$\vdash_{T} \forall y (\phi(\vec{\overline{k}}, y)\rightarrow y=\overline{f(\vec{k})})$.
\end{enumerate}
Inoltre se $f(\vec{k})$ = m si ottiene $\overline{f(\vec{k})} = \overline{m}$, per cui la condizione $(2')$ si puᅵ riscrivere come:
\begin{enumerate}
\item [(2')]$\vdash_{T} \forall y (\phi(\vec{\overline{k}}, y)\rightarrow y=\overline{m})$.
\end{enumerate}
\`E immediato ricavare la conclusione $\vdash_{T} \forall y (\phi(\vec{\overline{k}}, y)\leftrightarrow y = \overline{m})$ grazie alla seguente derivazione che ᅵ stata suddivisa in due alberi per migliorarne la leggibilit\`a.\\
Per prima cosa ricaviamo $\vdash_{T}\phi(\vec{\overline{k}}, y)\rightarrow y = \overline{m}$ come segue utilizzando $(2')$:

$$\prooftree
\vdash_{T} \forall y (\phi(\vec{\overline{k}}, y)\rightarrow y = \overline{m})
\quad
\[\phi(\vec{\overline{k}}, y)\rightarrow y = \overline{m}\vdash_{T}\phi(\vec{\overline{k}}, y)\rightarrow y = \overline{m}\using{\forall_{left}}
\justifies
\forall y (\phi(\vec{\overline{k}}, y)\rightarrow y = \overline{m})\vdash_{T} \phi(\vec{\overline{k}}, y)\rightarrow y = \overline{m}\]\using{cut}
\justifies
\vdash_{T}\phi(\vec{\overline{k}}, y)\rightarrow y = \overline{m}
\endprooftree$$
\\
\begin{flushleft}
Pertanto utilizzando $(1)$ otteniamo:
\end{flushleft}

$$\prooftree
\[\vdash_{T}\phi(\vec{\overline{k}}, y)\rightarrow y = \overline{m}
\quad
\[\[\vdash_{T} \phi(\vec{\overline{k}},\overline{m})
\quad
\phi(\vec{\overline{k}},\overline{m}), y = \overline{m} \vdash_{T} \phi(\vec{\overline{k}},y) \using{cut}
\justifies
y = \overline{m}\vdash_{T} \phi(\vec{\overline{k}}, y)\]\using{\rightarrow_{right}}
\justifies
\vdash_{T} y = \overline{m}\rightarrow \phi(\vec{\overline{k}}, y)\]\using{\&_{right}}
\justifies
\vdash_{T}\phi(\vec{\overline{k}}, y)\leftrightarrow y = \overline{m}\]\using{\forall_{right}}
\justifies
\vdash_{T} \forall y (\phi(\vec{\overline{k}}, y)\leftrightarrow y = \overline{m})
\endprooftree$$
\\
\begin{flushleft}
($\Leftarrow$) Vogliamo dimostrare che, se per ogni naturale $k_{1},\ldots k_{n}, m$ vale la:
$$se \ \ f(\vec{k}) = m \ \ allora \vdash_{T} \forall\ y\ (\phi(\vec{\overline{k}}, y)\leftrightarrow y = \overline{m}),$$ allora le condizioni $(1)$ e $(2)$ sono verificate.
\end{flushleft}
\vspace{0.1cm}
$(1)$ Valendo la condizione appena scritta si pu\`o dimostrare come segue che se $f(\vec{k}) = m$ allora $\vdash_{T} \phi(\vec{\overline{k}}, \overline{m})$:
\vspace{0.5cm}
$$\prooftree
\vdash_{T}\forall y (\phi(\vec{\overline{k}}, y)\leftrightarrow y = \overline{m})
\quad
\[\[\[\vdash_{T} \overline{m} = \overline{m} \quad \phi(\vec{\overline{k}}, \overline{m})\vdash_{T} \phi(\vec{\overline{k}}, \overline{m})\using{\rightarrow_{left}}
\justifies
\overline{m}=\overline{m}\rightarrow\phi(\vec{\overline{k}}, \overline{m})\vdash_{T} \phi(\vec{\overline{k}}, \overline{m}) \]\using{\&_{left}}
\justifies
\phi(\vec{\overline{k}}, \overline{m})\leftrightarrow \overline{m} = \overline{m}\vdash_{T} \phi(\vec{\overline{k}}, \overline{m})\]\using{\forall_{left}}
\justifies
\forall y (\phi(\vec{\overline{k}}, y)\leftrightarrow y = \overline{m})\vdash_{T} \phi(\vec{\overline{k}}, \overline{m})\]\using{cut}
\justifies
\vdash_{T} \phi(\vec{\overline{k}}, \overline{m})
\endprooftree$$
\vspace{0.5cm}
\begin{flushleft}
$(2)$ Con la stessa ipotesi la conclusione $\vdash_{T} (\exists ! y)\phi({\vec{\overline{k}}},y)$ ᅵ raggiungibile tramite la seguente derivazione che ᅵ stata spezzata in due alberi per renderla pi\`u leggibile:
\end{flushleft}
\vspace{0.5cm}
{\scriptsize{$$\prooftree
\[\[
\[\[\[\vdash_{T} \overline{m} = \overline{m}
\quad
\phi(\vec{\overline{k}},\overline{m})\vdash_{T}\phi(\vec{\overline{k}},\overline{m})\using{\rightarrow_{left}}
\justifies
\overline{m}=\overline{m}\rightarrow\phi(\vec{\overline{k}},\overline{m})\vdash_{T}\phi(\vec{\overline{k}},\overline{m})\]\using{\&_{left}}
\justifies
\phi(\vec{\overline{k}},\overline{m})\leftrightarrow \overline{m} = \overline{m}\vdash_{T} \phi(\vec{\overline{k}}, \overline{m})\]\using{\forall_{left}}
\justifies
\forall y (\phi(\vec{\overline{k}},y)\leftrightarrow y = \overline{m})\vdash_{T} \phi(\vec{\overline{k}}, \overline{m})\]
\quad
\[\[\[\phi(\vec{\overline{k}},t)\rightarrow t = \overline{m}\vdash_{T}
\phi(\vec{\overline{k}},t)\rightarrow t = \overline{m}\using{\&_{left}}
\justifies
\phi(\vec{\overline{k}},t)\leftrightarrow t = \overline{m}\vdash_{T}
\phi(\vec{\overline{k}},t)\rightarrow t = \overline{m}\]\using{\forall_{left}}
\justifies
\forall y (\phi(\vec{\overline{k}},y)\leftrightarrow y = \overline{m})\vdash_{T}
\phi(\vec{\overline{k}},t)\rightarrow t = \overline{m}\]\using{\forall_{right}}
\justifies
\forall y (\phi(\vec{\overline{k}},y)\leftrightarrow y = \overline{m})\vdash_{T}
\forall z(\phi(\vec{\overline{k}},z)\rightarrow z = \overline{m})\]\using{\&_{right}}
\justifies
\forall y (\phi(\vec{\overline{k}},y)\leftrightarrow y = \overline{m})\vdash_{T} \phi(\vec{\overline{k}},\overline{m})\& \forall z(\phi(\vec{\overline{k}},z)\rightarrow z = \overline{m})\]\using{\exists_{right}}
\justifies
\forall y (\phi(\vec{\overline{k}},y)\leftrightarrow y = \overline{m})\vdash_{T} \exists y (\phi(\vec{\overline{k}},y)\& \forall z(\phi(\vec{\overline{k}},z)\rightarrow z = y))\]
\justifies
\forall y (\phi(\vec{\overline{k}},y)\leftrightarrow y = \overline{m})\vdash_{T} \exists ! y\phi(\vec{\overline{k}},y)
\endprooftree$$}}
\\
\begin{flushleft}
Pertanto:
\vspace{0.2cm}
\end{flushleft}
$$\prooftree
\vdash_{T} \forall y (\phi(\vec{\overline{k}},y)\leftrightarrow y = \overline{m})
\quad
\forall y (\phi(\vec{\overline{k}},y)\leftrightarrow y = \overline{m})\vdash_{T} \exists ! y\phi(\vec{\overline{k}},y)\using{cut}
\justifies
\vdash_{T}\exists ! y\phi(\vec{\overline{k}},y)
\endprooftree$$
\hspace{\stretch{1}} $\Box$\\

\begin{prop}
Sia $T$ una teoria con uguaglianza nel linguaggio $\mathcal{L}_{A}$ tale che $\vdash _{T} 0\neq1$. Allora una relazione complementata $R$ \`e esprimibile in $T$ sse $\chi_{R}$ \`e rappresentabile in $T$.
\end{prop}
\vspace{0.2cm}
\textsc{\textbf{Dim:}} ($\Rightarrow$) Iniziamo con il supporre che $R$ sia esprimibile tramite la formula $\phi(\vec{x})$, ovvero che per ogni $\vec{\overline{k}}$ con $n$ componenti naturali valgano:
\begin{itemize}
  \item [(a)] se $R(\vec{k})$ \`e vera, allora $\vdash_{T} \phi(\vec{\overline{k}})$;
  \item [(b)] se $R(\vec{k})$ \`e falsa, allora $\vdash_{T} \neg \phi(\vec{\overline{k}})$.
\end{itemize}
Ricordiamo che la funzione caratteristica $\chi_{R}$ associata alla relazione $R$ ᅵ definita come segue:
$$\chi_{R}(\vec{k})=
\begin{cases} 1 & \text{se $R(\vec{k})$ \'e vera} \\ 0 & \text{se $R(\vec{k})$ \'e falsa}
\end{cases}$$\\
Vogliamo dimostrare che $\chi_{R}$ \`e rappresentata dalla formula $$\psi(\vec{x},y) \equiv (\phi(\vec{x}) \& y = \overline{1}) \lor (\neg \phi(\vec{x})\& y=0).$$
Per fare ci\`o occorre far vedere che la formula $\psi(\vec{x},y)$ soddisfa le due condizioni $(1)$ e $(2)$ date nella definizione 2 per ogni numero naturale $k_{1}, \ldots, k_{n}, m$.\\
\\
$(1)$ Vogliamo provare che se $\chi_{R}(\vec{k})=m$ allora $\vdash_{T} \psi(\vec{\overline{k}},\overline{m})$.\\
Distinguiamo due casi: $m=1$ o $m=0$. Supponiamo $m=1$ (il caso $m=0$ si dimostra in modo analogo).\\
Per come \'e definita la $\chi_{R}$ sappiamo che $R(\vec{k})$ \`e vera e, per (a), abbiamo come conseguenza che $\vdash_{T} \phi(\vec{\overline{k}})$.\\ 
Con la seguente derivazione otteniamo proprio quello che dovevamo dimostrare:
      $$\prooftree
      \[ \vdash_{T} \phi(\vec{\overline{k}}) \quad \vdash_{T} \overline{1}=\overline{1}
      \using {\&_{right}}
       \justifies
      \vdash_{T} \phi(\vec{\overline{k}}) \& \overline{1}=\overline{1} \]
       \justifies
      \vdash_{T} \psi(\vec{\overline{k}},\overline{1})
      \endprooftree $$
\\
$(2)$ Resta da dimostrare che $\vdash_{T} \exists ! y((\phi(\vec{\overline{k}}) \& y=\overline{1}) \lor (\neg \phi(\vec{\overline{k}})\& y=0))$. Per cercare di rendere pi\`u chiara la derivazione proviamo a suddividerla in vari pezzi.\\
Iniziamo con il far vedere che $\phi(\vec{\overline{k}}) \vdash_{T} (\phi(\vec{\overline{k}}) \& \overline{1}=\overline{1}) \lor (\neg \phi(\vec{\overline{k}}) \& \overline{1}=0)$:\\
      $$ \prooftree
      \[ \phi(\vec{\overline{k}}) \vdash_{T} \phi(\vec{\overline{k}})
      \quad
      \[ \vdash_{T} \overline{1}=\overline{1}
      \using {ind}
      \justifies
      \phi(\vec{\overline{k}}) \vdash_{T} \overline{1}=\overline{1} \]
      \using {\&_{right}}
      \justifies
      \phi(\vec{\overline{k}}) \vdash_{T}\phi(\vec{\overline{k}}) \& \overline{1}=\overline{1} \]
      \using {\vee_{right}}
      \justifies
      \phi(\vec{\overline{k}}) \vdash_{T} (\phi(\vec{\overline{k}}) \& \overline{1}=\overline{1}) \lor (\neg \phi(\vec{\overline{k}}) \& \overline{1}=0)
      \endprooftree $$\\

\begin{flushleft}
Mostriamo quindi che $\phi(\vec{\overline{k}}) \vdash_{T} \exists ! y  ((\phi(\vec{\overline{k}}) \& y=\overline{1}) \lor (\neg \phi(\vec{\overline{k}})\& y=0))$:\\
\vspace{0.5cm}
\end{flushleft}
\tiny     
      $$ \prooftree
      \[ \[
       \phi(\vec{\overline{k}}) \vdash_{T} (\phi(\vec{\overline{k}}) \& \overline{1}=\overline{1}) \lor (\neg \phi(\vec{\overline{k}}) \& \overline{1}=0)
      \quad
     \[ \[ \[
      \[ \[ z=\overline{1} \vdash_{T} z=\overline{1}
      \using {\&_{left}}
      \justifies
      \phi(\vec{\overline{k}}) \& z=\overline{1} \vdash_{T} z=\overline{1} \]
      \using {ind}
     \justifies
      \phi(\vec{\overline{k}}), \phi(\vec{\overline{k}}) \& z=\overline{1} \vdash_{T} z=\overline{1} \]
      \quad
     \[ \[ \phi(\vec{\overline{k}}) \vdash_{T} \phi(\vec{\overline{k}})
      \using {\neg_{left}}
      \justifies
      \phi(\vec{\overline{k}}), \neg \phi(\vec{\overline{k}}) \vdash_{T} z=\overline{1} \]
      \using {\&_{left}}
      \justifies
      \phi(\vec{\overline{k}}), \neg \phi(\vec{\overline{k}}) \& z=0 \vdash_{T} z=\overline{1} \]
      \using {\vee_{left}}
      \justifies
      \phi(\vec{\overline{k}}), (\phi(\vec{\overline{k}}) \& z=\overline{1}) \lor (\neg \phi(\vec{\overline{k}}) \& z=0) \vdash_{T} z=\overline{1} \]
      \using {\rightarrow_{right}}
      \justifies
      \phi(\vec{\overline{k}}) \vdash_{T} (\phi(\vec{\overline{k}}) \& z=\overline{1}) \lor (\neg \phi(\vec{\overline{k}}) \& z=0) \to z=\overline{1}\]
      \using {\forall_{right}}
      \justifies
      \phi(\vec{\overline{k}}) \vdash_{T} \forall z((\phi(\vec{\overline{k}}) \& z=\overline{1}) \lor (\neg \phi(\vec{\overline{k}}) \& z=0) \to z=\overline{1})\]
      \using {\&_{right}}
      \justifies
      \phi(\vec{\overline{k}}) \vdash_{T} ((\phi(\vec{\overline{k}}) \& \overline{1}=\overline{1}) \lor (\neg \phi(\vec{\overline{k}}) \& \overline{1}=0))\&\forall z((\phi(\vec{\overline{k}}) \& z=\overline{1}) \lor (\neg \phi(\vec{\overline{k}}) \& z=0) \to z=\overline{1})\]
      \using {\exists_{right}}
      \justifies
      \phi(\vec{\overline{k}}) \vdash_{T} \exists y ((\phi(\vec{\overline{k}}) \& y=\overline{1}) \lor (\neg \phi(\vec{\overline{k}}) \& y=0))\&\forall z((\phi(\vec{\overline{k}}) \& z=\overline{1}) \lor (\neg \phi(\vec{\overline{k}}) \& z=0) \to z=y)\]
      \justifies
      \phi(\vec{\overline{k}}) \vdash_{T} \exists ! y  ((\phi(\vec{\overline{k}}) \& y=\overline{1}) \lor (\neg \phi(\vec{\overline{k}})\& y=0))
      \endprooftree $$
\normalsize
\\
\begin{flushleft}
Analogamente si ha $\neg \phi(\vec{\overline{k}}) \vdash_{T} \exists ! y  ((\phi(\vec{\overline{k}}) \& y=\overline{1}) \lor (\neg \phi(\vec{\overline{k}})\& y=0))$.\end{flushleft} A questo punto possiamo arrivare alla tesi con la seguente derivazione che, utilizzando i risultati appena ottenuti e il fatto che essendo $R$ complementata ed esprimibile si ha $\vdash_{T} \phi(\vec{\overline{k}}) \lor \neg\phi(\vec{\overline{k}})$, otteniamo:\\
\vspace{0.4cm}
\tiny
      $$ \prooftree
      \vdash_{T} \phi(\vec{\overline{k}}) \lor \neg\phi(\vec{\overline{k}})
      \quad
      \[ \phi(\vec{\overline{k}}) \vdash_{T} \exists ! y ((\phi(\vec{\overline{k}}) \& y=\overline{1}) \lor (\neg \phi(\vec{\overline{k}})\& y=0)) \quad \neg \phi(\vec{\overline{k}}) \vdash_{T} \exists ! y ((\phi(\vec{\overline{k}}) \& y=\overline{1}) \lor (\neg \phi(\vec{\overline{k}})\& y=0))
      \using{\vee_{left}}
      \justifies
      \phi(\vec{\overline{k}}) \lor \neg\phi(\vec{\overline{k}}) \vdash_{T} \exists ! y ((\phi(\vec{\overline{k}}) \& y=\overline{1}) \lor (\neg \phi(\vec{\overline{k}})\& y=0)) \]
      \justifies
      \vdash_{T} \exists ! y((\phi(\vec{\overline{k}}) \& y=\overline{1}) \lor (\neg \phi(\vec{\overline{k}})\& y=0))
      \using {cut}
      \endprooftree $$
\normalsize
\vspace{0.5cm}      
\begin{flushleft}
($\Leftarrow$) Supponiamo ora che $\chi_{R}$ sia rappresentata dalla formula $\psi(\vec{x},y)$,\end{flushleft} ovvero supponiamo che valgano (1) e (2) per ogni naturale $k_{1}, \ldots, k_{n},m$. Mostriamo che $R$ \`e esprimibile tramite $\phi(\vec{x})\equiv \psi(\vec{x}, \overline{1})$. Devono quindi essere soddisfatte le condizioni $(a)$ e $(b)$ per ogni naturale $k_{1}, \ldots, k_{n}$.\\
(a) Supponiamo $R(\vec{k})$ vera. Allora $\chi_{R}(\vec{k})=1$ e, per il punto $(1)$, otteniamo $\vdash_{T} \psi (\vec{\overline{k}}, \overline{1})$.\\
(b) Sia ora $R(\vec{k})$ falsa. Allora $\chi_{R}(\vec{k})=0$ e, sempre per il punto $(1)$, abbiamo $\vdash_{T} \psi (\vec{\overline{k}}, 0)$.
Inoltre, per il punto (2), abbiamo la condizione di unicit\`a $\vdash_{T} \exists ! u(\psi(\vec{\overline{k}},u))$ che possiamo riscrivere come $$\vdash_{T} \exists u(\psi(\vec{\overline{k}},u) \& \forall v(\psi(\vec{\overline{k}},v) \to u=v)).$$ Allora, usando questi due risultati e ricordando l'ipotesi $\vdash_{T} 0\neq\overline{1}$, otteniamo la tesi mediante la seguente derivazione che abbiamo suddiviso in due parti per migliorarne la leggibilit\`a:
\\
\\
\scriptsize
      $$\prooftree
      \[
      \[
      \[
      \[
      \[
      \[    \psi(\vec{\overline{k}},0) \vdash_{T} \psi(\vec{\overline{k}},0) \quad
      \[    \psi(\vec{\overline{k}},\overline{1}) \vdash_{T} \psi(\vec{\overline{k}},\overline{1}) \quad
      \[    u=0, u=\overline{1} \vdash_{T} 0= \overline{1} \quad
      \[    \vdash_{T} 0\neq\overline{1}
     \using{\neg_{left}}
      \justifies
      0=\overline{1} \vdash_{T} \bot \]
     \using{cut}
      \justifies
      u=0, u=\overline{1} \vdash_{T} \bot \]
     \using{\rightarrow_{left}}
      \justifies
      \psi(\vec{\overline{k}},\overline{1}), u=0, \psi(\vec{\overline{k}},\overline{1}) \to u=\overline{1} \vdash_{T} \bot \]
     \using{\rightarrow_{left}}
      \justifies
      \psi(\overline{\vec{k}},\overline{1}) ,\psi(\vec{\overline{k}},0),\psi(\vec{\overline{k}},0) \to u=0, \psi(\vec{\overline{k}},\overline{1}) \to u=\overline{1} \vdash_{T} \bot \]
      \using{\neg_{right}}
      \justifies
      \psi(\vec{\overline{k}},0),\psi(\vec{\overline{k}},0) \to u=0, \psi(\vec{\overline{k}},\overline{1}) \to u=\overline{1} \vdash_{T} \neg \psi(\vec{\overline{k}},\overline{1}) \]
       \using{\forall_{left}}
      \justifies
      \psi(\vec{\overline{k}},0),\psi(\vec{\overline{k}},0) \to u=0, \forall v(\psi(\vec{\overline{k}},v) \to u=v) \vdash_{T} \neg \psi(\vec{\overline{k}},\overline{1}) \]
      \using{\forall_{left}}
      \justifies
      \psi(\vec{\overline{k}},0),\forall v(\psi(\vec{\overline{k}},v) \to u=v), \forall v(\psi(\vec{\overline{k}},v) \to u=v) \vdash_{T} \neg \psi(\vec{\overline{k}},\overline{1}) \]
      \using{cont.}
      \justifies
      \psi(\vec{\overline{k}},0),  \forall v(\psi(\vec{\overline{k}},v) \to u=v), \vdash_{T} \neg \psi(\vec{\overline{k}},\overline{1})\]
      \using{\&_{left}}
      \justifies
      \psi(\vec{\overline{k}},0), \psi(\vec{\overline{k}},u) \& \forall v(\psi(\vec{\overline{k}},v) \to u=v) \vdash_{T} \neg \psi(\vec{\overline{k}},\overline{1})\]
      \using{\exists_{left}}
      \justifies
      \psi(\vec{\overline{k}},0), \exists u(\psi(\vec{\overline{k}},u) \& \forall v(\psi(\vec{\overline{k}},v) \to u=v)) \vdash_{T} \neg   \psi(\vec{\overline{k}},\overline{1})
      \endprooftree $$
\\
\\
\tiny      
$$ \prooftree
      \vdash_{T} \exists u(\psi(\vec{\overline{k}},u) \& \forall v(\psi(\vec{\overline{k}},v) \to u=v)) 
      \quad
      \[
      \vdash_{T} \psi(\vec{\overline{k}},0)
      \quad
      \psi(\vec{\overline{k}},0), \exists u(\psi(\vec{\overline{k}},u) \& \forall v(\psi(\vec{\overline{k}},v) \to u=v)) \vdash_{T} \neg      \psi(\vec{\overline{k}},\overline{1})
      \using{cut}
      \justifies
      \exists u(\psi(\vec{\overline{k}},u) \& \forall v(\psi(\vec{\overline{k}},v) \to u=v)) \vdash_{T} \neg   \psi(\vec{\overline{k}},\overline{1})\]
      \using{cut}
      \justifies
      \vdash_{T} \neg   \psi(\vec{\overline{k}},\overline{1})
\endprooftree $$
\\
\normalsize
\hspace{\stretch{1}} $\Box$\\






Grazie a questo risultato ora siamo liberi di scegliere se usare l'esprimibilit\`a delle relazioni o la rappresentabilit\`a delle funzioni per dimostrare che il sistema $HA$ \`e in grado di provare meccanicamente tutto ci\`o che noi siamo in grado di fare informalmente. Poich\`e, fin dall'inizio abbiamo "`preferito"' le funzioni, ci\`o che faremo ora sar\`a dimostrare che ogni funzione ricorsiva \`e rappresentabile in $HA$. Per arrivare a questo dobbiamo prima vedere alcuni risultati che useremo poi nella dimostrazione finale. \\
Visto che da questo momento in poi ci riferiamo al sistema formale $HA$ lo lasceremo sottointeso, quindi per non appesantire la notazione useremo $\vdash$ sottointendendo $\vdash_{HA}$.

\begin{defi}
Chiamiamo \underline{funzione $\beta$ di G\"odel} la funzione cos\`i definita:
$$ \beta (x_{1}, x_{2}, x_{3}):=rm(1+(1+x_{3})x_{2}, x_{1})$$
dove con $rm(\cdot,\cdot)$ indichiamo la funzione che, dati due numeri $x$ e $y$, restisuisce il resto della divisione di $y$ per $x$. 
\end{defi}
\underline{Esempi}:
\begin{enumerate}
  \item $rm(13,27)=1$, infatti $27=13\cdot2+1$;
  \item $\beta(395,15,21)=rm(1+(1+21)\cdot15,395)=rm(331,395)=64$ infatti $395=331\cdot1+64$.
\end{enumerate}

Osserviamo che la funzione $rm(\cdot,\cdot)$ appena introdotta \`e primitiva ricorsiva, infatti la possiamo definire come\footnote{Dove indichiamo con $S$ la funzione successore e con $sg$ la funzione segno.}:

$$
\left \{ \begin{array} {ll}
rm(x,0)=0 \\
rm(x,y+1)=S(rm(x,y))\cdot sg(x- S(rm(x,y)))
\end{array} \right. 
$$ \newline

\begin{prop}
La funzione $\beta(x_{1}, x_{2}, x_{3})$ \`e rappresentabile in $S$ tramite la formula
$$ Bt(x_{1}, x_{2}, x_{3}, y)\equiv \exists w ((x_{1}=(1+(x_{3}+1)x_{2})w +y )\&(y<1+(x_{3}+1)x_{2})) $$
\end{prop}

\textsc{\textbf{Dim:}} Dobbiamo far vedere che per $Bt$ valgono le solite due condizioni. Dati $k_{1}, k_{2}, k_{3}, m \in \mathbb{N}$:
\begin{enumerate}
  \item Supponiamo $\beta(\vec{k})=m$. Ci\`o vuol dire che $k_{1}= (1+(k_{3}+1)k_{2})k + m$ per qualche $k \in \mathbb{N}$ e $m<1+(k_{3}+1)k_{2}$. Allora per ci\`o che \`e stato visto nel capitolo precedente \footnote{Usiamo il fatto che, dati $n, m \in \mathbb{N}$, se $n=m$ allora $\vdash \overline{n}=\overline{m}$.} e perch\'e la relazione $<$ \`e esprimibile abbiamo: $\vdash \overline{k}_{1}= (\overline{1}+(\overline{k}_{3}+\overline{1})\overline{k}_{2})\overline{k} + \overline{m}$ e $\vdash \overline{m}<\overline{1}+(\overline{k}_{3}+\overline{1})\overline{k}_{2}$. Quindi
      $$ \prooftree
      \[
      \vdash \overline{k}_{1}= (\overline{1}+(\overline{k}_{3}+\overline{1})\overline{k}_{2})\overline{k} + \overline{m} \quad
      \vdash \overline{m}<\overline{1}+(\overline{k}_{3}+\overline{1})\overline{k}_{2}
      \using{\&_{left}}
      \justifies
      \vdash ( \overline{k}_{1}= (\overline{1}+(\overline{k}_{3}+\overline{1})\overline{k}_{2})\overline{k} + \overline{m}) \& (\overline{m}<\overline{1}+(\overline{k}_{3}+\overline{1})\overline{k}_{2}) \]
      \using{\exists_{right}}
      \justifies
      \vdash \exists w ((\overline{k}_{1}=(1+(\overline{k}_{3}+1)\overline{k}_{2})w +\overline{m} )\& (\overline{m}<1+(\overline{k}_{3}+1)\overline{k}_{2}))
      \endprooftree $$
      Percio` $ \vdash Bt(\overline{k}_{1}, \overline{k}_{2}, \overline{k}_{3}, \overline{m})$.
  \item Per vedere che anche questa condizione \`e soddisfatta basta ricordare un risultato ottenuto in precedenza:
      %$$\vdash z\neq0 \to \exists u \exists v [x=z\cdot u+v \land v<z \land (\forall u_{1} \forall %v_{1}(x=z\cdot u_{1}+v_{1} \land v_{1}<z) \to u=u_{1} \land v=v_{1} )] $$
      %cio\`e,
      $$\vdash z\neq0 \to \exists ! u \exists ! v [(x=z\cdot u+v )\& (v<z)] $$
      Allora, se consideriamo $k_{1}, k_{2}, k_{3}$, $z=1+(k_{3}+1)k_{2}=S((k_{3}+1)k_{2})$ e $x=k_{1}$ otteniamo ovviamente $z\neq0$ e 
      $$ \prooftree
      \vdash z \neq 0 \quad z \neq 0 \vdash \exists ! u \exists ! v [(x=z\cdot u+v) \& (v<z)]
      \justifies
      \vdash \exists ! u Bt(k_{1}, k_{2}, k_{3}, u)
      \endprooftree $$
\end{enumerate}
\hspace{\stretch{1}} $\Box$\\

\begin{lem}
Per ogni sequenza $k_{0},\ldots, k_{n}$ di numeri naturali, esistono $b, c \in \mathbb{N}$ t.c. $\beta(b,c,i)=k_{i}$ per $0\leq i \leq n$.
\end{lem}

\textsc{\textbf{Dim:}} Sia $q= \max\{n, k_{0},\ldots,k_{n}\}$ e poniamo $c=q!$. \\Consideriamo $u_{i}=1+(i+1)c$ per $0\leq i\leq n$.
Allora, se $i\neq j$, $u_{i}$ e $u_{j}$ sono coprimi: infatti supponiamo che esista $p$ primo t.c. $p\mid u_{i}$ e $p\mid u_{j}$ con $i<j$ (analogo con $j<i$). Quindi $p\mid u_{j}-u_{i}=(j-i)c$.\\
Se $p\mid c$, allora abbiamo che $p\mid (i+1)c$ ma anche $p\mid u_{i}=1+(i+1)c$ e quindi $p\mid 1$ che \`e un assurdo.\\
Se $p\mid j-i$ si ha $0<j-i\leq n\leq q$ e quindi $p\mid q!=c$, cosa che abbiamo appena escluso.\\
Quindi per ogni $i,j\leq n$ con $i\neq j$, $u_{i}$ e $u_{j}$ sono coprimi ed inoltre, per $0\leq i\leq n$, $k_{i}\leq q\leq q!= c<1+(i+1)c=u_{i}$, quindi $k_{i}<u_{i}$ per ogni $i$. Per il Teorema Cinese del Resto esiste $b<u_{0}u_{1}\cdot\cdot\cdot u_{n}$ t.c. $rm(u_{i}, b)=k_{i}$ e quindi abbiamo che $\beta(b,c,i)=rm(1+(i+1)c,b)=rm(u_{i},b)=k_{i}$.
\hspace{\stretch{1}} $\Box$\\



Gli ultimi risultati ci permettono di esprimere in $T$ asserzioni riguardanti sequenze finite di numeri naturali e questo giocher\`a un ruolo cruciale nella dimostrazione del teorema seguente. Per semplicare la notazione, scriveremo x' in luogo di s(x) per indicare il successore di x.

\begin{thm}
Ogni funzione ricorsiva \`e rappresentabile in $T$.
\end{thm}

\textsc{\textbf{Dim:}} Questa dimostrazione si svolge per induzione sulla complessit\`a strutturale delle funzioni, quindi partiremo con il dimostrare che le funzioni base sono rappresentabili e poi mostreremo che l'essere rappresentabile \`e una propriet\`a chiusa rispetto alle regole che ci permettono di costruire nuove funzioni. \\
Funzioni base:
\begin{itemize}
  \item La funzione zero $Z(x)$ \`e rappresentabile dalla formula $x_{1}=x_{1} \land y=0$. Infatti, dati $k, m \in \mathbb{N}$ :
      \begin{enumerate}
        \item Supponiamo $Z(k)=m$. Allora $m=0$ e, per quanto visto, $\vdash \overline{m}=0$. Quindi
            $$ \prooftree
            \vdash \overline{k}=\overline{k} \quad \vdash \overline{m}=0
            \justifies
            \vdash \overline{k}=\overline{k} \land \overline{m}=0
            \endprooftree $$
        \item Basta notare:
        $$ \prooftree
        \[ \vdash x_{1}=x_{1} \quad \vdash 0=0 \quad
        \[ \[ \[ z=0 \vdash z=0
        \justifies
        x_{1}=x_{1} \land z=0\vdash z=0 \]
        \justifies
        \vdash (x_{1}=x_{1} \land z=0) \to z=0 \]
        \justifies
        \vdash \forall z((x_{1}=x_{1} \land z=0) \to z=0) \]
        \justifies
        \vdash x_{1}=x_{1} \land 0=0 \land \forall z((x_{1}=x_{1} \land z=0) \to z=0) \]
        \justifies
        \vdash \exists y(x_{1}=x_{1} \land y =0 \land \forall z((x_{1}=x_{1} \land z=0) \to z=y) )
        %GIUSTO?
        \endprooftree $$
      \end{enumerate}
  \item La funzione successore $S(x)=x+1$ \`e rappresentata dalla formula $y=x_{1}'$. Infatti, dati $k, m \in \mathbb{N}$ :
      \begin{enumerate}
        \item Supponiamo $S(k)=m$. Allora $m=k+1$ e quindi $\vdash \overline{m}= \overline{k}'$.
        \item \`E facile vedere che :
        $$ \prooftree
        \[ \vdash x_{1}'=x_{1}' \quad
        \[ \[ z= x_{1}' \vdash z= x_{1}'
        \justifies
        \vdash z= x_{1}' \to z= x_{1}' \]
        \justifies
        \vdash \forall z (z= x_{1}' \to z= x_{1}') \]
        \justifies
        \vdash x_{1}'=x_{1}' \land \forall z (z= x_{1}' \to z= x_{1}') \]
        \justifies
        \vdash \exists ! y (y=x_{1}')
        \endprooftree
        $$
      \end{enumerate}
  \item Analogamente alle precedenti, \`e facile vedere che la funzione \\ $P_{i}^{n}(x_{1},\ldots,x_{n})=x_{i}$ \`e rappresentata dalla formula \\ $x_{1}=x_{1}\land\ldots\land x_{n}=x_{n} \land y=x_{i}$. \\
\end{itemize}
Passiamo ora alle regole introdotte per costruire nuove funzioni: \\
\begin{itemize}
  \item \textbf{Composizione} \\
  Sia $f(\vec{x})=g(h_{1}(\vec{x}),\ldots,h_{m}(\vec{x}))$ una funzione ad $n$ argomenti. Allora, per ipotesi induttiva, $g(x_{1},\ldots,x_{m}), h_{1}(x_{1},\ldots,x_{n}),\ldots, h_{m}(x_{1},\ldots,x_{n})$ sono rappresentabilibi in $S$. Siano $\psi(\vec{x},z), \phi_{1}(\vec{x},y_{1}),\ldots,\phi_{m}(\vec{x},y_{m})$ \footnote{Per semplicit\`a notazionale uso il vettore $\vec{x}$, ma, come per le rispettive funzioni, bisogna stare attenti a non far confusione sul numero di argomenti. Quello in $\psi$ \`e un vettore di lunghezza $m$, mentre quelli nelle $\phi$ sono vettori di lunghezza $n$} le formule che le rappresentano. Dimostriamo allora che $f$ \`e rappresentata da:
  $$\eta(\vec{x},z)\equiv\exists y_{1}\ldots\exists y_{m}(\phi_{1}(\vec{x},y_{1})\land\ldots\land\phi_{m}(\vec{x},y_{m})\land\psi(y_{1},\ldots,y_{m},z)) $$
  \begin{enumerate}
    \item Dati $k_{i},r_{j},q \in \mathbb{N}$, con $0\leq i\leq n$ e $0\leq j\leq m$, supponiamo $f(\vec{k})=q$ e $h_{j}(\vec{k})=r_{j}$ per $0\leq j\leq m$. Ci\`o implica $g(\vec{r})=q$. Allora, per ipotesi induttiva, si ha:
        $$\vdash\phi_{j}(\overline{\vec{k}},\overline{r}_{j}) \ ,\ \vdash \psi(\overline{\vec{r}},\overline{q}), \ con \ 0\leq j\leq m.$$
        Quindi possiamo subito concludere applicando $m$ volte l'$\land$-formazione e $m$ volte l'$\exists$-formazione.
    \item L'esistenza, cio\`e $\vdash \exists z \ \eta(\vec{x},z)$, segue facilmente dall'esistenza per le $\phi_{j}, \psi$. Allora non resta che dimostrare l'unicit\`a e cio\`e che $\vdash \eta(\vec{x},u)\land \eta(\vec{x}, v) \to u=v$. Supponiamo $j=1$ per semplicit\`a, ma si pu\`o facilmente generalizzare a $j=m$. Allora, usando l'unicit\`a per le $\phi_{1}, \phi_{2}, \psi$ e saltando qualche piccolo passaggio, si ha:
        $$ \prooftree
        \[ \[ \phi_{1}(\vec{x}, b_{1}), \phi_{1}(\vec{x}, c_{1}) \vdash b_{1}=c_{1}
        \quad
        \[ b_{1}=c_{1}, \psi(b_{1},u), \psi(c_{1}, v) \vdash \psi(b_{1},u), \psi(b_{1},v)
        \quad
        \psi(b_{1},u), \psi(b_{1},v) \vdash u=v
        \justifies
        b_{1}=c_{1}, \psi(b_{1},u), \psi(c_{1}, v) \vdash u=v
        \]
        \justifies
        \phi_{1}(\vec{x}, b_{1}), \phi_{1}(\vec{x}, c_{1}), \psi(b_{1},u), \psi(c_{1}, v) \vdash u=v
        \]
        \justifies
        \eta(\vec{x},u) \land \eta(\vec{x},v) \vdash u=v \]
        \justifies
        \vdash \eta(\vec{x},u) \land \eta(\vec{x},v) \to u=v
        \endprooftree $$ \\
  \end{enumerate}
  \item \textbf{Ricorsione} \\
  Date $h$, $g$ consideriamo la funzione $f$ ad $n+1$ argomenti cos\`i definita:
  $$
  \left \{ \begin{array} {ll}
  f(\vec{x},0)= h(\vec{x}) \\
  f(\vec{x},y+1)=g(\vec{x},y,f(\vec{x},y))
  \end{array} \right.
  $$
  Allora, per ipotesi induttiva, $h$ e $g$ sono rappresentabili. $\phi(\vec{x},x_{n+1})$, $\psi(\vec{x},x_{n+1},x_{n+2},x_{n+3})$ siano, rispettivamente, le formule che le rappresentano. \\
  Osserviamo che $f(\vec{x},y)=z$ sse esiste una sequenza finita di numeri naturali $b_{0},\ldots,b_{y}$ tali che $b_{0}=h(\vec{x})$,\ldots,$b_{w+1}=f(\vec{x},w+1)=g(\vec{x},w,b_{w})$ per $w+1\leq y$, quindi $b_{y}=z$. Per il Lemma, possiamo riferirci a questa sequenza di numeri tramite la funzione $\beta$ e abbiamo visto che $\beta$ \`e rappresentata in $S$ dalla formula $Bt(x_{1},x_{2},x_{3},y)$. Mostriamo che allora $f$ \`e rappresentabile tramite la formula \\ $\eta(\vec{x},x_{n+1},x_{n+2})$ cos\`i definita:
  $$\exists u\exists v[\exists w(Bt(u,v,0,w)\land \phi(\vec{x},w)) \land Bt(u,v,x_{n+1},x_{n+2}) \land $$
  $$\ \ \forall w(w<x_{n+1} \to \exists y\exists z(Bt(u,v,w,y)\land Bt(u,v,w',z)\land \psi(\vec{x},w,y,z)))] $$
  \begin{enumerate}
    \item Dati $\vec{k}, p, m \in \mathbb{N}$, supponiamo $f(\vec{k},p)=m$. Vorremmo mostrare che $\vdash \eta(\overline{\vec{k}},\overline{p},\overline{m})$. Distinguiamo due casi: \\
        \\
        \textbf{($\mathbf{p=0}$)} Quindi $m=h(\vec{k})$. Consideriamo la sequenza formata solo da $m$. Allora sappiamo, per il Lemma, che esistono $b$, $c$ tali che $\beta(b,c,0)=m$. Ma $\beta$ ed $h$ sono rappresentabili, quindi:
        $$ \prooftree
        \vdash Bt(\overline{b},\overline{c},0,\overline{m})
        \quad
        \[ \[ \vdash Bt(\overline{b},\overline{c},0,\overline{m})
        \quad
        \vdash \phi(\overline{\vec{k}},\overline{m})
        \justifies
        \vdash Bt(\overline{b},\overline{c},0,\overline{m}) \land \phi(\overline{\vec{k}},\overline{m}) \]
        \justifies
        \vdash \exists w(Bt(\overline{b},\overline{c},0,w) \land \phi(\overline{\vec{k}},w)) \]
        \justifies
        \vdash \exists w(Bt(\overline{b},\overline{c},0,w) \land \phi(\overline{\vec{k}},w)) \land Bt(\overline{b},\overline{c},0,\overline{m})
        \endprooftree $$
        Inoltre, poich\'e per ogni termine $t$ vale che $\vdash t\geq0$, si ha:
        $$ \prooftree
        \[ \[ w<0\vdash \bot \quad \bot \vdash \exists y\exists z(Bt(\overline{b},\overline{c},w,y)\land Bt(\overline{b},\overline{c},w',z)\land \psi(\overline{\vec{k}},w,y,z))
        \justifies
         w<0\vdash  \exists y\exists z(Bt(\overline{b},\overline{c},w,y)\land Bt(\overline{b},\overline{c},w',z)\land \psi(\overline{\vec{k}},w,y,z)) \]
         \justifies
         \vdash w<0 \to  \exists y\exists z(Bt(\overline{b},\overline{c},w,y)\land Bt(\overline{b},\overline{c},w',z)\land \psi(\overline{\vec{k}},w,y,z)) \]
         \justifies
         \vdash \forall w(w<0 \to  \exists y\exists z(Bt(\overline{b},\overline{c},w,y)\land Bt(\overline{b},\overline{c},w',z)\land \psi(\overline{\vec{k}},w,y,z)))
         \endprooftree $$ \\
         Ora applicando una volta l'$\land$-formazione e due volte l'$\exists$-formazione otteniamo il risultato voluto. \\
         \\
         \textbf{($\mathbf{p>0}$)} Allora $f(\vec{k},p)$ \`e calcolata in $p+1$ passi. Sia $r_{j}=f(\vec{k},j)$. Allora, per il Lemma, sappiamo che data la sequenza $r_{0},\ldots,r_{p}$ esistono $b$, $c$ tali che $\beta(b,c,i)=r_{i}$ per $0\leq i\leq p$. E quindi, per la rappresentabilit\`a di $\beta$, $\vdash Bt(\overline{b},\overline{c},\overline{i},\overline{r}_{i})$. \\
         In particolare $r_{0}=\beta(b,c,0)$ e $r_{0}=f(\vec{k},0)=h(\vec{k})$, quindi:
         $$ \prooftree
         \[ \vdash Bt(\overline{b},\overline{c},0,\overline{r}_{0}) \quad
         \vdash \phi(\overline{\vec{k}}, \overline{r}_{0})
         \justifies
         \vdash Bt(\overline{b},\overline{c},0,\overline{r}_{0}) \land \phi(\overline{\vec{k}}, \overline{r}_{0}) \]
         \justifies
         \vdash \exists w Bt(\overline{b},\overline{c},0,w) \land \phi(\overline{\vec{k}}, w)
         \endprooftree $$
         Inoltre da $r_{p}=f(\vec{k},p)=m$, poich\'e $r_{p}=\beta(b,c,p)$, si ha:
         $$ \vdash Bt(\overline{b},\overline{c},\overline{p},\overline{m}) $$
         Ora, per $0<i\leq p-1$, valgono $\beta(b,c,i)=r_{i}=f(\vec{k},i)$ e $\beta(b,c,i+1)=r_{i+1}=f(\vec{k},i+1)=g(\vec{k},i,r_{i})$. Quindi, usando l'ipotesi induttiva, si ha:
         $$ \prooftree
         \[ \vdash Bt(\overline{b},\overline{c},\overline{i},\overline{r}_{i}) \quad
         \vdash Bt(\overline{b},\overline{c},\overline{i+1},\overline{r}_{i+1}) \quad
         \vdash \psi(\overline{\vec{k}},\overline{i},\overline{r}_{i},\overline{r}_{i+1})
         \justifies
         \vdash Bt(\overline{b},\overline{c},\overline{i},\overline{r}_{i}) \land Bt(\overline{b},\overline{c},\overline{i+1},\overline{r}_{i+1}) \land \psi(\overline{\vec{k}},\overline{i},\overline{r}_{i},\overline{r}_{i+1}) \]
         \Justifies
         \vdash \exists y \exists z( Bt(\overline{b},\overline{c},\overline{i},y) \land Bt(\overline{b},\overline{c},\overline{i+1},z) \land \psi(\overline{\vec{k}},\overline{i},y,z))
         \endprooftree $$
         per ogni $0<i\leq p-1$. \\
         Utilizzando il il fatto che per ogni numero naturale $p>0$ ed ogni formula $\phi$ si ha $\vdash\phi(0)\land \phi(\overline{1})\land\ldots\land \phi(\overline{p-1}) \leftrightarrow \forall x(x<\overline{p} \to \phi(x))$, otteniamo:
         $$ \vdash \forall w(w<p \to \exists y \exists z( Bt(\overline{b},\overline{c},\overline{i},y) \land Bt(\overline{b},\overline{c},\overline{i+1},z) \land \psi(\overline{\vec{k}},\overline{i},y,z))) $$
         A questo punto basta unire i risultati ottenuti e utilizzando l'$\land$-formazione e l'$\exists$-formazione si arriva al risultato voluto:
         $$\vdash \eta(\overline{\vec{k}}, \overline{p}, \overline{m})$$ \\
    \item Dobbiamo dimostrare $\vdash \exists ! x_{n+2} \eta(\overline{\vec{k}},\overline{p},x_{n+2})$. Lo faremo per induzione su $p$ nel metalinguaggio. Notiamo che basta provare l'unicit\`a. \\
        Per \textbf{($\mathbf{p=0}$)}, allora $f(\vec{k},0)=h(\vec{k})$, quindi l'unicit\`a per $\eta$ segue dall'unicit\`a per $\phi$.

     Per \textbf{($\mathbf{p>0}$)}, assumiamo ora che $\vdash \exists!x_{n+2} \eta(\overline{\vec{k}},\overline{p},x_{n+2})$ e poniamo per semplicit\`a $\alpha=h(\vec{k})$, $\delta=f(\vec{k},p)$, $\gamma=f(\vec{k},p+1)=g(\vec{k},p,\delta)$. Allora
        \begin{description}
          \item[(1)] $\vdash \psi(\overline{\vec{k}},\overline{p},\overline{\delta},\overline{\gamma})$
          \item[(2)] $\vdash \phi(\overline{\vec{k}},\overline{\alpha})$
          \item[(3)] $\vdash \eta(\overline{\vec{k}},\overline{p},\overline{\delta})$
          \item[(4)] $\vdash \eta(\overline{\vec{k}},\overline{p+1},\overline{\gamma})$
          \item[(5)] $\vdash \exists!x_{n+2} \eta(\overline{\vec{k}},\overline{p},x_{n+2})$ \\
        \end{description}
        Assumiamo
        \begin{description}
        \item[(6)] $\vdash \eta(\overline{\vec{k}},\overline{p+1},x_{n+2})$ \\
        \end{description}

        Dobbiamo dimostrare che $x_{n+2}=\gamma$.
        Da $\textbf{(6)}$: \\
        \begin{description}
          \item[(a)] $\exists w(Bt(b,c,0,w)\land \phi(\overline{\vec{k}},w))$
          \item[(b)] $Bt(b,c,\overline{p+1},x_{n+2})$
          \item[(c)] $\forall w(w<\overline{p+1} \to \exists y\exists z(Bt(u,v,w,y)\land Bt(u,v,w',z)\land \psi(\overline{\vec{k}},w,y,z)))$ \\
        \end{description}
        Da $\textbf{(c)}$, utilizzando due volte l'equivalenza vista prima \footnote{Per ogni numero naturale $p>0$ ed ogni formula $\phi$ si ha $\vdash\phi(0)\land \phi(\overline{1})\land\ldots\land \phi(\overline{p-1}) \leftrightarrow \forall x(x<\overline{p} \to \phi(x))$}, si ottiene: \\
        \begin{description}
          \item[(d)] $\forall w(w<\overline{p} \to \exists y\exists z(Bt(u,v,w,y)\land Bt(u,v,w',z)\land \psi(\overline{\vec{k}},w,y,z)))$
          \item[(e)] $Bt(b,c,\overline{p},d) \land Bt(b,c, \overline{p+1},e) \land \psi(\overline{\vec{k}},\overline{p},d,e)$ \\
        \end{description}
        Ora, semplicemente dalla definizione di $\eta$ e da $\textbf{(a)}$, $\textbf{(d)}$ ed $\textbf{(e)}$ (dopo aver ``sciolto '' gli $\land$) si ha: \\
        \begin{description}
          \item [(f)] $\eta(\overline{\vec{k}},\overline{p},d)$ \\
        \end{description}
        che assieme all'ipotesi $\textbf{(5)}$ porta a: \\
        \begin{description}
          \item [(g)] $d=\overline{\delta}$ \\
        \end{description}
        Ora, con una semplice sostituzione, da $\textbf{(g)}$ ed $\textbf{(e)}$ otteniamo: \\
        \begin{description}
          \item [(h)] $\psi(\overline{\vec{k}},\overline{p},\overline{\delta},e)$ \\
        \end{description}
        ma, dall'unicit\`a di $\psi$ utilizzando l'ipotesi $\textbf{(1)}$ abbiamo: \\
        \begin{description}
          \item [(i)] $\overline{\gamma}=e$ \\
        \end{description}
        Con un'altra sostituzione, da $\textbf{(e)}$ e $\textbf{(i)}$ si ha: \\
        \begin{description}
          \item [(j)] $Bt(b,c,\overline{p+1},\overline{\gamma})$ \\
        \end{description}
        ed infine, da $\textbf{(b)}$,$\textbf{(j)}$ e dall'unicit\`a di $Bt$ otteniamo il risultato voluto: $x_{n+2}=\overline{\gamma}$. \\
        La dimostrazione \`e semplice come idee, ma lunga e complicata da scrivere come derivazione con il calcolo dei sequenti dato il numero di variabili diverse che bisogna utilizzare. In realt\'a basta stare attenti a quando si usa l'$\exists$-riflessione e ripercorrere le idee sopra viste utilizzando pi\`u che altro contrazione, indebolimento e taglio. \\
  \end{enumerate}

  \item \textbf{Operatore di Minimo} \\
  Consideriamo la funzione $f(\vec{x})=\mu[x]g(\vec{x})$ con $g(\vec{x},y)$ rappresentabile tramite la formula $\eta(\vec{x},x_{n+1},x_{n+2})$. Vediamo che $f$ \`e rappresentata da:
  $$\phi(\vec{x},x_{n+1})=\eta(\vec{x},x_{n+1},0) \land \forall y(y<x_{n+1} \to \neg \eta(\vec{x},y,0))$$
  \begin{enumerate}
    \item Siano $\vec{k}, m \in \mathbb{N}$. Supponiamo $f(\vec{k})=m$ e distinguiamo due casi: \\
        \textbf{($\mathbf{m=0}$)} Allora $g(\vec{k},0)=0$ e dalla rappresentabilit\`a di $g$ si ha
        $$\vdash \eta(\overline{\vec{k}},0,0)$$
        Inoltre:
        $$ \prooftree
        \[ \[
        y<0 \vdash \bot \quad \bot \vdash \neg \eta(\overline{\vec{k}},y,0)
        \justifies
        y<0 \vdash \neg \eta(\overline{\vec{k}},y,0)
         \using{cut} \]
         \justifies
         \vdash y<0 \to \neg \eta(\overline{\vec{k}},y,0) \]
         \justifies
         \vdash \forall y(y<0 \to \neg \eta(\overline{\vec{k}},y,0)) \\
         \endprooftree $$
         e con un'$\land$-formazione si conclude. \\
         \\
         \textbf{($\mathbf{m>0}$)} Allora $g(\vec{k},m)=0$ e  $g(\vec{k},l)\neq 0$ per $l<m$. Quindi, poich\`e per ipotesi induttiva $g$ \`e rappresentabile, abbiamo:
        $$\vdash \eta(\overline{\vec{k}}, \overline{m},0)$$
        $$\vdash \neg \eta(\overline{\vec{k}}, \overline{l},0) \ per \  l<m$$
        Quindi, utilizzando il fatto che per ogni $p>0$ ed ogni formula $\phi$ vale $\vdash\phi(0)\land \phi(\overline{1})\land\ldots\land \phi(\overline{p-1}) \leftrightarrow \forall x(x<\overline{p} \to \phi(x))$, otteniamo: \\
        $$ \prooftree
        \vdash \eta(\overline{\vec{k}}, \overline{m},0) \quad
        \[
        \vdash \neg \eta(\overline{\vec{k}},0,0) \quad \ldots \quad \vdash \neg \eta(\overline{\vec{k}},\overline{m-1},0)
        \Justifies
         %\proofdotseparation=1.2ex
         %\proofdotnumber=4
         %\leadsto
        \vdash \forall y(y<\overline{m} \to \neg \eta(\overline{\vec{k}},y,0))
         \]
        \justifies
        \vdash \phi(\overline{\vec{k}},\overline{m})
        \endprooftree $$ \\
    \item \`E sufficiente provare l'unicit\`a. Supponiamo $\phi(\overline{\vec{k}},\overline{m})$ e, utilizzando il fatto che per ogni $t$, $s$ vale $\vdash t<s \lor t=s \lor s<t$, proviamo che $\vdash \phi(\overline{\vec{k}},u) \to u=\overline{m}$. Osserviamo che: \\
        $$ \prooftree
        \[ \overline{m}<u \vdash \overline{m}<u \quad
        \[ \eta(\overline{\vec{k}},\overline{m},0) \vdash \eta(\overline{\vec{k}},\overline{m},0) \quad \bot \vdash u=\overline{m}
        \justifies
        \eta(\overline{\vec{k}},\overline{m},0), \neg \eta(\overline{\vec{k}},\overline{m},0) \vdash u=\overline{m} \]
        \justifies
        \overline{m}<u, \eta(\overline{\vec{k}},\overline{m},0),  \overline{m}<u \to \neg \eta(\overline{\vec{k}},\overline{m},0) \vdash u=\overline{m} \]
        \justifies
        \overline{m}<u, \eta(\overline{\vec{k}},\overline{m},0), \forall y(y<u \to \eta(\overline{\vec{k}},y,0)) \vdash u=\overline{m}
        \Justifies
        \overline{m}<u, \phi(\overline{\vec{k}},\overline{m}), \phi(\overline{\vec{k}},u) \vdash u=\overline{m}
        \endprooftree $$ \\
        Analogamente si ha $ u<\overline{m}, \phi(\overline{\vec{k}},\overline{m}), \phi(\overline{\vec{k}},u) \vdash u=\overline{m}$ e pi\`u semplicemente:
        $$ \prooftree
        u=\overline{m} \vdash u=\overline{m}
        \justifies
        u=\overline{m}, \phi(\overline{\vec{k}},\overline{m}), \phi(\overline{\vec{k}},u) \vdash u=\overline{m}
        \endprooftree $$ \\
        Ora, con un'$\lor$-riflessione e un'$\to$-formazione, otteniamo il risultato voluto. \\
  \end{enumerate}
\end{itemize}

Abbiamo cos\`i dimostrato che ogni funzione ricorsiva \`e rappresentabile in $S$ e che quindi il nostro sistema \`e sufficientemente forte.

\hspace{\stretch{1}} $\Box$\\

\begin{corol}
Ogni relazione ricorsiva \`e esprimibile in $S$.
\end{corol}

\textsc{\textbf{Dim:}} Sia $R(\vec{x})$ una relazione ricorsiva. Allora la sua funzione caratteristica $\chi_{R}$ \`e ricorsiva. Allora, per il teorema appena visto, $\chi_{R}$ \`e rappresentabile e, quindi, per quanto visto ad inizio capitolo, $R$ \`e esprimibile. \\

\hspace{\stretch{1}} $\Box$\\

\newcommand{\R}{\mathbb{R}}
\newcommand{\C}{\mathbb{C}}
\newcommand{\f}{\frac}
\newcommand{\E}{\mathbb{E}}
%%%%%%%%%%%%%%%%%%%%%%%%%%%%%%%%%%%%%%%%%%%%%%%%%%%%%%%%%%%%
% ARITMETTIZZAZIONE DELLA SINTASSI                         %
% Marta Pressato & Maria Chiara Bertolini                  %
%%%%%%%%%%%%%%%%%%%%%%%%%%%%%%%%%%%%%%%%%%%%%%%%%%%%%%%%%%%%

\newtheorem{p1}{Proposizione}[chapter]
\newtheorem{p2}{Proposizione}[chapter]
\newtheorem{p3}{Proposizione}[chapter]
\newtheorem{p4}{Proposizione}[chapter]
\newtheorem{p5}{Proposizione}[chapter]
\newtheorem{p6}{Proposizione}[chapter]

\chapter{Aritmetizzazione della Sintassi}
\label{chapter:aritmetizzazione}

\section{Introduzione}
Questo capitolo si pone in una fase intermedia nella trattazione dimostrativa del primo teorema di incompletezza di $G\ddot{o}del$. Abbiamo iniziato con la trattazione di un' importante teoria aritmetica assiomatica, quella di Heyting, e abbiamo visto che \`e possibile esprimere tramite formule del suo linguaggio tutto ci\`o che  \`e computabile da una macchina di Turing. Quindi il concetto di macchina \`e assimilabile ad un sistema formale.
In questa nostra ricerca di limiti della teoria in questione gioca un ruolo essenziale una intuizione geniale di $G\ddot{o}del$:
studiare la dimostrabilit\`a di HA attraverso HA stessa. Per capire pi\`u facilmente in cosa consiste quest'idea,
immaginiamo di lavorare con due robot: HA e MHA. HA \`e istruito con la logica ed \`e in grado di rappresentare tutte
le funzioni primitive ricorsive totali;
MHA serve per capire cosa HA sa dimostrare e quindi anche quanto ``HA sa di se stesso''.
Per istruire MHA dobbiamo tenere in considerazione che esso deve capire il linguaggio di HA; lo ``dotiamo'' quindi della logica intuizionistica e supponiamo che le istruzioni che gli possiamo dare siano del tipo: \emph{HA ha i seguenti assiomi e le seguenti regole;
se HA dimostra $\varphi$ allora $\varphi$ vale su $\mathbb {N}$; se $t=s$ allora HA dimostra che $t=s$.}\\
Gli oggetti invece di MHA sono termini, formule e derivazioni, ossia prove, tramite sequenti, di $\Gamma\vdash_{HA}\varphi$.
Accenniamo, non in modo formale, a due predicati che approfondiremo e utilizzeremo in seguito:
\begin{itemize}
\item $Der(\pi,\Gamma,\varphi)$ che significa $\pi$ \`e una derivazione di $\Gamma\vdash_{HA}\varphi$ cio\`e MHA, date $\Gamma$ e $\varphi$ sa decidere se $\pi$ \'e effettivamente una prova di $\Gamma\vdash_{HA}\varphi$;
\item $Thm(\varphi)=\exists\pi Der(\pi,\emptyset,\varphi)$ che significa $\varphi$ \`e un teorema di HA.
\end{itemize}
Utilizzando i ``nostri'' due robot, l'idea geniale di $G\ddot{o}del$ equivale a dimostrare che HA e MHA sono la stessa cosa. Infatti attraverso l'aritmetizzazione della sintassi, che ad ogni segno associa un numero n.G. (numero di $G\ddot{o}del$), ogni predicato di MHA diventa un predicato numerico di HA. Inizieremo codificando sequenze finite di numeri naturali, passando poi alla sintassi per concludere con le derivazioni: in tal modo saremo in grado di parlare di ci\`o che HA pu\`o dimostrare dentro HA stesso.





\section{Codifica di $G\ddot{o}del$}
Esistono diversi modi per la codifica di una sequenza di numeri naturali. Precisiamo subito che non \`e nostro interesse trovare quello pi\`u efficiente e quindi non ci interesseremo del grado di complessit\`a di calcolo. Utilizzeremo la codifica chiamata \emph{$G\ddot{o}del$ numbering} che si basa sul teorema di unicit\`a della scomposizione in fattori primi di un numero naturale.

Definiamo una funzione che associa un numero naturale, il numero di $G\ddot{o}del$, ad ogni sequenza finita $(n_1,\ldots,n_{k})$, con $n_1,\ldots,n_{k}\in \mathbb {N}$. Ad ogni sequenza deve corrispondere uno ed un solo numero naturale e quindi tale funzione deve necessariamente essere iniettiva; consideriamo allora:
\begin{displaymath}
\langle \bullet \rangle : \{\bigcup_{k\in\mathbb{N}} \mathbb{N}^k\} \rightarrow \mathbb{N}
\end{displaymath}

\begin{displaymath}
\langle n_1,\ldots,n_{k} \rangle = 2^{n_1+1} \cdot 3^{n_2+1}\cdots p_{k}^{n_{k}+1}
\end{displaymath}

\begin{displaymath}
\langle \rangle = 1\hspace{1cm}\text{(sequenza vuota).}
\end{displaymath}

Si noti che in generale il fatto di aggiungere $1$ ad ogni esponente $n_i$ ci permette di riconoscere la presenza di uno zero nella sequenza finita (si pensi all'analogia con la macchina di Turing) e garantisce l'iniettivit\`a della funzione sopra definita. Osserviamo ora che non tutti i numeri naturali $n$ sono il codice di una sequenza finita. Per esempio, $n = 3^5\cdot 11^4$ non lo \`e poich\'e, in tale fattorizzazione, non compaiono tutti i numeri primi minori di $11$. Per riconoscere quando $n$ \`e effettivamente codice di una sequenza finita introduciamo il predicato primitivo ricorsivo $Seq(n)$:
$$
Seq(n) := \forall p,q \leq n (Prime(p) \& Prime(q) \& p < q \& q|n \rightarrow p|n) \& n\ne 0,
$$
 dove  $Prime$ \`e il predicato (anch'esso primitivo ricorsivo) cos{\`{i}} definito:
$$
Prime(n) := \forall u,v \leq n[ (uv = x) \rightarrow (u=1 \vee v=1) ]
$$
 Il predicato $Seq(n)$ afferma che se un primo $q$ divide $n$ allora tutti i primi minori di $q$ dividono $n$.
 Una volta riconosciuta tale propriet\`a di $n$ \`e utile risalire agli elementi che compongono la sequenza originale. A tale scopo ci serviremo delle seguenti funzioni primitive ricorsive:
\begin{itemize}
\item[1)] {$exp(n,i)$, restituisce l'i-esimo elemento della sequenza originale, quindi $n_i$, ossia l'esponente meno $1$ dell'i-esimo primo nella fat\-to\-riz\-za\-zio\-ne di $n$: $$exp(n,i) := (\mu x \leq n){ [p_{i}^{x}|n \& \neg (p_i^{x+1}|n)]-1.}$$
   Facciamo un esempio per rendere pi\'u chiaro come agisce tale predicato. Prendiamo $n=72$ e applichiamo $exp$:
per come l'abbiamo definito, $epx(72,1)$ calcola il minimo $x$ tale che $2^x$ $(p_1=2)$ divide $72$ ma non $2^{x+1}$ e ad esso toglie uno. Poich\'e, per la nostra convenzione, il primo esponente nella fattorizzazione in primi \`e $n_1+1$, $exp(72,1)$ restituisce semplicemente $n_0=2$ (primo elemento della sequenza). Analogamente si ricava $exp(72,2)=1$.}

\item[2)] $len(n)$, calcola la lunghezza della sequenza originale, ossia il numero dei distinti fattori primi di $n$:
    $$len(n):= (\mu x \leq n)[\neg (p_x|n)] - 1;$$
    Prendiamo anche in questo caso, per fare un esempio, $72=2^3\cdot3^2$: $len$ calcola il minimo indice $x$ dei primi che non dividono $72$ e ci sottrae uno. Nel nostro caso risulta $p_3=5$ e quindi $x=2$.

\item[3)] $n\star m$, calcola il codice della sequenza ottenuta concatenando due sequenze di numeri codificate da $n$ e $m$: $$n\star m= n\cdot \prod_{i=1}^{len(m)} p_{len(n)+i}^{exp(m,i)+1}.$$


Tale predicato utilizza predicati di cui si sono gi\`a fatti esempi: non dovrebbe essere difficile quindi calcolare, per esempio, la concatenazione delle sequenze $n=\langle 2, 1\rangle = 2^3\cdot 3^2 $ e $m=\langle 3, 6\rangle = 2^4\cdot 3^7$. Essa risulta $2^3\cdot 3^2\cdot 5^4 \cdot 7^7$.

\item[4)] $last(n)$, restituisce l'ultimo valore della sequenza codificata da $n$:
$$last(n) := \left\{
\begin{array}{ll}
exp(n,len(n)), & \textrm{se $len(n)> 0$} \\
0, & \textrm{se $len(n)= 0$}
\end{array}
\right.$$

Consideriamo sempre, come esempio, $72=2^3\cdot 3^2$, allora si ha $exp(72, len(72)) = exp(72,2) = 1$ (chiaramente $len(72)>0$).
Nel caso in cui abbiamo $len(n)=0$ abbiamo solo il caso della sequenza vuota gi\`a vista.
\end{itemize}



Passiamo ora alla codifica della sintassi e per fare ci\`o iniziamo assegnando un codice a tutti i simboli del linguaggio come indicato nella tabella:

\vspace{0.5cm}

\begin{tabular}{|c|c|c|c|c|c|c|c|c|c|c|c|c|c|c|c|}
\hline
Simbolo & $\neg$ & $\&$ & $\vee$ & $\rightarrow$ & $\exists$ & $\forall$ & $($ & $)$ & $=$ & $0$ & $s$ & $+$ & $\cdot$ & $,$ & $x_i$ \\
\hline
Codice  & $1$ & $3$ & $5$ & $7$ & $9$ & $11$ & $13$ & $15$ & $17$ & $19$ & $21$ & $23$ & $25$ & $27$ & $2 i+29$\\
\hline
\end{tabular}
\vspace{0.5cm}

Con $\ulcorner \bullet \urcorner$ indichiamo il codice corrispondente a $\bullet$, per esempio \\ $\ulcorner \rightarrow \urcorner = 7$.

Definiamo alcuni predicati per variabili e funzioni:
\begin{eqnarray*}
Var(x) &:=& \exists i \leq x(x=2i+29), \text{indica che $x$ \`e una variabile;} \\
Const(x) &:=& x=  \ulcorner 0  \urcorner, \text{indica che $x$ \`e la costante;}\\
Fun_1(x) &:=& x=  \ulcorner s \urcorner, \text{indica che $x$ \`e la funzione unaria s;}\\
Fun_2(x) &:=& x=  \ulcorner  + \urcorner \vee  x= \ulcorner \cdot \urcorner, \text{indica che $x$ \`e la funzione binaria $+$ o $\cdot$.}
\end{eqnarray*}

 \section{Codifica dei termini}
Procediamo con la codifica della sintassi considerando ora i termini $Trm$; come gi\`a visto:
\begin{itemize}

\item[-]{$0 \in Trm$}
\item[-]{$x_i \in Trm$}
\item[-]{se $t_1, t_2 \in Trm$ allora $f(t_1,t_2) \in Trm$, dove $f\in \{+, \cdot, s\}$}.
\end{itemize}



Quindi, per un generico termine, definiamo:
\begin{displaymath}
 \ulcorner f(t_1,t_2) \urcorner :=  \langle  \ulcorner f \urcorner,  \ulcorner ( \urcorner,  \ulcorner t_1 \urcorner,  \ulcorner , \urcorner,  \ulcorner t_2 \urcorner ,  \ulcorner ) \urcorner\rangle .
\end{displaymath}

Esiste poi un predicato, $Term(n)$, che ci dice se $n$ \`e il codice di un termine; dalla definizione che segue notiamo che esso risulta primitivo ricorsivo:

\small{
\begin{displaymath}
 Term(n) := \left. \begin{array}{l}  Const(n) \vee Var(n) \vee\\
(Seq(n) \, \& \, len(n) = 4 \,\& \, Fun_1(exp(n,1)) \,\& \\
exp(n,2) = \ulcorner ( \urcorner \,\&\, Term(exp(n,3)) \,\&
exp(n,4) = \ulcorner ) \urcorner) \vee \\
(Seq(n) \, \& \, len (n)=6 \, \& \, Fun_2(exp(n,1)) \,\&\, exp(n,2)= \ulcorner ( \urcorner  \\
\,\&\,Term(exp(n,3)) \,\&\, exp(n,4) = \ulcorner , \urcorner \,\&\,Term(exp(n,5)) \,\&\, exp(n,6) = \ulcorner ) \urcorner
).
\end{array} \right.
\end{displaymath}
}

Cerchiamo di dare l'idea di come agisce tale predicato. Abbiamo visto che un termine \`e o una costante, o una variabile o una funzione $f\in \{+, \cdot, *\}$ applicata ad un termine: $Const(n)$, $Var(n)$ e $(\dots)$ riconoscono rispettivamente questi tre casi. Analizziamo quello tra parentesi, ossia $(\dots)$; in sostanza  deve succedere che
$n$ \`e una sequenza ``valida'' ($Seq(n)$); la sua lunghezza \`e pari a 4 ($len(n)=4$); il primo elemento della sequenza ($exp(n,1)$) \`e la funzione unaria; il secondo una parentesi aperta ($exp(n,2) = \ulcorner ( \urcorner$); il terzo un termine ($Term(exp(n,3))$) e il quarto una parentesi chiusa ($exp(n,4) = \ulcorner ) \urcorner)$).
Altrimenti pu\'o anche essere (parte finale): la sequenza \`e valida, la lunghezza \`e pari a 6, il primo elemento \`e la funzione binaria, il secondo una parentesi, il terzo e il quinto due termini, il quarto la virgola e il sesto una parentesi.
 

\section{Codifica delle formule}

In modo analogo operiamo con le formule $Frm$, ricordando che:
\begin{itemize}
\item[-]{se $t_1,\dots, t_k \in Trm$ e $R$ \`e una relazione a $k$ argomenti, allora $R(t_1,\dots, t_k) \in Frm$ ($R(t_1,\dots, t_k)$ si dice formula atomica)};
\item[-]{Frm \`e chiuso per $\&$, $\vee$, $\rightarrow$, $\exists$, $\forall$}.
\end{itemize}

Siano $t_i$ termini e $\varphi, \psi$ formule, allora $\ulcorner \bullet \urcorner$ opera nel seguente modo:

\begin{eqnarray*}
\ulcorner (t_1=t_2) \urcorner                     &:=&
\langle  \ulcorner ( \urcorner,  \ulcorner t_1 \urcorner,  \ulcorner = \urcorner,  \ulcorner t_2 \urcorner ,  \ulcorner ) \urcorner\rangle \\
\ulcorner (\varphi \& \psi) \urcorner         &:=&  \langle  \ulcorner ( \urcorner,  \ulcorner \varphi \urcorner,
\ulcorner \& \urcorner,  \ulcorner \psi \urcorner ,  \ulcorner ) \urcorner\rangle \\
\ulcorner (\varphi \vee \psi) \urcorner           &:=&  \langle  \ulcorner ( \urcorner,  \ulcorner \varphi \urcorner,
\ulcorner \vee \urcorner,  \ulcorner \psi \urcorner ,  \ulcorner ) \urcorner\rangle\\
\ulcorner (\varphi \rightarrow \psi) \urcorner  &:=&  \langle  \ulcorner ( \urcorner,  \ulcorner \varphi \urcorner,
\ulcorner \rightarrow \urcorner,  \ulcorner \psi \urcorner ,  \ulcorner ) \urcorner\rangle\\
\ulcorner (\forall\,x_i\,\varphi) \urcorner    &:=&  \langle  \ulcorner ( \urcorner,  \ulcorner \forall \urcorner,
\ulcorner x_i \urcorner,  \ulcorner \varphi \urcorner ,  \ulcorner ) \urcorner\rangle\\
\ulcorner (\exists\, x_i\, \varphi) \urcorner  &:=&  \langle  \ulcorner ( \urcorner,  \ulcorner \exists \urcorner,  \ulcorner x_i \urcorner,  \ulcorner \varphi \urcorner ,  \ulcorner ) \urcorner\rangle\\
\end{eqnarray*}

Quindi, per esempio, $\ulcorner( x_1 \vee x_2 )\urcorner = \langle 13, 2+29, 5, 4+29, 15 \rangle =
     2^{13+1}\cdot 3^{31+1} \cdot 5^{5+1}\cdot 7^{33+1} \cdot 11^{15+1}.$

Come per i termini, esiste un predicato primitivo ricorsivo, $Form(n)$, che ci dice se $n$ \`e il codice di una formula. Esso \`e definito come segue:


{\small{

\begin{displaymath}
Form(n) := \left. \begin{array}{l} Seq(n) \, \& \, len(n)= 5 \,\& \, exp(n,1) = \ulcorner ( \urcorner  \, \& \, exp(n,5)=\ulcorner ) \urcorner \,\&\,\\
((Term(exp(n,2)) \, \& \, Term(exp(n,4)) \, \& \, exp(n,3) = \ulcorner = \urcorner) \vee\\
(Form(exp(n,2)) \,\&\, Form(exp(n,4)) \,\&\\
(exp(n,3) = \ulcorner \& \urcorner \vee exp(n,3) = \ulcorner \vee \urcorner \vee exp(n,3)= \ulcorner \rightarrow \urcorner)) \vee\\
((exp(n,2) = \ulcorner \forall \urcorner \vee exp(n,2) =
\ulcorner \exists \urcorner) \,\& \\
Var(exp(n,3)) \,\&\, Form(exp(n,4)))) .
\end{array} \right.
\end{displaymath}}}

Per la coprensione del predicato $Form(n)$ si prenda spunto dalla spiegazione del predicato $Term(n)$.

\section{Codifica delle derivazioni}

A questo punto restano da trattare le derivazioni. Per fare ci\`o abbiamo bisogno di definire  l'operatore di sostituzione e ulteriori predicati.
Data una formula $\varphi$, un termine $t$ ed una variabile $x_1$ usiamo $\varphi [t/x_1]$ per denotare  $\varphi$ in cui tutte le occorrenze di $x_1$ vengono sostituite con $t$.

Gli operatori di sostituzione per  termini e  formule possono essere definiti  come segue:

\textbf{$r$ \`e un termine}
\begin{itemize}

\item se $r$ \`e una costante $c$ allora:
$$
c[t/x_1] := c
$$
\item se $r$ \`e una variabile $x_2$ allora:
\begin{displaymath}
 x_2[t/x_1]:= \left \{ \begin{array}{ll}
x_2, & \textrm{se $x_1 \ne x_2$}\\
t, & \textrm{se $x_1 = x_2$}
\end{array}\right.
\end{displaymath}
\item se $r$ \`e del tipo $f(t_1,\ldots, t_n)$, con $f \in \{s,+,\cdot \}$ e $t_i$ termini, allora:
$$
f(t_1,\ldots, t_n)[t/x_1] := f(t_1[t/x_1],\ldots, t_n[t/x_1]).
$$
\end{itemize}

\textbf{$\varphi$ \`e una formula}

\begin{itemize}
 \item se $\varphi$ \`e una formula atomica e $t_1,t_2$ sono termini, allora:
$$
(t_1=t_2)[t/x_1] := (t_1[t/x_1]=t_2[t/x_1])
$$

\item se $\varphi$ \`e costruita con le formule $\psi$ e $\chi$ utilizzando connettivi (che denoteremo con $\circ$), allora:
$$
(\psi \circ \chi)[t/x_1] := (\psi[t/x_1]\circ \chi[t/x_1])
$$

\item se $\varphi$ \`e costruita con la formula $\psi$ utilizzando i quantificatori, allora:

\begin{eqnarray*}
 \forall\, x_2\, \psi [t/x_1]  &:=&    \left \{ \begin{array}{ll}
\forall\, x_2\, \psi [t/x_1], & \textrm{se $x_1 \ne x_2$}\\
\forall\, x_2\, \psi, & \textrm{se $x_1 = x_2$}
\end{array}\right.\\
\exists\, x_2\, \psi [t/x_1]  &:=& \left \{ \begin{array}{ll}
\exists\, x_2\, \psi [t/x_1], & \textrm{se $x_1 \ne x_2$}\\
\exists\, x_2\, \psi, & \textrm{se $x_1 = x_2$}.
\end{array}\right.
\end{eqnarray*}

\end{itemize}

In quest'ultimo caso, lavorando cio\`e con i quantificatori, \`e necessario prestare attenzione alle variabili che ci interessano nella sostituzione, in particolare se queste compaiono libere o quantificate.

Facciamo un ripasso sul significato di variabile libera in una formula:
\begin{itemize}
 \item $\varphi$ formula atomica: $x$ occorre libera in $\varphi$ se $x$ compare in $\varphi$;
\item $\varphi=\psi\circ \chi$: $x$ occorre libera in $\varphi$ se occorre libera in $\psi$ o $\chi$;
\item $\varphi=\forall \,x_2\, \psi$ o $\varphi=\exists \,x_2\, \psi$: $x$ occorre libera in $\varphi$ se occorre libera in $\psi$ e $x \ne x_2$. (In questo caso $x_2$ non occorre libera, ma quantificata.)
 \end{itemize}

Consideriamo, per capire meglio, $\forall x_1\exists x_2 (x_1=2x_2)$: in tale espressione $x_1$ appare quantificata e risulta chiaramente privo di senso volerla sostituire, per esempio, con $x_1=3$. Se tale variabile fosse libera avremmo $\exists x_2 (x_1=2x_2)$ e in questo caso, invece, la precedente sostituzione sarebbe sensata.

Esiste un altro caso in cui la sostituzione precedentemente definita risulta fallimentare. Consideriamo, per esempio, la formula $\exists x_1(x_1=x_2)$.
Supponiamo di voler sostituire ad $x_2$ il termine $s(x_1)$, allora otteniamo, per quanto appena visto, $\exists x_1(x_1=s(x_1))$.
Chiaramente la prima espressione ha significato in HA, mentre la seconda risulta assurda.
Ci\`o \`e dovuto al fatto che si \`e effettuata una sostituzione con il termine $s(x_1)$ contenente la variabile $x_1$ che risultava quantificata.

Per ovviare a questo problema possiamo procedere in due modi. Il primo consiste nell'effettuare una modifica nella definizione di sostituzione: l'idea \`e quella di cambiare il nome delle variabili quantificate in modo che esse non compaiano anche nei termini che si vogliono sostituire. La regola di sostituzione diventa quindi:

\begin{eqnarray*}
 \forall\, x_2\, \psi [t/x_1]  &:=&    \left \{ \begin{array}{ll}
\forall\, x_3\, \psi [x_3/x_2] [t/x_1], & \textrm{se $x_1 \ne x_2$ e $x_3$ non compare in $t$}\\
\forall\, x_2\, \psi, & \textrm{se $x_1 = x_2$}
\end{array}\right.\\
\exists\, x_2\, \psi [t/x_1]  &:=& \left \{ \begin{array}{ll}
\exists\, x_3\, \psi [x_3/x_2] [t/x_1], & \textrm{se $x_1 \ne x_2$ e $x_3$ non compare in $t$}\\
\exists\, x_2\, \psi, & \textrm{se $x_1 = x_2$}.
\end{array}\right.
\end{eqnarray*}

Il secondo, che utilizzeremo, consiste nell'introduzione di due nuovi pre\-di\-ca\-ti che riconoscono quando una variabile compare libera. Ricordiamo che dato un termine $t$, una variabile $x_1$ e una formula $\varphi$ , $x_1$ \`e libera per $t$ in $\varphi$ se:
\begin{itemize}
 \item $\varphi$ \`e formula atomica;
\item $\varphi=\psi\circ \chi$ e $x_1$ \`e libera per $t$ in $\psi$ e $\chi$;
\item $\varphi=\forall \,x_2\, \psi$ o $\varphi=\exists \,x_2\, \psi$ e $x_2$ non \`e libera per $t$ mentre $x_1$ \`e libera per $t$ in $\psi$.
\end{itemize}

Il predicato $Fv(l, m)$ riconosce se la variabile (codificata da) $l$ appare libera per il termine (codificato da) $m$, mentre $FreeFor(l,m,n)$ riconosce se la variabile (codificata da) $m$ appare libera per il termine (codificato da) $l$ nella formula (codificata da) $n$.

{\tiny
\begin{displaymath}
Fv(l,m) := \left. \begin{array}{l}(Var(l) \,\&\, Term(m) \,\&\, \neg Const(m) \,\&\, \\
((Var(l)\rightarrow l=m) \vee (Fun_1(exp(m,1))\rightarrow Fv(l,exp(m,3))) \vee \\
(Fun_2(exp(m,1))\rightarrow (Fv(l,exp(m,3))\vee Fv(l,exp(m,5))))
)) 
\end{array} \right.
\end{displaymath}}

{\tiny
\begin{displaymath}
FreeFor(l,m,n):= \left.\begin{array}{l} Term(l) \,\&\, Var(m) \,\&\, Form(n) \,\&\, \\
(((exp(n,3) = \ulcorner = \urcorner)\rightarrow (Fv(m,exp(2,n))\vee Fv(m,exp(4,n))))\vee \\ 
((exp(n,3)= \ulcorner \& \urcorner \ \vee exp(n,3)= \ulcorner \vee \urcorner)\rightarrow \\
(FreeFor(l,m,exp(n,2)) \,\&\, FreeFor(l,m,exp(n,4)))) \vee \\
((exp(n,2) = \ulcorner \forall \urcorner \vee exp(n,2) = \ulcorner \exists \urcorner)\rightarrow \\
(\neg Fv(exp(n,3),l) \,\&\, exp(n,3)\ne m \,\&\, FreeFor(l,m,exp(n,4)))))
\end{array}\right.
\end{displaymath}}

Definiamo inoltre un predicato $Bound(l,n)$ che riconosce se la variabile (codificata da) $l$ \`e quantificata nella formula (codificata da) $n$.
{\tiny
\begin{displaymath}
 Bound(l,n):= \left.\begin{array}{l} Var(l) \,\&\, Form(n)\,\&\, \\
 (((exp(n,3) = \ulcorner \& \urcorner \ \vee exp(n,3)= \ulcorner \vee \urcorner) \rightarrow \\
(Bound(l,exp(n,2))\vee Bound(l,exp(n,4)))) \vee \\
((exp(n,2)= \ulcorner \forall \urcorner \vee \exp(n,2) = \ulcorner \exists \urcorner) \rightarrow l= exp(n,3)))
\end{array}\right.
\end{displaymath}}



Anche in questo caso notiamo che i predicati sono primitivi ricorsivi.

Fatta tale precisazione, possiamo ora codificare anche l'operatore di sostituzione che abbiamo definito prima, in modo che:
$$
Sub(\ulcorner t \urcorner, \ulcorner x_1 \urcorner, \ulcorner \varphi \urcorner) = \ulcorner \varphi[t/x_1] \urcorner,
$$
ossia $Sub$ restituisce il n.G. della formula $\varphi$ in cui si \`e sostituito $t$ a $x_1.$

{\tiny
\begin{displaymath}
Sub(l,m,n):= \left\{ { \begin{array}{l}
n, \hspace{1.8cm} \text{se $Const(n)$}\\
n, \hspace{1.8cm}\text {se $Var(n) \,\&\, n\ne m$} \\
l, \hspace{1.9cm}\text {se $Var(n) \,\&\, n= m$} \\
\langle exp(n,1), \ulcorner ( \urcorner, Sub(l,m,exp(n,3)), \ulcorner ) \urcorner \rangle,\\
\hspace{2.1cm}  \text{se $Term(n) \,\&\, Fun_1(exp(n,1))$}\\
\langle exp(n,1), \ulcorner ( \urcorner, Sub(l,m,exp(n,3)), Sub(l,m,exp(n,5)),\ulcorner ) \urcorner \rangle,\\
\hspace{2.1cm}  \text{se $Term(n) \,\&\, Fun_2(exp(n,1))$}\\
\langle \ulcorner ( \urcorner, Sub(l,m,exp(n,2)), exp(n,3), Sub(l,m,exp(n,4)),\ulcorner ) \urcorner \rangle,\\
\hspace{2.1cm}  \text{se $Form(n) \,\&\, FreeFor(l,m,n) \,\&\, exp(n,2)\ne \ulcorner \forall\urcorner \,\&\,$}\\
\hspace{2.1cm} \text{$exp(n,2)\ne \ulcorner \exists \urcorner$}\\
\langle \ulcorner ( \urcorner, exp(n,2), exp(n,3), Sub(l,m,exp(n,4)),\ulcorner ) \urcorner \rangle,\\
\hspace{2.1cm}  \text{se $Form(n) \,\&\, FreeFor(l,m,n) \,\&\, exp(n,2) = \ulcorner \forall\urcorner \,\vee\,$}\\
\hspace{2.1cm} 	\text{$exp(n,2) = \ulcorner \exists \urcorner$}\\
n, \hspace{1.8cm} \text{altrimenti}\\
\end{array}}\right.
\end{displaymath}}



A questo punto, possiamo trattare le derivazioni: inizieremo associando ad ogni regola logica un codice e arriveremo a capire dove e perch\'e \`e necessario introdurre i predicati $Der(\pi,\Gamma,\varphi)$, primitivo ricorsivo, e $Thm(\varphi)$, ricorsivamente enumerabile.
Ricordiamo, prima di iniziare con la codifica della regole, che le prove sono effettuate utilizzando sequenti e quindi \`e necessario esplicitare come essi si devono trattare. Definiamo a tal proposito:

\begin{displaymath}
\ulcorner \Gamma\vdash_{HA} \varphi \urcorner :=  \langle \langle \ulcorner \psi_1, \urcorner,  \ldots ,  \ulcorner \psi_m \urcorner \rangle ,  \ulcorner \varphi \urcorner \rangle ,
\end{displaymath}
dove $\Gamma = \psi_1,\ldots,\psi_m$ e $\varphi$ e $\psi_i$ sono formule.

Inoltre bisogna tenere presente che il nostro sistema $HA$ ha alcuni assiomi e la regola di induzione per svolgere le deduzioni. Tramite sequenti essa diviene:
\begin{displaymath}
\frac{\Gamma, \varphi(x_1) \vdash \varphi(s(x_1))}{\Gamma, \varphi(0)\vdash \forall x_2 \varphi(x_2)}.
\end{displaymath}
Per quanto riguarda gli assiomi invece, essi vengono chiamati \emph{sequenti iniziali} e il loro codice \`e:

\begin{displaymath}
[ \Gamma\vdash \varphi ] :=  \langle 0,  \ulcorner \Gamma\vdash \varphi \urcorner  \rangle
\end{displaymath}

Fatte queste precisazioni possiamo ora definire la codifica di tutte le regole logiche. Osserviamo che ogni connettivo ha due regole che lo caratterizzano in cui esso compare, o nelle ipotesi o nella tesi. Prendiamo per esempio $\rightarrow$: si ha rispettivamente $\rightarrow$ -left e $\rightarrow$ -right. Nella codifica tale distinzione deve emergere ed \`e per questo motivo che tutte le regole left verranno etichettate tramite un $1$ mentre tutte le regole right tramite uno $0$.

\vspace{0.5cm}
\textbf{\&-right}

$$
\left [
\begin{array}{cc}
\Omega_1 & \Omega_{2} \\
\vdots & \vdots\\
\Gamma \vdash \varphi & \Gamma \vdash \psi\\
\hline
\multicolumn{2}{c}{\Gamma \vdash \varphi \& \psi}\\
\end{array}
\right ]
:= \langle \langle 0,\ulcorner \& \urcorner \rangle ,
\left [
\begin{array}{c}
\Omega_1\\
\vdots\\
\Gamma \vdash \varphi\\
\end{array}
\right ],
\left [
\begin{array}{c}
\Omega_2\\
\vdots\\
\Gamma \vdash \psi\\
\end{array}
\right],
\ulcorner \Gamma \vdash \varphi \& \psi \urcorner \rangle
$$

\vspace{0,5cm}

\textbf{\&-left}

$$
\left [
\begin{array}{c}
\Omega\\
\vdots\\
\Gamma, \varphi \vdash \Delta\\
\hline
\Gamma, \varphi \& \psi \vdash \Delta\\
\end{array}
\right ]
:= \langle \langle 1,\ulcorner \& \urcorner \rangle ,
\left [
\begin{array}{c}
\Omega\\
\vdots\\
\Gamma, \varphi \vdash \Delta\\
\end{array}
\right ],
\ulcorner \Gamma, \varphi \& \psi \vdash \Delta \urcorner \rangle
$$

\vspace{0.5cm}

\textbf{$\vee$ -right}

$$
\left [
\begin{array}{c}
\Omega\\
\vdots\\
\Gamma \vdash \varphi\\
\hline
\Gamma \vdash \varphi \vee \psi\\
\end{array}
\right ]
:= \langle \langle 0,\ulcorner \vee \urcorner \rangle ,
\left [
\begin{array}{c}
\Omega\\
\vdots\\
\Gamma \vdash \varphi\\
\end{array}
\right ],
\ulcorner \Gamma \vdash \varphi \vee \psi \urcorner \rangle
$$

\vspace{0.5cm}

\textbf{$\vee$ -left}

{\tiny
$$
\left [
\begin{array}{cc}
\Omega_1 & \Omega_{2} \\
\vdots & \vdots\\
\Gamma, \varphi \vdash \Delta & \Gamma, \psi \vdash \Delta\\
\hline
\multicolumn{2}{c}{\Gamma, \varphi \vee \psi \vdash \Delta}\\
\end{array}
\right ]
:= \langle \langle 1,\ulcorner \vee \urcorner \rangle ,
\left [
\begin{array}{c}
\Omega_1\\
\vdots\\
\Gamma,\varphi \vdash \Delta\\
\end{array}
\right ],
\left [
\begin{array}{c}
\Omega_2\\
\vdots\\
\Gamma, \psi \vdash \Delta\\
\end{array}
\right],
\ulcorner \Gamma, \varphi \vee \psi \vdash \Delta \urcorner \rangle
$$}

\vspace{0.5cm}

\textbf{$\rightarrow$ -right}

$$
\left [
\begin{array}{c}
\Omega\\
\vdots\\
\Gamma, \varphi \vdash \psi\\
\hline
\Gamma \vdash \varphi \rightarrow \psi\\
\end{array}
\right ]
:= \langle \langle 0,\ulcorner \rightarrow \urcorner \rangle ,
\left [
\begin{array}{c}
\Omega\\
\vdots\\
\Gamma, \varphi \vdash \psi\\
\end{array}
\right ],
\ulcorner \Gamma \vdash \varphi \rightarrow \psi \urcorner \rangle
$$

\vspace{0.5cm}

\textbf{$\rightarrow$ -left}

{\tiny
$$
\left [
\begin{array}{cc}
\Omega_1 & \Omega_{2} \\
\vdots & \vdots\\
\Gamma_1, \vdash \varphi & \Gamma_2, \psi \vdash \Delta\\
\hline
\multicolumn{2}{c}{\Gamma_1, \Gamma_2, \varphi \rightarrow \psi \vdash \Delta}\\
\end{array}
\right ]
:= \langle \langle 1,\ulcorner \rightarrow \urcorner \rangle ,
\left [
\begin{array}{c}
\Omega_1\\
\vdots\\
\Gamma_1 \vdash \varphi\\
\end{array}
\right ],
\left [
\begin{array}{c}
\Omega_2\\
\vdots\\
\Gamma_2, \psi \vdash \Delta\\
\end{array}
\right],
\ulcorner \Gamma_1, \Gamma_2,  \varphi \rightarrow \psi \vdash \Delta \urcorner \rangle
$$}

\vspace{0.5cm}

\textbf{$\forall$ -right} ($x_2$ non deve essere libera in $\Gamma$)

$$
\left [
\begin{array}{c}
\Omega\\
\vdots\\
\Gamma \vdash \varphi(x_2)\\
\hline
\Gamma  \vdash \forall x_1\,\varphi(x_1)\\
\end{array}
\right ]
:= \langle \langle 0,\ulcorner \forall \urcorner \rangle ,
\left [
\begin{array}{c}
\Omega\\
\vdots\\
\Gamma \vdash \varphi(x_2)\\
\end{array}
\right ],
\ulcorner \Gamma \vdash \forall x_1\,\varphi(x_1) \urcorner \rangle
$$

\vspace{0.5cm}

\textbf{$\forall$ -left}

$$
\left [
\begin{array}{c}
\Omega\\
\vdots\\
\Gamma, \varphi(x_2) \vdash \Delta\\
\hline
\Gamma, \forall x_1\,\varphi(x_1) \vdash \Delta\\
\end{array}
\right ]
:= \langle \langle 1,\ulcorner \forall \urcorner \rangle ,
\left [
\begin{array}{c}
\Omega\\
\vdots\\
\Gamma, \varphi(x_2) \vdash \Delta\\
\end{array}
\right ],
\ulcorner \Gamma, \forall x_1\,\varphi(x_1) \vdash \Delta \urcorner \rangle
$$

\vspace{0.5cm}

\textbf{$\exists$ -right}

$$
\left [
\begin{array}{c}
\Omega\\
\vdots\\
\Gamma \vdash \varphi(x_2)\\
\hline
\Gamma  \vdash \exists x_1\,\varphi(x_1)\\
\end{array}
\right ]
:= \langle \langle 0,\ulcorner \exists \urcorner \rangle ,
\left [
\begin{array}{c}
\Omega\\
\vdots\\
\Gamma \vdash \varphi(x_2)\\
\end{array}
\right ],
\ulcorner \Gamma \vdash \exists x_1\,\varphi(x_1) \urcorner \rangle
$$

\vspace{0.5cm}

\textbf{$\exists$ -left} ($x_2$ non deve essere libera in $\Gamma$)



$$
\left [
\begin{array}{c}
\Omega\\
\vdots\\
\Gamma, \varphi(x_2) \vdash \Delta\\
\hline
\Gamma, \exists x_1\,\varphi(x_1) \vdash \Delta\\
\end{array}
\right ]
:= \langle \langle 1,\ulcorner \exists \urcorner \rangle ,
\left [
\begin{array}{c}
\Omega\\
\vdots\\
\Gamma, \varphi(x_2) \vdash \Delta\\
\end{array}
\right ],
\ulcorner \Gamma, \exists x_1\,\varphi(x_1) \vdash \Delta \urcorner \rangle
$$




Ci manca a questo punto la codifica delle regole strutturali, che caratterizzano le derivazioni tramite sequenti. Anche in questo caso la codifica di ogni regola inizier\`a con una coppia di numeri che la render\`a subito riconoscibile. Ricordiamo che tali regole strutturali sono l'identit\`a, la regola di scambio, l'indebolimento, la contrazione e il taglio e la loro codifica risulta:


\vspace{0.5cm}
\textbf{identit\`a}
$$
[ \varphi \vdash \varphi ] := \langle 0, \ulcorner \varphi \vdash \varphi \urcorner \rangle
$$

\vspace{0.5cm}

\textbf{scambio}

$$
\left [
\begin{array}{c}
\Omega\\
\vdots\\
\varphi, \psi \vdash \Delta\\
\hline
\psi, \varphi \vdash \Delta\\
\end{array}
\right ]
:= \langle \langle 0, 0\rangle ,
\left [
\begin{array}{c}
\Omega\\
\vdots\\
\varphi, \psi \vdash \Delta\\
\end{array}
\right ],
\ulcorner \psi, \varphi \vdash \Delta \urcorner \rangle
$$

\vspace{0.5cm}

\textbf{indebolimento}

$$
\left [
\begin{array}{c}
\Omega\\
\vdots\\
\Gamma \vdash \Delta\\
\hline
\Gamma, \Sigma \vdash \Delta\\
\end{array}
\right ]
:= \langle \langle 0, 1 \rangle ,
\left [
\begin{array}{c}
\Omega\\
\vdots\\
\Gamma \vdash \Delta\\
\end{array}
\right ],
\ulcorner \Gamma, \Sigma \vdash \Delta \urcorner \rangle
$$

\vspace{0.5cm}

\textbf{contrazione}

$$
\left [
\begin{array}{c}
\Omega\\
\vdots\\
\Gamma, \varphi, \varphi \vdash \Delta\\
\hline
\Gamma, \varphi \vdash \Delta\\
\end{array}
\right ]
:= \langle \langle 1, 0 \rangle ,
\left [
\begin{array}{c}
\Omega\\
\vdots\\
\Gamma, \varphi, \varphi \vdash \Delta\\
\end{array}
\right ],
\ulcorner \Gamma, \varphi \vdash \Delta \urcorner \rangle
$$

\vspace{0,5cm}

\textbf{taglio}

$$
\left [
\begin{array}{cc}
\Omega_1 & \Omega_2 \\
\vdots & \vdots\\
\Gamma_1 \vdash \varphi & \Gamma_2, \varphi \vdash \Delta\\
\hline
\multicolumn{2}{c}{\Gamma_1, \Gamma_2 \vdash \Delta}\\
\end{array}
\right ]
:= \langle \langle 1, 1 \rangle ,
\left [
\begin{array}{c}
\Omega_1\\
\vdots\\
\Gamma_1 \vdash \varphi\\
\end{array}
\right ],
\left [
\begin{array}{c}
\Omega_2\\
\vdots\\
\Gamma_2, \varphi \vdash \Delta\\
\end{array}
\right],
\ulcorner \Gamma_1, \Gamma_2 \vdash \Delta \urcorner \rangle
$$



Per concludere, la codifica della regola di induzione risulta essere:


\vspace{0.5cm}
\textbf{induzione}
{\tiny
$$
\left [
\begin{array}{c}
\Omega\\
\vdots\\
\Gamma, \varphi(x_1) \vdash \varphi(s(x_1))\\
\hline
\Gamma, \varphi(0) \vdash \forall x_2\,\varphi(x_2)\\
\end{array}
\right ]
:= \langle \langle 1, 2 \rangle ,
\left [
\begin{array}{c}
\Omega\\
\vdots\\
\Gamma, \varphi(x_1) \vdash \varphi(s(x_1))\\
\end{array}
\right ],
\ulcorner \Gamma, \varphi(0) \vdash \forall x_2\,\varphi(x_2) \urcorner \rangle
$$}

\vspace{0.5cm}

A questo punto abbiamo tutti gli elementi per capire meglio il predicato  $Der(\pi,\Gamma,\varphi)$: tramite la codifica appena esplicitata gli argomenti di $Der$ diventano valori effettivi. Possiamo ottenere, infatti, $Der(l, m, n)$, in cui $l$ \`e il codice della derivazione $\pi$ della formula $\varphi$ codificata tramite $n$ a partire dalle ipotesi $\Gamma$ codificate invece tramite $m$.
Tutto ci\`o significa che $HA$, tramite $Der(l,m,n)$, \`e in grado di decidere se $\pi$ \`e effettivamente una prova in $HA$ stesso.

Tralasciando la definizione formale di $Der(l,m,n)$ ci limitiamo a dare solo un'idea intuitiva del perch\'e essa sia primitiva ricorsiva. $Der$ si costruisce utilizzando esclusivamente tutti i predicati fin qui definiti che, come ab\-bia\-mo visto, sono primitivi ricorsivi.  $Der$ \`e quindi primitiva ricorsiva poich\'e composizione di predicati con tale propriet\`a; risulta inoltre, per come \`e stata definita, decidibile.
Precisiamo che il predicato $Der$ \`e definibile in modo modulare rispetto agli assiomi, infatti qualunque sia il set di assiomi, $Der$ risulta sempre primitivo ricorsivo.

Consideriamo ora il caso particolare in cui  $\Gamma = \emptyset$ e quindi $Der(l,m,n)=Der(l,\langle \rangle,n)$.
Utilizzando la precedente, introduciamo il predicato $Thm(n)$ che indica che $n$ \`e (il codice di) un teorema in $HA$:
$$
Thm(n) := \exists \, l \, Der(l,\langle \rangle,n) .
$$

Ricordando infine che, tramite il predicato di Kleene, quantificando esistenzialmente un predicato decidibile, si ottiene un predicato ricorsivamente enumerabile, $Thm(n)$ risulta chiaramente ricorsivamente enumerabile.

Quindi abbiamo dimostrato che HA e MHA sono la stessa cosa, avendo trasformato ogni predicato di MHA in un predicato numerico di HA.

%\include{07_logica_dimostrabilita}

\chapter{Primo Teorema di Incompletezza}
	\hyphenation{lo-gi-ca a-rit-me-ti-ca rap-pre-sen-ta-bi-le ri-cor-si-va-men-te in-de-ci-di-bi-le e-sis-te in-tro-du-cia-mo e-qui-va-len-za ab-bia-mo}

\section{Risultati precedenti su decidibilit\`a effettiva
e decidibilit\`a ricorsiva}

	Pu\`o essere utile, per avere una panoramica concettuale
	completa, anteporre alla dimostrazione del primo teorema di
	incompletezza, risultato che nel framework del corso acquisisce carattere
	conclusivo, un breve riassunto dei risultati fin qui ottenuti.
	
	Una funzione $f$ dai numeri naturali ai numeri naturali si dice
	\textit{\emph{effettivamente} \emph{calcolabile}}
	se esiste una procedura
	effettiva che permette di calcolare $f(x_1,\,\dots,x_n)$ per ogni
	$n$-upla $(x_1,\,\dots,x_n)$. 	
	Ogni funzione ricorsiva totale \`e effettivamente calcolabile. Il viceversa,
	cio\`e che ogni funzione effettivamente calcolabile
	\`e ricorsiva totale, \`e noto
	come \textit{\emph{tesi di Church}}.
	
	Un insieme di numeri naturali \`e detto \emph{\textit{effettivamente
	decidibile}} se esiste una procedura effettiva che, per ogni
	numero, se applicata ad esso, in una quantit\`a
	finita di tempo, risponde correttamente
	alla domanda se tale numero appartenga \textit{o meno} all'insieme.
	Questa propriet\`a \`e equivalente alla propriet\`a per la funzione caratteristica
	dell'insieme di essere effettivamente calcolabile.
	Un insieme \`e detto \emph{\textit{ricorsivo}} se la sua funzione
	caratteristica \`e ricorsiva totale.
	Un insieme ricorsivo \`e effettivamente calcolabile, il viceversa segue dalla
	tesi di Church.
	
	Una relazione \`e \textit{\emph{effettivamente decidibile}} se l'insieme
	delle $n$-uple che la soddisfano lo \`e. \`E \textit{\emph{ricorsiva}}
	se l'insieme delle $n$-uple che la soddisfano lo \`e. Una relazione
	ricorsiva \`e effettivamente decidibile, il viceversa segue dalla tesi di
	Church.
	
	Una funzione $f$ da un sottoinsieme dei 
	numeri naturali ai numeri naturali si dice
	\textit{\emph{effettivamente semicalcolabile}}
	se esiste una procedura effettiva che permette di calcolare
	$f(x_1,\,\dots,x_n)$ per ogni $n$-upla $(x_1,\,\dots,x_n)$ su cui
	$f$ \`e definita.
	Ogni funzione ricorsiva parziale \`e effettivamente semicalcolabile.
	Il viceversa, 	cio\`e che ogni funzione effettivamente semicalcolabile
	\`e ricorsiva parziale, segue dalla tesi di Church.
	
	Un insieme \`e detto \textit{\emph{effettivamente semidecidibile}}
	se esiste una
	procedura effettiva che, per ogni numero, se applicata ad esso, se il numero
	appratiene all'insieme dar\`a risposta affermativa in un intervallo finito di
	tempo, se non vi appartiene, non dar\`a risposta.
	Questa propriet\`a \`e equivalente alla fatto che esiste una funzione
	semicalcolabile di valore costante $1$ il cui dominio \`e l'insieme stesso.
	Questa funzione \`e detta \textit{\emph{semicaratteristica}}.
	Un insieme \`e detto \emph{\textit{ricorsivamente enumerabile}} se
	la sua funzione semicaratteristica \`e ricorsiva.
	Un insieme ricorsivamente enumerabile \`e effettivamente semidecidibile,
	il viceversa segue dalla tesi di Church.
	
	Una relazione \`e \textit{\emph{effettivamente semidecidibile}}
	se l'insieme
	delle $n$-uple che la soddisfano lo \`e. \`E
	\textit{\emph{ricorsivamente enumerabile}}
	se l'insieme delle $n$-uple che la soddisfano lo \`e. Una relazione
	ricorsivamente enumerabile \`e effettivamente semidecidibile,
	il viceversa segue dalla tesi di Church.

\section{Risultati precedenti su rappresentabilit\`a
e ricorsivit\`a}
	Sia $T$ una teoria con uguaglianza nel linguaggio $\mathcal{L}_A$
	dell'aritmetica. Una funzione $f$ si dice \emph{\textit{rappresentabile}}
	in $T$ se esiste una formula $\varphi(x_1,\,\dots,x_n,\,y)$ di $T$
	tale che:\\
	
	\begin{quote}
	Se $f(k_1,\,\dots,\,k_n)=m$ allora $\vdash_T\forall y(\varphi(
	\overline{k_1},\,\dots,\,\overline{k_n},\,y)\leftrightarrow y=\overline{m})$.
	\end{quote}
	Una relazione $R$ si dice \emph{esprimibile} in T se
	esiste una formula $\varphi(x_1,\,\dots,x_n)$ di $T$
	tale che:\\
	
	\begin{quote}
	Se $R(k_1,\,\dots,\,k_n)$ \`e vera allora $\vdash_T\varphi(\overline{k_1},\,
	\dots,\,\overline{k_n})$,\\
	Se $R(k_1,\,\dots,\,k_n)$ \`e falsa allora $\vdash_T\neg\varphi(\overline{k_1},\,
	\dots,\,\overline{k_n})$.\\
	\end{quote}
	Ogni funzione ricorsiva totale \`e rappresentabile in $T$. Ogni relazione
	ricorsiva (complementata) \`e esprimibile in $T$.
	
	Una relazione $R$ si dice \textit{\emph{semirappresentabile}} in $T$
	se esiste una formula $\varphi(x_1,\,\dots,\,x_n)$ tale che:
	
	$$
	R(k_1,\,\dots,\,k_n) \mbox{\:\`e vera sse\:} \vdash_T\varphi(\overline{k_1},\dots,\,
	\overline{k_n})
	$$
	Una relazione \`e semirappresentabile se e solo se \`e ricorsivamente enumerabile.
	
	\textit{La rappresentabilit\`a e l'esprimibilit\`a ci permettono di parlare
	di funzioni e relazioni all'interno del sistema formale stesso}.
	
\section{Decidibilit\`a}
	
	\begin{defi}
		Un insieme di simboli, o espressioni, o oggetti pi\`u complicati \`e detto
		\emph{ricorsivo} sse l'insieme dei numeri di codifica
		dei suoi elementi \`e ricorsivo.
	\end{defi}
	
	\begin{defi}
		Un linguaggio si dice \emph{ricorsivo} se
		l'insieme dei numeri di codifica dei suoi simboli \`e ricorsivo.
	\end{defi}

	\begin{defi}
		Si dice \emph{teoria} un insieme di enunciati\footnote{Con enunciato
		si intende formula della teoria senza variabili libre.} che contiene
		tutti gli enunciati del suo linguaggio che sono dimostrabili a partire
		da essa.
	\end{defi}
	
	\begin{defi}
		Una teoria $T$ si dice \emph{assiomatizzabile} se esiste un
		un insieme decidibile $\Gamma$ di enunciati  tale che $T$ consiste
		di tutte e soli gli enunciati dimostrabili da $\Gamma$. Si dice
		\emph{ricorsivamente assiomatizzabile} se $\Gamma$ \`e ricorsivo.
		Per la tesi di Church le due definizioni sono equivalenti e noi le
		confonderemo.
	\end{defi}
	
	\begin{defi}
		Una teoria si dice \emph{decidibile} se \`e un insieme ricorsivo.
	\end{defi}
	
	\begin{defi}
		Una teoria $T$ si dice \emph{\textit{sintatticamente consistente}} o
		\emph{\textit{coerente}}
		se non esiste un enunciato $\varphi$ in $T$ tale che $\vdash_{T}\varphi$
		e $\vdash_{T}\neg\varphi$; equivalentemente, per la regola \textit{ex falso},
		se non dimostra ogni formula del suo linguaggio.
	\end{defi}
		
	\begin{defi}
		Un enunciato $\varphi$ nel linguaggio $\mathcal{L}_T$ di una teoria $T$
		si dice \emph{refutabile} se $\vdash_T\neg\varphi$.
	\end{defi}
	
	\begin{defi}
		Un enunciato $\varphi$ nel linguaggio $\mathcal{L}_T$ di una teoria $T$
		si dice \emph{indecidibile} se non \`e dimostrabile n\'e refutabile.
	\end{defi}
	
	\begin{defi}
		una teoria $T$ si dice \emph{\textit{completa}} se per ogni enunciato 
		$\varphi$ del suo linguaggio vale $\vdash_{T}\varphi$ oppure
		$\vdash_{T}\neg\varphi$.
	\end{defi} 
	
	\begin{defi}
		Una teoria si dice \textit{\emph{incompleta}} se ammette un enunciato
		indecidibile.
	\end{defi}

\subsection{Premessa}
	Nei prossimi tre paragrafi forniremo quattro versioni della dimostrazione
	del teorema di incompletezza di G\"odel.
	
	Le prime tre sono classiche. La prima di queste:
	
	\begin{itemize}
	\item segue l'approccio di Boolos;
	\item \`e classica\footnote{Quando la dimostrazione \`e
	classica, scriveremo $PA$ anzich\'e $HA$, riferendoci all'
	``aritmetica di Peano''.};
	\item \`e molto elegante;
	\item non suppone l'$\omega$-consistenza;
	\item usa il diagonalization lemma;
	\item non produce esplicitamente un enunciato indecidibile.
	\end{itemize}
	La seconda:
	\begin{itemize}
	\item segue l'approccio di G\"odel;
	\item \`e classica;
	\item \`e pi\`u macchinosa;
	\item suppone l'$\omega$-consistenza;
	\item non usa il diagonalization lemma;
	\item produce esplicitamente un enunciato indecidibile.
	\end{itemize}
	La terza:
	\begin{itemize}
	\item segue l'approccio di Rosser, raffinamento del risultato di G\"odel;
	\item \`e classica;
	\item \`e pi\`u macchinosa;
	\item rilassa l'ipotesi di $\omega$-consistenza;
	\item non usa il diagonalization lemma;
	\item produce esplicitamente un enunciato indecidibile.
	\end{itemize}
	La quarta:
	\begin{itemize}
	\item segue l'approccio di Sambin e Maietti;
	\item \`e intuizionista;
	\item \`e elegante;
	\item non necessita dell'ipotesi di $\omega$-consistenza in quanto
	in $HA$ si dimostra l'existence property;
	\item usa il diagonalization lemma;
	\item produce esplicitamente un enunciato indecidibile.
	\end{itemize}

\section{Primo teorema di incompletezza-prima versione}

	I teoremi di incompletezza di G\"odel esprimono i limiti di un qualunque
	sistema formale che sia abbastanza forte da permetterci di ``riprodurre''
	al suo interno la teoria dei numeri. In realt\`a il lavoro iniziale di G\"odel
	faceva riferimento a un sistema specifico, quello descritto da B. Russell e
	A. N. Whitehead nei \textit{Principia ma\-the\-ma\-ti\-ca}, ma il principio
	\`e in generale applicabile anche a tutti gli altri sistemi formali dotati di
	certe propriet\`a, come, nella nostra trattazione, $PA$.\\
	Prima di procedere stabiliamo che con teoria, impiantata su un qualche
	linguaggio, intendiamo un insieme che contiene tutte le formule senza
	variabili libere di quel linguaggio, tali che siano in essa dimostrate.
		
	La codifica di $MPA$ in $PA$, ottenuta attraverso l'aritmetizzazione,
	render\`a possibile definire in $PA$ formule autoreferenziali ed in
	particolare una formula che nega la propria dimostrabilit\`a. Il primo
	teorema di incompletezza prender\`a il via da queste considerazioni e,
	supposta la consistenza di $PA$, mostrer\`a l'esistenza di una formula
	indecidibile\footnote{E dunque l'incompletezza di $PA$.}.

	\begin{prop}
	Sia $T$ una teoria assiomatizzabile. Se $T$ \`e completa, $T$ \`e decidibile.
	\end{prop}
	
	\textsc{Dimostrazione.}\\
	Dobbiamo mostrare che l'insieme $T^*$ dei codici dei teoremi di $T$
	\`e ricorsivo. Che sia ricorsivamente enumerabile lo sappiamo dai
	capitoli precedenti, pertanto dobbiamo solo mostrare che anche il
	suo complemento \`e ricorsivamente enumerabile. Il complemento di
	$T^*$ \`e l'unione dell'insieme $X$ dei numeri che non sono codici
	di formule e dell'insieme $Y$ dei codici di formule la cui negazione
	sta in $T$, per completezza. Ma $X$ \`e ricorsivo visto che lo \`e
	il suo complemento, e $Y$ \`e ricorsivamente enumerabile: la sua
	funzione semicaratteristica si ottiene facilmente componendo la
	funzione primitiva ricorsiva $neg$ con la funzione semicaratteristica
	di $T*$, operazione che lasciamo al lettore. Dunque per il teorema
	di Kleene possiamo concludere.
	\begin{flushright}$\Box$\end{flushright}

\subsection{Diagonalizzazione di una formula}
	L'apparato formale necessario alla dimostrazione dei teoremi di
	G\"odel \`e finalmente completo. L'obiettivo delle prossime
	righe sar\`a quello di implementare e di studiare il problema
	dell'autoreferenza in $PA$. Cominciamo dalla nozione di
	diagonalizzazione:

	\begin{defi}Data una qualunque formula $\varphi$, la sua
	diagonalizzazione \`e la formula $\exists x(x=\overline
	{\gdnum{\varphi}} \& \varphi)$.
	\end{defi}
	In particolare, se $\varphi(x)$ \`e una formula dotata di una
	sola variabile libera $x$, notiamo l'equivalenza logica $\exists
	x(x=\overline{\gdnum{\varphi}} \& \varphi)\leftrightarrow\varphi
	(\overline{\gdnum{\varphi}})$, che possiamo leggere come: "il
	numerale del G\"odeliano di $\varphi$ la verifica".

	\begin{prop}
	\label{pro:funPriRic}
	Esiste una funzione primitiva ricorsiva $$diag:\mathbb{N}\rightarrow\mathbb{N}$$
	tale che, per ogni formula $\varphi$, $diag(\gdnum{\varphi})=\gdnum{\exists
	x(x=\overline{\gdnum{\varphi}} \& \varphi)}$.
	\end{prop}
	
	\textsc{Dimostrazione.}\\
	La funzione
	
	\begin{eqnarray*}
	num: \mathbb{N} & \longrightarrow & \mathbb{N}\\
	n &\mapsto& \gdnum{\overline{n}}
	\end{eqnarray*}
	\`e definita dallo schema di ricorsione
	
	$$
	\left\{\begin{array}{l}
	num(0)=19\\
	num(n+1)=\gdnum{s} \star \gdnum{(} \star num(n) \star\gdnum{)}
	\end{array}\right.
	$$
	ed \`e dunque primitiva ricorsiva. Applicata a un numero qualunque,
	restituisce il codice del numerale che rappresenta quel numero.
	Grazie ad essa possiamo definire
	
	$$
	diag(n):=\gdnum{\exists x(x=}\star num(n)\star\gdnum{\&}\star n\star\gdnum{)}.
	$$
	Cos\`i costruita, la funzione $diag$ \`e primitiva ricorsiva,
	essendo composizione di funzioni primitive ricorsive. Non resta
	che verificare la condizione definitoria. Data $\varphi$ formula,
	si PA
	
	\begin{eqnarray*}
	diag(\gdnum{\varphi})=\gdnum{\exists x(x=}\star num(\gdnum{\varphi})
	\star\gdnum{\&}\star \gdnum{\varphi}\star\gdnum{)}=\\
	=\gdnum{\exists x(x=}\star \gdnum{\overline{\gdnum{\varphi}}})\star
	\gdnum{\&}\star \gdnum{\varphi}\star\gdnum{)}=\\
	=\gdnum{\exists x(x=\overline{\gdnum{\varphi}}\&\varphi)}
	\end{eqnarray*}
	e dunque la tesi.\begin{flushright}$\Box$\end{flushright}
	
	In altri termini, la funzione $diag$ appena definita \`e primitiva ricorsiva
	(quindi totale) ed associa, in particolare, al codice di una formula il codice
	della corrispondente diagonalizzazione.
	
\subsection{Dimostrazione del primo teorema di incompletezza di G\"odel}
%lemma diagonale
%indefinibilità di theta
%indecidibilità essenziale
%primo teorema di incompletezza
	
	Forti dei risultati a cui siamo arrivati, procediamo verso la prova del
	primo teorema di incompletezza di G\"odel, dal quale ci separano soltanto
	qualche lemma e un teorema. Il primo passo consiste nella dimostrazione del:
	
	\begin{prop}[Lemma diagonale]
	\label{lem:diagLemma}
	Sia $T$ una teoria abbastanza forte da esprimere l'aritmetica, e sia
	$L$ il linguaggio sul quale essa \`e impiantata. Per ogni formula
	$\psi(x)$ di $L$ esiste una formula $\delta_\psi$ tale che
	
	$$
	\vdash_{T} \delta_\psi \leftrightarrow \psi(\overline{\ulcorner
	\delta_\psi \urcorner})
	$$
	\end{prop}
	Dove con  ``teoria abbastanza forte da esprimere l'aritmetica'' si intende
	una teoria che possa rappresentare i naturali, l'uguaglianza e l'ordine dei
	naturali, la somma e il prodotto. Il requisito viene dettato dalla necessit\`a
	di effettuare l'aritmetizzazione per conseguire l'autoriferimento, il che
	risulter\`a chiaro dalla dimostrazione. Evidentemente $PA$ risulta essere
	una teoria abbastanza forte. Prima di dimostrare il lemma, osserviamo che
	lo possiamo leggere come: ``per qualunque propriet\`a si possa descrivere
	attraverso una formula del linguaggio $L$, esiste un'altra formula nello
	stesso linguaggio che viene dimostrata dalla teoria se e solo se il suo
	codice gode della propriet\`a in questione''.
	
	\textsc{Dimostrazione.}\\ 
	Dimostriamo il lemma esibendo la $\delta_\psi$ per una generica formula $\psi(x)$.\\
	Sappiamo che la funzione $diag:\mathbb{N}\rightarrow\mathbb{N}$, che applicata
	al codice di una formula restituisce il codice della relativa diagonalizzazione,
	\`e primitiva ricorsiva. Quindi  \`e totale e soprattutto \`e rappresentabile.
	Ricordiamo che questo significa che esiste una formula $\varphi_{diag}$ tale che
	se $diag(n)=m$ allora $\vdash_{T} \forall y(\varphi_{diag}(\overline{n},y)
	\leftrightarrow y=\overline{m})$.\\
	Definiamo:
	
	$$
	\chi(x)\equiv\exists y(\varphi_{diag}(x,y)\&\psi(y))
	$$
	nella quale notiamo la dipendenza da $\psi$, e chiamiamo il suo numero di G\"odel:
	
	$$
	a\equiv\ulcorner\chi(x)\urcorner
	$$
	quindi costruiamo la $\delta_\psi$ di interesse come diagonalizzazione di $\chi(x):$
	
	$$
	\delta_\psi \equiv \exists x(x=\overline{a} \& \chi(x))
	$$
	e diamo un nome al suo G\"odeliano:
	
	$$
	d\equiv \ulcorner\delta_\psi \urcorner
	$$
	Dal momento che in generale $\exists x(x = t \& A(x))$
	equivale ad $A(t)$, notiamo che nello specifico vale l'equivalenza
	
	$$
	\vdash_{T} \delta_\psi \leftrightarrow \chi(\overline{a})
	$$
	Da questo e dalla definizione di $\chi(x)$, se riusciamo a
	mostrare che $$\vdash_{T} \exists y(\varphi_{diag}(\overline{a},y)
	\&\psi(y)) \leftrightarrow \psi(\overline{d})
	$$
	possiamo concludere.

	Dato che $\delta_\psi$ \`e la diagonalizzazione di $\chi(x)$ si PA che
	$diag(a)=d$ e dunque, per rappresentabilit\`a:
	
	$$
	\vdash_{T} \forall y(\varphi_{diag}(\overline{a},y)\leftrightarrow y=\overline{d})
	$$
	Ma allora, ricordandosi che in generale se vale $A \leftrightarrow B$
	vale anche $(A \& C) \leftrightarrow (B \& C)$, abbiamo che
	
	$$
	\vdash_{T} \forall y((\varphi_{diag}(\overline{a},y)\&\psi(y))\leftrightarrow
	(y=\overline{d}\&\psi(y)))
	$$
	Per cui, visto che se vale una formula quantificata universalmente allora
	anche l'esistenziale vale di sicuro, possiamo affermare che
	
	$$
	\vdash_{T} \exists y((\varphi_{diag}(\overline{a},y)\&\psi(y))\leftrightarrow
	(y=\overline{d}\&\psi(y)))
	$$
	Dalla quale si arriva a
	
	$$
	\vdash_{T} \exists y(y=\overline{d}\&\psi(y)) \leftrightarrow \psi(\overline{d})
	$$
	Dunque la tesi.
	\begin{flushright}$\Box$\end{flushright}

	La dimostrazione appena conclusa fornisce un procedimento esplicito
	per costruire $\delta_{\psi}$ a partire dalla formula $\psi$ data.\\  
	Diamo ora la nozione di definibilit\`a di un insieme di naturali.
	
	\begin{defi} Un insieme $S$ di naturali \`e definibile in una
	teoria consistente $T$ se esiste una formula $D(x)$ tale che
	per ogni naturale $n$ vale $\vdash_{T}D(\overline{n})$ se $n$
	sta in $S$ e $\vdash_{T}\neg D(\overline{n})$ se $n$ non sta in $S$.
	\end{defi}
	
	Osserviamo che:
	
	\begin{prop}
	Sia $S$ un insieme ricorsivo (o decidibile), allora S \`e
	definibile.
	\end{prop}
	
	\textsc{Dimostrazione.}\\
	Se $S$ \`e decidibile allora la sua funzione caratteristica $\chi_S$ \`e
	calcolabile totale, quindi rappresentabile da un predicato a due posti
	$\varPhi(x,y)$. Allora \`e immediato verificare che possiamo definire $S$
	con il predicato $D(x)\equiv \varPhi(x, \overline{1}) $.
	\begin{flushright}$\Box$\end{flushright}
	
	Ora abbiamo il vocabolario per dimostrare che:
	
	\begin{prop}
	Sia $T$ una teoria consistente abbastanza forte da esprimere l'aritmetica,
	l'insieme dei numeri di G\"odel dei teoremi di $T$ non \`e definibile in $T$.
	\end{prop}
	
	\textsc{Dimostrazione.}\\
	Sia $\Theta$ l'insieme dei codici dei teoremi di $T$. Supponiamo che il predicato
	$\theta(y)$ definisca $\Theta$. Per il lemma diagonale sappiamo che esiste una
	formula chiusa $G$ tale che
	
	$$
	\vdash_{T} G \leftrightarrow \neg\theta(\overline{\ulcorner G \urcorner})
	$$
	Quindi se $\vdash_{T} G$ allora $\overline{\ulcorner G \urcorner}$ sta in
	$\Theta$ e $\vdash_{T} \theta(\overline{\ulcorner G \urcorner})$\\
	ma se $\vdash_{T} G$ allora anche $\vdash_{T} \neg\theta(\overline
	{\ulcorner G \urcorner})$,
	contro la consistenza.\\
	Se $\not\vdash_{T} G$ allora $\overline{\ulcorner G \urcorner}$ non sta
	in $\Theta$ e $\vdash_{T} \neg\theta(\overline{\ulcorner G \urcorner})$\\
	ma se $\vdash_{T} \neg\theta(\overline{\ulcorner G \urcorner})$ allora
	anche $\vdash_{T} G$, il che \`e una contraddizione.\\
	L'errore sta nell'aver assunto $\Theta$ definibile, dunque la tesi.
	\begin{flushright}$\Box$\end{flushright}
	
	\begin{thm}[Indecidibilit\`a essenziale]
	Nessuna teoria T, consistente e abbastanza forte da esprimere l'aritmetica,
	\`e decidibile.
	\end{thm}
	
	\textsc{Dimostrazione.}\\
	Dal lemma precedente l'insieme $\Theta$ dei codici dei teoremi di $T$ non
	\`e definibile in $T$. Ma ogni insieme ricorsivo \`e definibile in $T$,
	quindi $\Theta$ non \`e ricorsivo, ovvero $T$ non \`e decidibile.
	\begin{flushright}$\Box$\end{flushright}
	
	\begin{thm}[Primo teorema di incompletezza]
	Non esiste nessuna teoria T, abbastanza forte da esprimere l'aritmetica,
	che sia assiomatizzabile, completa e sintatticamente consistente.
	\end{thm}
	
	\textsc{Dimostrazione.}\\
	Ogni teoria assiomatizzabile e completa \`e decidibile, ma ogni teoria consistente
	in grado di esprimere l'aritmetica \`e indecidibile.
	\begin{flushright}$\Box$\end{flushright}
	
\section{Primo teorema di incompletezza-seconda e terza versione.}
		
\subsection{Enunciati indecidibili}
	
	Sia $T$ una teoria del prim'ordine con gli stessi simboli di $PA$.
	
	\begin{defi}
	$T$ si dice $\omega$-consistente sse, per ogni formula $\varphi(x)$ di $T$,
	se $\vdash_T\varphi(\overline{n})$ per ogni numero naturale $n$, allora
	$\not\vdash\exists x\neg\varphi(x)$.
	\end{defi}
	
	\begin{thm}
	Se $T$ \`e $\omega$-consistente, allora $T$ \`e consistente.
	\end{thm}
	\textsc{Dimostrazione.} Supponiamo $T$ $\omega$-consistente.
	Sia $\varphi(x)$ una formula tale che $\vdash_T
	\varphi(x)$, segue che $\vdash_T\varphi(\overline{n})$ per ogni
	numero naturale $n$. Per $\omega$-consistenza segue che
	$\not\vdash_T\exists x\neg\varphi(x)$. Dunque $K$
	\`e consistente, altrimenti per la regola \textit{ex falso} ogni formula
	sarebbe dimostrabile.\begin{flushright}$\Box$\end{flushright}
	
	Introduciamo le seguenti relazioni
	
	$$
	W_1(u,\,y):=Form(u)\& Fv(u,\,\gdnum{x_1})\& Proof(y,\,Sub(
	\gdnum{\overline{u}},\,\gdnum{x_1},\,u))
	$$
	che dice: $u$ \`e il numero di codifica di una formula $\varphi(x_1)$
	che contiene la variabile libera $x_1$
	e $y$ \`e il numero di codifica di una dimostrazione di $\varphi(\overline{u})$;
	
	$$
	W_2(u,\,y):=Form(u)\& Fv(u,\,\gdnum{x_1})\& Proof(y,\,Sub(
	\gdnum{\overline{u}},\,\gdnum{x_1},\,\gdnum{\neg}\ast u))
	$$
	che dice: $u$ \`e il numero di codifica di una formula $\varphi(x_1)$
	che contiene la variabile libera $x_1$
	e $y$ \`e il numero di codifica di una dimostrazione di
	$\neg\varphi(\overline{u})$.
	
	$W_1(u,\,y)$ \`e primitiva ricorsiva, dunque \`e esprimibile in $PA$ per mezzo
	di una formula $\mathcal{W}_1(x_1,\,x_2)$ con $x_1$ e $x_2$ variabili libere.
	Consideriamo la formula:
	
	$$
	g_{PA}(x_1)\equiv\forall x_2\neg\mathcal{W}_1(x_1,\,x_2)
	$$
	sia $m=\gdnum{g_{PA}}$, consideriamo l'enunciato:
	
	$$
	G_{PA}\equiv\forall x_2\neg\mathcal{W}_1(\overline{m},\,x_2)
	$$
	detto \emph{enunciato di G\"odel}. Si ha che
	(I) $W_1(m,\,y)$ vale se e solo se $y$ \`e il numero di codifica
	di una dimostrazione in $PA$ di $G_{PA}$.
	
	\begin{thm}[Teorema di G\"odel per $PA$, 1931]
	Se $PA$ \`e consistente, allora $G_{PA}$ non \`e
	dimostrabile in PA. Se PA \`e $\omega$-consistente,
	allora $G_{PA}$ non \`e refutabile in PA. In
	particolare, se PA \`e $\omega$-consistente, $G_{PA}$
	\`e indecidibile.
	\end{thm}
	
	\textsc{Dimostrazione.}\\
	Supponiamo $PA$ consistente e $G_{PA}$ dimostrabile in $PA$.\\
	Sia $k$ il numero di codifica di una dimostrazione in $PA$.\\
	Per (I) abbiamo
		$$W_1(m,\,k)$$
	per esprimibilit\`a segue che
		$$\vdash_{PA} \mathcal{W}_1(\overline{m},\,\overline{k}).$$
	Ma poich\'e 
		$$\vdash_{PA}G_{PA}$$
	per pura logica segue che
		$$\vdash_{PA}\neg\mathcal{W}_1(\overline{m},\,\overline{k}).$$
	Da cui segue che 
		$$\vdash_{PA}\mathcal{W}_1(\overline{m},\,\overline{k})$$
	e
		$$\vdash_{PA}\neg\mathcal{W}_1(\overline{m},\,\overline{k})$$
	e ci\`o \`e contro la consistenza di $PA$.
	
	Supponiamo $PA$ $\omega$-consistente e $G_{PA}$ refutabile.\\
	Per consistenza segue che
		$$\not\vdash_{PA} G_{PA}.$$
	Da cui segue che per ogni numero naturale $n$, $n$ non \`e il numero di codifica di
	una dimostrazione in $PA$ di $G_{PA}$.\\
	Per (I) segue che per ogni $n$, $W_1(m,\,n)$ \`e falsa.\\
	Per rappresentabilit\`a segue che per ogni $n$
		$$\vdash_{PA}\neg\mathcal{W}_1(\overline{m},\,\overline{n}).$$
	Per $\omega$-consistenza segue che
		$$\not\vdash_{PA}\exists x_2\neg\neg \mathcal{W}_1(\overline{m},\,x_2)$$
	da cui segue che
	$$\not\vdash_{PA}\exists x_2\mathcal{W}_1(\overline{m},x_2).$$
	Ci\`o \`e contro la refutabilit\`a di $G_{PA}$.
	\begin{flushright}$\Box$\end{flushright}
	
	$W_2(u,\,y)$ \`e primitiva ricorsiva, dunque \`e esprimibile in $PA$ per mezzo
	di una formula $\mathcal{W}_2(x_1,\,x_2)$ con $x_1$ e $x_2$ variabili libere.
	Consideriamo la formula:
	
	$$
	r_{PA}(x_1)\equiv
	\forall x_2(\mathcal{W}_1(x_1,\,x_2)\rightarrow
	\exists x_3(x_3\leq x_2\& \mathcal{W}_2(x_1,\,x_3)))
	$$
	sia $n=\gdnum{r_{PA}}$, consideriamo l'enunciato:
	$$
	R_{PA}\equiv
	\forall x_2(\mathcal{W}_1(\overline{n},\,x_2)\rightarrow
	\exists x_3(x_3\leq x_2\& \mathcal{W}_2(\overline{n},\,x_3)))
	$$
	detto \emph{enunciato di Rosser}. Si ha che
	(II) $W_1(n,\,y)$ vale se e solo se $y$ \`e il numero di codifica
	di una dimostrazione in $PA$ di $R_{PA}$ e (III)
	$W_2(n,\,y)$ vale se e solo se $y$ \`e il numero di codifica
	di una dimostrazione in $PA$ di $\neg R_{PA}$.
	
	\begin{thm}[Teorema di G\"odel-Rosser, 1936]
	Se $PA$ \`e consistente, allora $R_{PA}$ \`e indecidibile.
	\end{thm}
	\textsc{Dimostrazione.} Supponiamo $PA$ consistente e $R_{PA}$ dimostrabile.\\
	Sia $k$ il numero di codifica di una dimostrazione in $PA$.\\
	Per (II) segue 
		$$W_1(n,\,k).$$
	Per esprimibilit\`a segue che
		$$\vdash_{PA}\mathcal{W}_1(\overline{n},\,\overline{k}).$$
	Ma per dimostrabilit\`a di $R_{PA}$ e per pura logica segue che
		$$\vdash_{PA}\mathcal{W}_1(\overline{n},\,\overline{k})
		\rightarrow\exists x_3(x_3\leq\overline{k}\&\mathcal{W}_2
		(\overline{n},\,x_3)).$$
	Per pura logica segue che
		$$\vdash_{PA}\exists x_3(x_3\leq\overline{k}\&\mathcal{W}_2
		(\overline{n},\,x_3)).$$
	Ora, dalle ipotesi segue che
		$$\not\vdash\neg R_{PA}.$$
	Per (III) segue che $W_2(n,\,y)$ \`e falsa per ogni
	numero naturale $y$.\\
	Per esprimibilit\`a segue che $\vdash_{PA}
	\neg\mathcal{W}_2(\overline{n},\,\overline{j})$ per ogni numero naturale
	$j$.\\
	Da cui segue che 
		$$\vdash_{PA}
		\neg\mathcal{W}_2(\overline{n},\,0)\&
		\neg\mathcal{W}_2(\overline{n},\,\overline{1})\&
		\dots\&
		\neg\mathcal{W}_2(\overline{n},\,\overline{k}).$$
	Per proposizioni precedenti su $HA$ e dunque $PA$\footnote{In quanto
	ci\`o che vale in logica intuizionista, vale in logica classica.}
	segue che
		$$\vdash_{PA}\forall x_3(x_3\leq\overline{k}
		\rightarrow\neg\mathcal{W}_2(\overline{n},\,x_3)).$$
	Per pura logica segue che
		$$\vdash_{PA}\neg\exists x_3(x_3\leq\overline{k}
		\&\mathcal{W}_2(\overline{n},\,x_3)).$$
	Ma questa \`e la negazione di una formula derivata precedentemente.
	Ci\`o contraddice la consistenza di $PA$.
	
	Supponiamo $R_{PA}$ refutabile.\\
	Sia $r$ il numero di codifica di una
	dimostrazione di $\neg R_{PA}$.\\
	Per (III) vale $W_2(n,\,r)$.\\
	Per rappresentabilit\`a segue che 
		$$\vdash_{PA}\mathcal{W}_2(\overline{n},\,\overline{r}).$$
	Per consistenza di $PA$ segue che che 
		$$\not\vdash_{PA}R_{PA}.$$
	Per (II) segue che $W_1(n,\,y)$ \`e falsa per ogni
	numero naturale $y$.\\
	Da cui segue che $\vdash_{PA}\neg\mathcal{W}_1
	(\overline{n},\,\overline{j})$ per ogni numero naturale $j$.\\
	In particolare
		$$\vdash_{PA}\neg\mathcal{W}_1(\overline{n},\,0)\&
		\neg\mathcal{W}_1(\overline{n},\,\overline{1})\&\dots
		\&\neg\mathcal{W}_1(\overline{n},\,\overline{r}).$$
	Per proposizioni precedenti su $HA$ segue che:
	
	\begin{quote}
	a) $\vdash_{PA} x_2\leq\overline{r}\rightarrow\neg\mathcal{W}_1
	(\overline{v},\,x_2)$.
	\end{quote}
	D'altra parte, consideriamo la seguente deduzione
	
	$$
	\begin{array}{ll}
	\vdash_{PA}\overline{r}\leq x_2						 								& \mbox{ipotesi}\\
	\vdash_{PA}\mathcal{W}_2(\overline{n},\,\overline{r})								& \mbox{gi\`a dimostrato prima}\\
	\vdash_{PA}\overline{r}\leq x_2\&\mathcal{W}_2(\overline{n},\,\overline{r})		& \mbox{1, 2, pura logica}\\
	\vdash_{PA}\exists x_3(x_3\leq x_2\&\mathcal{W}_2(\overline{n},\,x_3))			& \mbox{3, pura logica}
	\end{array}
	$$
	Per $1-4$ e per pura logica segue che:
	
	\begin{quote}
	b) $\vdash_{PA}\overline{r}\leq x_2\rightarrow
	\exists x_3(x_3\leq x_2\&\mathcal{W}_2(\overline{n},\,x_3)).$
	\end{quote}
	Per proposizioni precedenti su $HA$ segue che:
	
	\begin{quote}
	c) $\vdash_{PA} x_2\leq\overline{r}\vee\overline{r}\leq x_2.$
	\end{quote}
	Per $a-c$ e per pura logica segue che:
	
	$$
	\vdash_{PA}\neg\mathcal{W}_1(\overline{n},\,x_2)\vee\exists x_3
	(x_3\leq x_2\&\mathcal{W}_2(\overline{n},\,x_3)).
	$$
	Per pura logica segue che:
	
	$$
	\vdash_{PA}\forall x_2(\mathcal{W}_1(\overline{n},\,x_2)\rightarrow\exists x_3
	(x_3\leq x_2\&\mathcal{W}_2(\overline{n},\,x_3))).
	$$
	Da cui segue che
		$$\vdash_{PA} G_{PA}.$$
	Per refutabilit\`a ci\`o contraddice la consistenza di
	$PA$.\begin{flushright}$\Box$\end{flushright}
	
	Se si rende $PA$ pi\`u forte, ad esempio aggiungendovi $G_{PA}$ e
	ottenendo $PA_1$, $PA_1$ ammette ancora un
	enunciato indecidibile. Infatti
	qualunque funzione ricorsiva, essendo rappresentabile in $PA$, \`e
	rappresentabile in $PA_1$ e, ovviamente, la relazione $W_{1,PA_1}$,
	scritta per $PA_1$, sar\`a primitiva ricorsiva. Ma questo \`e
	tutto ci\`o di cui abbiamo
	bisogno per ottenere il risultato di G\"odel.
	
	Pi\`u in generale, poich\'e per giungere ai teoremi
	di G\"odel e G\"odel-Rosser abbiamo
	usato la rappresentabilit\`a delle funzioni in $PA$ e la primitiva
	ricorsivit\`a di $Proof$, cio\`e di $Der$, ogni estensione $T$ di $PA$ che
	soddisfa a queste propriet\`a ammetter\`a un enunciato indecidibile.
	
	Poich\'e per ottenere $Der$ in $T$ primitivo ricorsivo \'e sufficiente
	che $T$ sia ricorsivamente assiomatizzabile segue che:
	
	\begin{thm}
	Ogni estensione consistente e ricorsivamente assiomatizzabile di $PA$
	ammette un enunciato indecidibile.
	\end{thm}
	
\subsection{Dimostrabilit\`a e verit\`a}
	
	Poich\'e $\mathcal{W}_1$ esprime $W_1$ in $PA$,
	l'interpretazione standard di $G_{PA}:=\forall x_2\neg
	\mathcal{W}_1(\overline{m},\,x_2)$ afferma che $W_1(m,\,x_2)$ \`e falsa per
	ogni numero naturale $x_2$. Per (I) $\not\vdash_{PA}G_{PA}$.
	Cio\`e \textit{$G_{PA}$ afferma la propria indimostrabilit\`a in
	$PA$}. Per il teorema di G\"odel, se $PA$ \`e consistente,
	$G_{PA}$ \`e indimostrabile. Da questo segue che $G_{PA}$ \textit{\`e vera per
	l'interpretazione standard ma non \`e dimostrabile}. Un ragionamento analogo
	si pu\`o fare per $R_{PA}$.
	
\section{Primo teorema di incompletezza-quarta versione}

	\`E fondamentale la seguente propriet\`a di $HA$:

	\begin{thm}[\textbf{existence property}]
	Data una qualsiasi formula $\phi(x)$ dotata di un'unica variabile libera $x$,
	se vale $\vdash_{HA}\exists x\phi(x)$ allora esiste un numero naturale
	$n$ tale che $\vdash_{HA}\phi(\overline{n})$.
	\end{thm}
	La dimostrazione, che tralasciamo per difficolt\`a e lunghezza,
	\`e basata sull'analisi delle possibili prove.

	Introduciamo un nuovo
	simbolo: per ogni formula $\varphi$ definiamo
	$\Box\varphi\equiv TH(\overline{\gdnum{\varphi}})$.
	Esprime il fatto che $\varphi$ \`e dimostrabile in $HA$.
	Per diagonalization lemma, precedentemente dimostrato,
	posto $\psi(x)=\neg TH(x)$, otteniamo:

	\begin{prop}
	\label{auto}
	Esiste una formula $G$ tale che $\vdash_{HA} G \leftrightarrow\neg\Box G$.
	$G$ \`e detta \textit{enunciato di G\"odel}.
	\end{prop}

	\begin{thm} Il predicato $\Box$ soddisfa le seguenti condizioni:

	\begin{itemize}
 	\item[\small{HBL.1}:] se $\vdash_{HA}A$ allora $\vdash_{HA}\Box A$;
 	\item[\small{HBL.2}:] $\vdash_{HA} \Box A\rightarrow \Box(\Box A)$;
 	\item[\small{HBL.3}:] $\vdash_{HA} \Box(A\rightarrow B)
 	\rightarrow (\Box A\rightarrow \Box B)$;
	\end{itemize}
	dette \textit{condizioni HBL di derivabilit\`a}, con HBL che
	sta per Hilbert-Bernays-L\"ob.
	\end{thm}

	Da HBL.1 segue:

	\begin{prop}
	\label{hg1}Se $\vdash_{HA} G$ allora $\vdash_{HA}\Box G$.
	\end{prop}

	\begin{oss}
	\label{oss:HBL}
	La condizione $HBL.1$ pu\`o essere rafforzata, valendo anche l'implicazione inversa:
	\begin{eqnarray}
	\vdash_{HA}\Box A\:\:\:\:\Rightarrow\:\:\:\:\vdash_{HA} A.\nonumber
	\end{eqnarray}
	Si supponga infatti che $\vdash_{HA}\Box A$, ossia $\vdash_{HA}\exists
	yPR(\overline{y},\overline{\gdnum{A}})$.
	Per ``existence property''  esiste un numero naturale $n$ tale
	che $\vdash_{HA} PR(\overline{n},\overline{\gdnum{A}})$.
	Ci\`o significa che $n$ codifica la prova di $A$, pertanto a
	partire da $n$ si pu\`o costruire la derivazione di $A$ in $HA$.
	Si conclude che $\vdash_{HA} A$.
	\end{oss}

	\begin{thm}
	\label{teo:nonG}
	Se $HA$ \`e consistente allora $\not\vdash_{HA} G$.
	\end{thm}

	\textsc{Dimostrazione.}\\
	Supponiamo $\vdash_{HA} G$. Per proposizione \ref{hg1}
	segue che $\vdash_{HA}\Box G$.
	D'altra parte per proposizione \ref{auto} segue che
	$\vdash_{HA}\neg\Box G$. Ci\`o \`e 
	contro l'ipotesi di consistenza di $HA$.
	\begin{flushright}$\Box$\end{flushright}

	\begin{prop}
	\label{lem:hbl3bis}
	Date $\phi$ e $\psi$ formule, si ha
	$$
	\begin{array}{ll}
	1. & \vdash_{HA}\phi\rightarrow\psi
	\Rightarrow\vdash_{HA}\Box\phi\rightarrow\Box\psi;\\
	2. & \vdash_{HA} \Box\phi\&\Box\psi\rightarrow\Box(\phi\&\psi);\\
	3. & \vdash_{HA} \neg \Box \bot \rightarrow G.
	\footnote{Ricordiamo che $\bot$ \`e
	definita nel linguaggio di $HA$
	come $(0=1)$. Dagli assiomi di $HA$ segue che soddisfa gli
	assiomi standard del falso.}
	\end{array}
	$$
	\end{prop}

	\textsc{Dimostrazione.}\\

	Punto 1.\\
	Per HBL.1 seghe che se

	$$\vdash_{HA} \phi \rightarrow \psi$$
	allora 

	$$\vdash_{HA} \Box(\phi\rightarrow\psi)$$
	per HBL.3 segue la tesi.

	Punto 2.\\
	Poich\'e vale 
	$$\vdash_{HA} \phi \rightarrow (\psi \rightarrow \phi\&\psi)$$
	ed \`e possibile derivare in logica intuizionistica la seguente regola
	$$\prooftree
	\Gamma\vdash A \rightarrow B \justifies \Gamma,A\vdash B\using{\rightarrow ri}
	\endprooftree$$
	segue che:
	$$\prooftree
	\[\[\[\[\[
	\vdash_{HA} \phi \rightarrow (\psi \rightarrow \phi\&\psi)\justifies
 	 \vdash_{HA} \Box\phi\rightarrow\Box(\psi\rightarrow\phi\&\psi)\using{\text{proposizione \ref{lem:hbl3bis}}}\] \justifies
      \Box\phi\vdash_{HA}\Box(\psi\rightarrow\phi\&\psi)\using{\rightarrow ri}\] \justifies
       \Box\phi\vdash_{HA}\Box\psi\rightarrow\Box(\phi\&\psi)\using{\text{\tiny HBL.3}}\]\justifies
        \Box\phi,\Box\psi\vdash_{HA}\Box(\phi\&\psi)\using{\rightarrow ri}\]\justifies
         \Box\phi\&\Box\psi\vdash_{HA}\Box(\phi\&\psi)\using{\& left}\]\justifies
          \vdash_{HA}\Box\phi\&\Box\psi\rightarrow\Box(\phi\&\psi)\using{\rightarrow right}
	\endprooftree$$
	
	Punto 3.\\
	Per proposizione \ref{auto} e per contronominale si ha

	$$\vdash_{HA} \neg\neg\Box G \leftrightarrow \neg G$$
	Poich\'e in generale vale

	$$\vdash_{HA} \Box G \rightarrow \neg\neg\Box G$$
	segue che
	
	$$\vdash_{HA}\Box G \rightarrow \neg G$$
	Per HBL1 segue che
	
	$$\vdash_{HA}\Box(\Box G \rightarrow\neg G)$$
	per HBL3 segue che

	$$\vdash_{HA}\Box\Box G \rightarrow \Box\neg G$$
	per HBL2 segue che
	$$\vdash_{HA} \Box G \rightarrow \Box\Box G$$
	che insieme all'ultimo risultato comporta che

	$$\vdash_{HA} \Box G \rightarrow \Box\neg G$$
	dal cui segue che

	$$\vdash_{HA} \Box G \rightarrow (\Box\neg G \& \Box G)$$
	per proposizione \ref{lem:hbl3bis} segue che
	
	$$\vdash_{HA} \Box G \rightarrow \Box(\neg G \& G)$$
	da cui segue che
	
	$$\vdash_{HA} \Box G \rightarrow \Box\bot$$
	per contronominale segue che
	
	$$\vdash_{HA} \neg\Box\bot \rightarrow \neg\Box G$$
	per proposizione \ref{auto} e per composizione segue che:

	$$\vdash_{HA} \neg\Box\bot \rightarrow G.$$
	Ci\`o \`e quanto volevamo dimostrare.
	\begin{flushright}$\Box$\end{flushright}

	Siamo pronti per dimostrare:

	\begin{thm}[Primo teorema di incompletezza per $HA$]
	\label{teo:inc1}
	Se HA \`e consistente allora $\vdash_{HA}\Box\bot$ \`e indecidibile.
	\end{thm}
	
	\textsc{dimostrazione}\\
 	Assumiamo $\vdash_{HA}\Box\bot$. Per osservazione
 	\ref{oss:HBL} vale $\vdash_{HA}\bot$.
	Poich\`e nel sistema c\`e l'assioma $\vdash_{HA}\neg\bot$,
	segue che che HA \`e inconsistente, contro l'ipotesi.
 	Supponiamo $\vdash_{HA}\neg\Box\bot$.
	Per proposizione \ref{lem:hbl3bis} $\vdash_{HA}\neg\Box\bot\rightarrow G$,
	segue che $\vdash_{HA} G$. Ci\`o \`e contro teorema \ref{teo:nonG}.
	\begin{flushright}$\Box$\end{flushright}
	
	Una dimostrazione alternativa abbastanza intuitiva potrebbe essere la seguente:
	
	\begin{thm}[Primo teorema di incompletezza per $HA$]
	\label{teo:inc1}
	\begin{array}{ll}
	1. Se $\vdash_{HA}\phi$ allora $\vdash_{PA}\phi$;\\
	2. esiste G tale che $\nvdash_{PA} G$ e $\nvdash_{PA} \neg G$ $$
	\end{array}
	Allora da 1. e 2. segue che:\\
	3. esiste G tale che $\nvdash_{HA}$ e $\nvdash_{HA} \neg G$.
	\end{thm}
	
	\textsc{dimostrazione}\\
 	Assumiamo $\vdash_{HA}G$. Allora per 1. segue che $\vdash_{PA} G$, contro l'ipotesi 2..
	Assumiamo $\vdash_{HA}\neg G$. Allora per 1. segue che $\vdash_{PA} \neg G$, contro l'ipotesi 2..
	\begin{flushright}$\Box$\end{flushright}
	Questa dimostrazione, per\`{o}, risulta non soddisfacente ai nostri scopi, in quanto la caratteristica costruttiva che delinea le nostre dimostrazioni viene a mancare nella seconda ipotesi del teorema. Per questo motivo confermiamo come quarta versione del primo teorema di completezza quella data da Maietti-Sambin.

% incompletezza_secondo_per_compilazione

\chapter{Il predicato \ensuremath{\mathrm{Th}}\ e il Secondo Teorema di Incompletezza}

\noindent Nel capitolo \ref{chapter:aritmetizzazione} si è introdotto il predicato $\ensuremath{\mathrm{Th}}(y)\equiv\exists x\ensuremath{\mathrm{Dim}}(x,y)$, ove $\ensuremath{\mathrm{Dim}}(x,y)$ è la formula che rappresenta la relazione binaria $\ensuremath{\mathrm{DIM}}$ che sussiste tra $n$ e $m$ quando $n$ codifica una dimostrazione della formula codificata da $m$.

In questo capitolo studieremo le proprietà formali di questo predicato, mostrando che una loro analisi porta speditamente a risultati significativi sull'aritmetica HA, quali il Teorema di L\"ob e il Secondo Teorema di Incompletezza.

Si può inoltre osservare che, sebbene la presente trattazione sia basata sul sistema HA, tutte le proprietà di \ensuremath{\mathrm{Th}}\ che utilizzeremo sono in realtà condivise dal corrispondente predicato di dimostrabilità $\ensuremath{\mathrm{Th}}_{PA}$ di PA, e che pertanto gli stessi argomenti presentati in questo capitolo possono essere usati per dimostrare tanto il Teorema di L\"ob che il Secondo Teorema di Incompletezza in PA.\\
\bigskip

\section{Condizioni di derivabilità di Hilbert-Bernays-L\"ob}

\noindent In questa sezione presenteremo alcune fondamentali proprietà di \ensuremath{\mathrm{Th}}\ che, combinate con il Lemma di diagonalizzazione, saranno tutto ciò che serve per dimostrare il Teorema di L\"ob. Dal momento che in questa sezione si fa un uso massiccio del predicato $\ensuremath{\mathrm{Th}}$, spesso annidato, è molto utile snellire la notazione introducendo la seguente abbreviazione.\\

\noindent \textbf{Notazione} Se $\varphi$ è una formula aritmetica, poniamo $\Box\varphi\equiv\ensuremath{\mathrm{Th}}(\overline{\ulcorner\varphi\urcorner})$.

\begin{thm}[Condizioni di derivabilità di Hilbert-Bernays-L\"ob] Per tutte le formule aritmetiche $\varphi$ e $\psi$:
\begin{description}
\item[HBL1] se $\vdash_{HA}\varphi$ allora $\vdash_{HA}\Box\varphi$;
\item[HBL2] $\vdash_{HA}\Box(\varphi\to\psi)\to(\Box\varphi\to\Box\psi)$;
\item[Cor] se $\vdash_{HA}\varphi\to\psi$ allora $\vdash_{HA}\Box\varphi\to\Box\psi$;
\item[HBL3] $\vdash_{HA}\Box\varphi\to\Box\Box\varphi$.
\end{description}
\end{thm}

\begin{proof} Siano date $\varphi$ e $\psi$ qualunque.
\begin{description}

\item[HBL1] Se $\vdash_{HA}\varphi$ allora esiste una dimostrazione $D$ di $\varphi$ in HA.\\
Ma allora, per definizione, la relazione $\ensuremath{\mathrm{DIM}}(\ulcorner D\urcorner,\ulcorner\varphi\urcorner)$ sussiste.\\
Poiché $\ensuremath{\mathrm{DIM}}$ è rappresentata dal predicato aritmetico $\ensuremath{\mathrm{Dim}}$, allora vale $\vdash_{HA}\ensuremath{\mathrm{Dim}}(\ulcorner D\urcorner,\ulcorner\varphi\urcorner)$. Dunque $\vdash_{HA}\exists x\ensuremath{\mathrm{Dim}}(x,\ulcorner\varphi\urcorner)$, cioè $\vdash_{HA}\Box\varphi$.\\

\item[HBL2] \`{E} chiaro che in ogni sistema di derivazione, una dimostrazione di $\psi$ può essere ottenuta da una dimostrazione $D$ di $\varphi\to\psi$ e da una dimostrazione $D'$ di $\varphi$ in maniera uniforme. In particolare, nel calcolo dei sequenti:


$$\prooftree
  \stackrel{\stackrel{D'}{\vdots}}{\vdash} \varphi \qquad  \[\stackrel{\stackrel{D}{\vdots}}{\vdash} \varphi\rightarrow \psi \qquad \[\varphi \vdash \varphi \qquad \psi \vdash \psi \justifies \varphi\rightarrow \psi, \varphi \vdash \psi\using{\rightarrow_{left}}\] \justifies \varphi\vdash\psi\using {cut}\]
   \justifies
 \vdash \psi
 \using {cut}\
\endprooftree$$


Se usassimo un sistema in cui le deduzioni fossero mere sequenze di formule, per esempio il calcolo alla Hilbert, la cosa sarebbe ancora più semplice: una dimostrazione di $\psi$ sarebbe data dalla sequenza ottenuta concatenando $D$, $D'$ e aggiungendo la formula $\psi$.

Questo si riflette nel fatto che, indipendentemente dal particolare si\-ste\-ma utilizzato, esiste una funzione (primitiva) ricorsiva $r$ tale che, dati $m$ ed $n$, se $m$ codifica una dimostrazione di $\varphi\to\psi$ e $n$ codifica una dimostrazione di $\varphi$, allora $r(m,n)$ codifica una dimostrazione di $\psi$.

Questa funzione sarà rappresentata in HA da una formula $\rho(x,y)$. Non solo, ma il fatto appena menzionato sarà una semplice proprietà aritmetica dimostrabile in HA, cioè si avrà:
    $$\vdash_{HA}\forall x\forall y(\ensuremath{\mathrm{Dim}}(x,\ulcorner\varphi\to\psi\urcorner)\land\ensuremath{\mathrm{Dim}}(y,\ulcorner\varphi\urcorner)\to \ensuremath{\mathrm{Dim}}(\rho(x,y),\ulcorner\psi\urcorner))$$
    da cui segue:
    $$\vdash_{HA}\exists x\ensuremath{\mathrm{Dim}}(x,\ulcorner\varphi\to\psi\urcorner)\to(\exists x\ensuremath{\mathrm{Dim}}(x,\ulcorner\varphi\urcorner)\to \exists x\ensuremath{\mathrm{Dim}}(x,\ulcorner\psi\urcorner))$$
   cioè $\vdash_{HA}\Box(\varphi\to\psi)\to(\Box\varphi\to\Box\psi)$.\\
    


%$$\prooftree
%\vdash \forall x \forall y (A(x)\ \mbox{\&}\ B(y)\rightarrow C(t)) \qquad \[\[A(x)\ \mbox{\&}\ B(y)\rightarrow C(t)\vdash A(x)\ \mbox{\&}\ B(y)\rightarrow C(t)
%\justifies
%\forall y\ (A(x)\ \mbox{\&}\ B(y)\rightarrow C(t))\vdash A(x)\ \mbox{\&}\ B(y)\rightarrow C(t)
%\using {\forall_{left}}\]
%\justifies
%\forall x \forall y\ (A(x)\ \mbox{\&}\ B(y)\rightarrow C(t))\vdash A(x)\ \mbox{\&}\ B(y)\rightarrow C(t)
%\using {\forall_{left}}\]
%\justifies
%\vdash A(x)\ \mbox{\&} B(x)\rightarrow C(t)
%\using{cut}\
%\endprooftree$$




%$$\prooftree
% \qquad  \[
%A(x)\vdash A(x)
%\using {ind}
%\justifies
%A(x), B(y) \vdash A(x)
%\qquad \[A(x), B(y) \vdash B(y) \justifies A(x), B(y) \vdash A(x)\ \mbox{\&}\ B(y) \using{\mbox{\&}_{right}}\]
%\qquad C(t) \vdash C(t)\justifies
%A(x)\ \mbox{\&} B(x)\rightarrow C(t), A(x), B(y) \vdash C(t)
%\using{\rightarrow_{left}}\
%\endprooftree$$
 
   
   
%$$\prooftree
%\[\[\[\[\[
%A(x), B(y)\vdash C(t)
%\using {cut}\
%\justifies
%A(x), B(y)\vdash \exists x C(x)
% \using {{\exists}_{r}}\]
% \justifies
%A(x), \exists x B(x)\vdash \exists x C(x)
%\using {{\exists}_{F}}\]
%\justifies
%\exists x A(x), \exists x B(x)\vdash \exists x C(x)
%\using {{\exists}_{F}}\]
%\justifies
%\exists x A(x)\ \mbox{\&}\ \exists x B(x)\vdash \exists x C(x)
%\using {\mbox{\&}_{left}}\]
%\justifies
%\exists x A(x)\vdash \exists x B(x)\rightarrow \exists x C(x)
%\using {\rightarrow_{right}}\]
% \justifies
% \vdash \exists x A(x)\rightarrow \left(\exists x B(x)\rightarrow \exists x C(x)\right)
%\using {\rightarrow_{right}}\
%\endprooftree$$

    
\item[Cor] Supponiamo $\vdash_{HA}\varphi\to\psi$. Allora per (HBL1) $\vdash_{HA}\Box(\varphi\to\psi)$.\\
Ma d'altra parte $\vdash_{HA}\Box(\varphi\to\psi)\to(\Box\varphi\to\Box\psi)$ per (HBL2), pertanto $\vdash_{HA}\Box\varphi\to\Box\psi$.\\

\item[HBL3] Introduciamo anzitutto la seguente:

\begin{defi}
Una $\Sigma$ (o $\Sigma_1$) formula è una formula che non contiene quantificazioni universali non limitate; in altre parole, se le uniche quantificazioni universali che vi occorrono sono della forma
\begin{center}
$\forall x(x\le\overline k\to\varphi)$, per qualche $k$.\\
\end{center}
\end{defi}

\noindent Si dimostra il seguente teorema (e lo faremo successivamente, procedendo per induzione sulla struttura di $\psi$):
\begin{center}
$\vdash_{HA}\psi\to\Box\psi$, per ogni $\Sigma$ formula.
\end{center}
 
\noindent Applicando questo risultato a $\psi=\Box\varphi=\exists x\ensuremath{\mathrm{Dim}}(x,\ulcorner\varphi\urcorner)$, che è una $\Sigma$ formula, si ottiene la tesi:
\begin{center}
$\vdash_{HA} \Box\varphi \rightarrow \Box\Box\varphi$.
\end{center}

\noindent Osserviamo che $\psi$ sopra definita è una $\Sigma$ formula, poiché $\ensuremath{\mathrm{Dim}}(x,\ulcorner\varphi\urcorner)$ è una $\Delta_0$ formula, cioè non contenente alcuna quantificazione illimitata (più esplicitamente, una $\Sigma$ formula può contenere un $\exists$ illimitato, mentre le quantificazioni in $\Delta_0$ sono limitate).\\
Ma perché $\ensuremath{\mathrm{Dim}}(x,\ulcorner\varphi\urcorner)$ è una $\Delta_0$ formula?\\
Ricordiamo che il predicato $\ensuremath{\mathrm{DIM}}(m,n)$ è un predicato primitivo ricorsivo: in esso tutte le quantificazioni sono limitate (cioè vi è un bound preciso entro cui sceglierle) ma ciò non vale per la minimalizzazione, per la quale c'è una richiesta illimitata. Pertanto la formula che rappresenta $\ensuremath{\mathrm{DIM}}(m,n)$, cioè $\ensuremath{\mathrm{Dim}}(x,\ulcorner\varphi\urcorner)$, è una $\Delta_0$ formula.

Introduciamo ora definizioni, lemmi e proposizioni che ci per\-met\-te\-ran\-no di dimostrare il teorema sopra:

\begin{defi}
Un termine è chiuso se nessuna variabile compare in esso; una formula è chiusa (e si dice enunciato) se nessuna variabile è libera in essa.\\
Ogni termine chiuso $t$ denota un unico numero naturale; indichiamo con $\overline i$ il numerale per il numero $i$ (e diciamo che $\overline i$ denota $i$).\\
\end{defi}

\begin{lem}
Se $t$ è un termine chiuso e $t$ denota $i$, allora $\vdash t=\overline i$.
\end{lem}

\textsc{\textbf{Dim:}} Per induzione sulla costruzione di $t$:
\begin{itemize}
	\item se $t$ è $\overline 0$, allora $t$ denota $0$, e dunque $\vdash \overline 0=\overline 0$;
	\item se $t$ denota $i$ e $t'$ denota $j$, allora $t+t'$ denota $i+j$. Sia $k=i+j$.\\
	Per l'ipotesi induttiva, $\vdash t=\overline i$ e $t'=\overline j$.\\
	Per la proprietà che se $i+j=k$ allora $\vdash \overline{i+j}=\overline k$, si ha: $\vdash \overline{i+j}=\overline k$. Allora $\vdash t+t'=\overline{i+j}=\overline k$.\\
	Analogamente per successore e moltiplicazione.\\
\end{itemize}

\begin{prop}
Se $t$ e $t'$ sono chiusi e $t=t'$ è vera, allora $\vdash t=t'$.
\end{prop}

\textsc{\textbf{Dim:}} $t$ denoti $i$ e $t'$ denoti $i'$.\\
Per il Lemma 11.1, $\vdash t=\overline i$ e $\vdash t'=\overline {i'}$. Se $t=t'$ è vera, allora $i=i'$, e $\overline i$ è lo stesso numerale di $\overline{i'}$. Allora $\vdash t=t'$.\\

\begin{lem}
$\vdash x<sy\leftrightarrow x<y \vee x=y$.
\end{lem}

\textsc{\textbf{Dim:}} Per definizione $x<y$ è la formula $\exists z\  x+sz=y$. Si ha:\\
$\vdash x<sy\leftrightarrow \exists z\ x+sz=sy$,\\
$\vdash x<sy\leftrightarrow \exists z\ s(x+z)=sy$,\\
$\vdash x<sy\leftrightarrow \exists z\ x+z=y$,\\
$\vdash x<sy\leftrightarrow x+\overline{0}=y \vee \exists w\ x+sw=y$,\\
$\vdash x<sy\leftrightarrow x=y \vee x<y$.\\

\begin{defi}
$\bigvee \left\{x=\overline j: j<i\right\}$ è la disgiunzione di tutte le proposizioni $x=\overline j$ per $j<i$ ed è $\bot$ se $i=0$.\\
\end{defi}

\begin{prop}
$\vdash x<\overline {i} \leftrightarrow \bigvee \left\{x=\overline j: j<i\right\}$.
\end{prop}

\textsc{\textbf{Dim:}} Per induzione su $i$:
\begin{itemize}
\item se $i=0$, $\vdash \neg x<\overline 0$, da cui $\vdash x<\overline 0 \leftrightarrow \bot$;
\item supponiamo che $\vdash x<\overline i\leftrightarrow \bigvee \left\{x=\overline j: j<i\right\}$.\\
Allora per il Lemma 11.2 e l'ipotesi induttiva:
$\vdash x<s\overline i\leftrightarrow \left(x<\overline i \vee x=\overline i\right)$,\\
$\vdash x<s\overline i\leftrightarrow \left(\bigvee \left\{x=\overline j: j<i\right\}\vee x=\overline i\right)$,\\
$\vdash x<s\overline i\leftrightarrow \bigvee \left\{x=\overline j: j<i+1\right\}$.\\
\end{itemize}

\textsc{\textbf{Notazione:}} Scriviamo $\forall y<x\ F$ per abbreviare $\forall y \left(y<x\rightarrow F\right)$;\\
scriviamo $\exists y<x\ F$ per abbreviare $\exists y \left(y<x\wedge F\right)$.\\

\begin{defi}
Diciamo che una formula è una $\Sigma$ formula stretta se fa parte della più piccola classe che contiene tutte le formule
\begin{center}
$u=v$, $\overline 0=u$, $su=v$, $u+v=w$, $u\times v=w$,
\end{center}
e, se $F$ e $G$ sono sono $\Sigma$ formule strette, allora anche $F\wedge G$, $F\vee G$, $\exists x\ F$ e $\forall x<y\ F$ ("`per ogni"' limitato!) sono $\Sigma$ formule strette.\\
\end{defi}

\begin{oss}
\begin{itemize}
	\item tutte le formule atomiche sono $\Sigma$ formule;
	\item ogni $\Sigma$ formula è equivalente (dimostrabilmente equivalente in HA, cioè è dimostrabile in HA la doppia implicazione) ad una formula costruita per congiunzione e quantificazione esistenziale dalle cinque formule sopra riportate, cioè ogni $\Sigma$ formula è equivalente per de\-fi\-ni\-zio\-ne ad una $\Sigma$ formula stretta.\\
Per esempio, $x+sy=s\overline 0$ è equivalente a\\
$\exists u\ \exists v\ \exists w\ \left(sy=u \wedge x+u=v \wedge \overline 0=w \wedge sw=v\right)$.\\
\end{itemize}
\end{oss}

\begin{defi}
Un $\Sigma$ enunciato è una $\Sigma$ formula che è un enunciato.\\
Se $F$ è una $\Sigma$ formula e $S$ è un enunciato ottenuto da $F$ con la sostituzione di termini chiusi, come i numerali, a variabili libere in $F$, allora $S$ è un $\Sigma$ enunciato.
\end{defi}

\begin{oss}
Il seguente teorema fornisce una proprietà importante dei $\Sigma$ enunciati.
\end{oss}

\begin{thm}
Se $S$ è un $\Sigma$ enunciato vero, allora $\vdash S$. 
\end{thm}

\textsc{\textbf{Dim:}} Per induzione su $i$:
\begin{itemize}
\item se $S$ è una formula atomica vera, allora $\vdash S$ (per la Prop. 11.1);
\item se $\left(S\wedge S'\right)$ è vera, allora $S$ e $S'$ sono vere, da cui $\vdash S$ e $\vdash S'$, e dunque $\vdash \left(S\wedge S'\right)$;
\item se $\left(S\vee S'\right)$ è vera, allora $S$ o $S'$ sono vere, da cui $\vdash S$ o $\vdash S'$, e dunque $\vdash \left(S\vee S'\right)$;
\item se $\exists x\ F$ è vera, allora per qualche $i$, $F\left(\overline i\right)$ è vera (cioè il risultato della sostituzione di $\overline i$ a $x$ in $F$ è vera); allora $\vdash F\left(\overline i\right)$, e dunque $\vdash \exists x\ F$;
\item se $\forall x<\overline {i}\ F$ è vera, allora per ogni $j<i$, $F\left(\overline j\right)$ è vera, e allora per ogni $j<i$, $\vdash F\left(\overline j\right)$ e $\vdash x=\overline j \rightarrow F$.\\
Ma $\vdash x<\overline i\leftrightarrow \bigvee \left\{x=\overline j: j<i\right\}$ (per la Prop. 11.2).\\
E dunque si ha: $\vdash x<\overline i\rightarrow F$ e $\vdash \forall x<\overline {i}\ F$;
\item infine, se $S$ è equivalente ad un enunciato dimostrabile, allora $S$ è dimostrabile. Cioè: quando il teorema è dimostrato per le $\Sigma$ formule strette (come abbiamo fatto), allora, poiché ogni $\Sigma$ formula è equivalente ad una $\Sigma$ formula stretta e se una formula è dimostrabile allora lo è anche l'altra, allora il teorema è dimostrato anche per le $\Sigma$ formule.
\end{itemize}
\end{description}
\end{proof}

\begin{oss}
Questo teorema non è affatto banale. Infatti, l'enunciato del teorema sarebbe falso se non ci restringessimo ai $\Sigma$ enunciati (e dunque alle $\Sigma$ formule). In generale, NON è vero: se $S$ è vero, allora $\vdash S$.\\
Questo, infatti, non è altro che l'enunciato del Primo Teorema di Incompletezza: esiste una formula $G$ tale che né $G$ né $\neg G$ è dimostrabile. Ma d'altra parte sappiamo che $G$ è vera perché afferma la propria indimostrabilità, dunque in definitiva $G$ è una formula vera ma non dimostrabile. Dunque $G$ NON è equivalente ad una $\Sigma$ formula perché in essa compare $\neg \exists$ che è intuizionisticamente equivalente a $\forall \neg$, che è un quantificatore esistenziale illimitato!
\end{oss}


\section{Il Teorema di L\"ob ed il Secondo Teorema di Incompletezza}

\noindent Consideriamo la formula $\Box\varphi\to\varphi$: questa esprime il fatto che HA sia corretta riguardo $\varphi$, cioè che se HA dimostra $\varphi$ allora $\varphi$ è vera. Ora, quali sono le formule $\varphi$ per cui HA dimostra di essere corretta?

Intuitivamente, poiché le condizioni di dimostrabilità di ogni formula implicano le sue condizioni di verità, si potrebbe pensare che $\Box\varphi\to\varphi$ dovrebbe essere dimostrabile per ogni $\varphi$.

Al contrario, per dirla con le parole di Rohit Parikh, ``HA non potrebbe essere più modesta riguardo la propria veridicità'': infatti, HA dimostra di essere corretta per $\varphi$ solo quando in effetti essa dimostra già $\varphi$ stessa. Questo è il contenuto del Teorema di L\"ob.

\begin{thm}[Teorema di L\"ob]
Per ogni formula aritmetica $\varphi$,
\begin{center}
se $\vdash_{HA}\Box\varphi\to\varphi$ allora $\vdash_{HA}\varphi$.
\end{center}
\end{thm}

\noindent Un altro modo di porre la questione è il seguente: la dimostrazione del Primo Teorema di Incompletezza mostra che ogni punto fisso della formula $\neg\ensuremath{\mathrm{Th}}(x)$ non è dimostrabile in HA. Cosa si può dire invece dei punti fissi della formula $\ensuremath{\mathrm{Th}}(x)$?

Certamente, ogni teorema è un punto fisso di $\ensuremath{\mathrm{Th}}(x)$. Infatti se $\vdash_{HA}\varphi$, allora per (HBL1) anche $\vdash_{HA}\Box\varphi$ e perciò banalmente $\vdash_{HA}\varphi\leftrightarrow\Box\varphi$. Ma esistono formule che sono punti fissi di $\ensuremath{\mathrm{Th}}(x)$ in modo non banale, cioè senza essere teoremi? Il Teorema di L\"ob dà risposta negativa a questo interrogativo: tutti i punti fissi di $\ensuremath{\mathrm{Th}}(x)$ sono teoremi.

Come preannunciato, la dimostrazione del Teorema di L\"ob fa uso solamente delle tre condizioni di derivabilità e del lemma di diagonalizzazione.

\begin{proof} Si assuma $\vdash_{HA}\Box\varphi\to\varphi$. Applicando il lemma di dia\-go\-na\-liz\-za\-zio\-ne alla formula $\chi_{\varphi}(x)\equiv\ensuremath{\mathrm{Th}}(x)\to\varphi$, si ottiene l'esistenza di una formula $\delta$ tale che $\vdash_{HA}\delta\leftrightarrow(\ensuremath{\mathrm{Th}}(\overline{\ulcorner\delta\urcorner})\to\varphi)$, cioè tale che $\vdash_{HA}\delta\leftrightarrow(\Box\delta\to\varphi)$. Si hanno allora i seguenti fatti, dove i passaggi che non sono esplicitamente giustificati sono pura logica proposizionale (intuizionistica).\\

\begin{tabular}{l l l}
1 & $\vdash_{HA}\delta\leftrightarrow(\Box\delta\to\varphi)$ 														& 									 \\
2 & $\vdash_{HA}\delta\to(\Box\delta\to\varphi)$ 														& da 1							 \\
3 & $\vdash_{HA}\Box\delta\to\Box(\Box\delta\to\varphi)$ 										& da 2 per (Cor)		 \\
4 & $\vdash_{HA}\Box(\Box\delta\to\varphi)\to(\Box\Box\delta\to\Box\varphi)$ 		& per (HBL2)				\\
5 & $\vdash_{HA}\Box\delta\to(\Box\Box\delta\to\Box\varphi)$									& da 3 e 4					 \\
6 & $\vdash_{HA}\Box\delta\to\Box\Box\delta$															& per (HBL3)				 \\
7 & $\vdash_{HA}\Box\delta\to\Box\varphi$																	& da 5 e 6					 \\
8 & $\vdash_{HA}\Box\varphi\to\varphi$																		& per ipotesi				 \\
9 & $\vdash_{HA}\Box\delta\to\varphi$																			& da 7 e 8					 \\
10& $\vdash_{HA}\delta$																								& da 1 e 9					 \\
11&	$\vdash_{HA}\Box\delta$																						& da 10 per (HBL1)	\\
12& $\vdash_{HA}\varphi$																								 & da 9 e 11\\					
\end{tabular}

\noindent Questa catena di deduzioni mostra che $\vdash_{HA}\varphi$, come affermato dall'enunciato del teorema.
\end{proof}

Il Teorema di L\"ob è un risultato sorprendente e di grande potenza. Si vedrà nella prossima sezione come da esso segua speditamente il Secondo Teorema di incompletezza come corollario. Ora vediamo invece che il Teorema di L\"ob può essere internalizzato, cioè che il corrispettivo formale del Teorema di L\"ob è dimostrabile in HA.

\begin{thm}[Teorema di L\"ob internalizzato] \label{internalized lob} Per ogni formula aritmetica $\varphi$,
$$\vdash_{HA}\Box(\Box\varphi\to\varphi)\to\Box\varphi$$
\end{thm}

\begin{proof} Si prenda $\delta$ come nella dimostrazione del Teorema di L\"ob, cioè tale che $\vdash_{HA}\delta\leftrightarrow(\Box\delta\to\varphi)$. In tale dimostrazione, facendo uso esclusivamente di tale proprietà di $\delta$ si era trovato (vedi sopra) $\vdash_{HA}\Box\delta\to\Box\varphi$. Utilizzando questo fatto abbiamo:

\begin{tabular}{l l l}
1 & $\vdash_{HA}\delta\leftrightarrow(\Box\delta\to\varphi)$ 														& 									 \\
2 & $\vdash_{HA}\Box\delta\to\Box\varphi$																	& 									 \\
3 & $\vdash_{HA}\Box(\Box\delta\to\varphi)\to\Box\varphi$										&	da 1 e 2\\
4 & $\vdash_{HA}\varphi\to(\Box\delta\to\varphi)$ & logicamente valida\\
5 & $\vdash_{HA}\varphi\to\delta$															&	da 4 e 1\\
6 &	$\vdash_{HA}\Box\varphi\to\Box\delta$																	& da 5 per (Cor)		 \\
7 & $\vdash_{HA}\Box\varphi\leftrightarrow\Box\delta$																& da 2 e 6					 \\
8 & $\vdash_{HA}\Box(\Box\varphi\to\varphi)\to\Box\varphi$										&	da 3 e 7					 \\
\end{tabular}\\

\end{proof}

\section{Secondo Teorema di Incompletezza}

\noindent Il programma formalista propugnato da Hilbert si proponeva di fondare tutte le teorie matematiche presentandole come sistemi formali. Nell'idea di Hilbert, la solidità di una fondazione di questo tipo avrebbe dovuto essere sancita mediante una dimostrazione con metodi finitistici della consistenza del sistema formale, cioè della sua impossibilità di ricavare contraddizioni.

Nel caso del sistema formale HA, poichè se è dimostrabile una qualunque contraddizione allora è dimostrabile $\bot$, la consistenza è espressa dalla formula $\neg\Box\bot$. \`{E}allora naturale chiedersi se la consistenza di HA sia o meno dimostrabile all'interno di HA stessa.

Il Secondo Teorema di Incompletezza risponde a questa domanda, asserendo che HA non dimostra la propria consistenza, a meno che non sia in effetti inconsistente.

\begin{thm}[Secondo Teorema di Incompletezza]
Se HA è consistente, $\not\vdash_{HA}\mathrm{Con}_{HA}$.
\end{thm}

\begin{proof} Si supponga $\vdash_{HA}\mathrm{Con}_{HA}$, cioè $\vdash_{HA}\Box\bot\to\bot$. Allora dal Teorema di L\"ob si avrebbe $\vdash_{HA}\bot$. Pertanto, per contrapposizione, se HA è consistente, cioè $\not\vdash_{HA}\bot$, allora $\not\vdash_{HA}\mathrm{Con}_{HA}$.
\end{proof}

\noindent Questo significa anche che ogni dimostrazione di consistenza di HA non potrà essere riproducibile in HA e dovrà pertanto fare uso di principi che vanno oltre quelli validi in HA. Inoltre, lo stesso può dirsi per PA. Poiché appare inverosimile che metodi che trascendono il potere di PA possano legittimamente essere detti ``finitistici'', è opinione diffusa che il Secondo Teorema di incompletezza mostri l'inattuabilità del programma di Hilbert per assicurare la non-contraddittorietà della matematica.


%\chapter{Conseguenze}

\begin{abstract}
Queste note sono divise in tre momenti: dopo un inquadramento
storico del problema dei fondamenti della matematica prenderemo in
con\-si\-de\-ra\-zio\-ne alcuni aspetti del pensiero di David Hilbert esponendo
in particolare il suo programma fondazionale, quindi parleremo delle
conseguenze fon\-da\-zio\-na\-li dei Teoremi di Incompletezza di G\"odel sottolineando le difficoltà in cui si imbattè il programma di Hilbert a seguito di questi.
\end{abstract}



\section{Introduzione all'epoca}

Il diciannovesimo secolo fu un'epoca di grandi trasformazioni e un'età durante la quale la matematica subì dei cambiamenti così profondi che non è esagerato parlare di una vera e propria seconda nascita della materia così come nell'età greca c'era stata la prima. Analisi, geometria, algebra e logica furono completamente rivoluzionate.

L'analisi, dopo un secolo di travolgenti successi, attraversò la fase cosiddetta di \emph{rigorizzazione dell'analisi}, durante la quale i concetti di funzione, serie, integrale e continuità vennero affinati da parte di Cauchy, Fourier e Riemann, fino a li\-be\-rar\-li dall'uso degli infinitesimi ed infiniti attuali, tramite il concetto di li\-mi\-te. Un passo importante di tale processo fu la definizione rigorosa, nel 1972, dei numeri reali ad opera di Weierstrass, Dedekind e Cantor.

In geometria, nel tentativo di dimostrare la dipendenza del quinto postulato della geometria euclidea dagli altri quattro, si assistette alla nascita delle geometrie non-euclidee con i lavori di Gauss, Lobacewskij e Bolyai. La sco\-per\-ta di tali geometrie metteva così in crisi la nozione di intuizione geometrica e l'idea che gli assiomi debbano necessariamente codificare un aspetto univoco ben determinato della nostra concezione dello spazio.

L'algebra assunse una forma completamente astratta: importanti furono gli sviluppi in teoria delle equazioni ad opera di Galois e il lavoro di Grassman sulle algebre che generalizzano i quaternioni di Hamilton.\\
Si assistette inoltre alla nascita della scuola britannica di algebristi, tra i quali ricordiamo Hamilton, Cayley, Boole e Sylvester.\\
Di notevole portata furono, infine, le interazioni tra l'algebra e le altre discipline della matematica: in particolare, i moderni metodi algebrici sostituirono i metodi sintetici in geometria e permisero lo sviluppo dello studio delle funzioni ellittiche (con Kummer, Dedekind a Abel) e dell'analisi (con Dirichlet, Jacobi, Riemann e Hamilton).

La logica subì una rigorosa definizione dei suoi linguaggi formali, in particolare Frege e Peano provvidero alla definizione del linguaggio della logica dei predicati del primo ordine.\\
Si diede avvio inoltre allo studio algebrico della logica, con Boole e De Morgan, e si assistette alla nascita della teoria dei modelli.

Questa rivoluzione delle scienze matematiche portò ad un acceso dibattito sulla definizione dell'oggetto proprio degli studi matematici: la tradizionale definizione della matematica quale scienza della grandezza, della misura e della quantità non era più adeguata!

Strettamente legata a questa esigenza di caratterizzazione della scienza ma\-te\-ma\-ti\-ca, vi fu un'esigenza di unificazione del sapere matematico, ormai fra\-zio\-na\-to in moltissime teorie interdipendenti.

Tale obiettivo fu raggiunto attraverso quella che ancor oggi è una delle teorie più discusse e interessanti di tutta la matematica, ossia la teoria degli insiemi, ad opera del matematico tedesco Georg Cantor.

La potenza del linguaggio insiemistico portò alla nascita di nuove teorie (topologia, analisi funzionale, teoria della misura di Lebesgue, geometria algebrica,...) e al rinnovamento di teorie già affermate (analisi reale, algebra astratta,...).

D'altra parte, la scoperta a cavallo del 1900, da parte di Cantor e Russell, dell'esistenza di paradossi all'interno della teoria degli insiemi, aprì il periodo di \emph{crisi dei fondamenti}.
Visto il ruolo fondazionale attribuito alla teoria degli insiemi, la comparsa dei paradossi rendeva dunque pressante il bisogno di una sicura fondazione della matematica.

Tre furono i principali indirizzi circa le proposte fondazionali: il \emph{logicismo} con Frege e Russell, l'\emph{intuizionismo} con Brouwer, il \emph{formalismo} con Hilbert.\\
\bigskip


\section{Il Formalismo}

Per l'importanza dei suoi risultati nelle discipline più svariate, per la profondità delle sue intuizioni e per l'influenza che esercitò sui suoi contemporanei, David Hilbert è indubbiamente una delle figure chiave della storia della matematica, ideologo e principale esponente del Formalismo.

Hilbert venne a contatto con numerosi matematici illustri della sua epoca, come Karl Weierstrass, Richard Dedekind e Georg Cantor, e a G\"ottingen rappresentò per quattro decenni la figura guida di un'intera scuola: tra i suoi studenti vi furono matematici del calibro di Herman Weyl, Felix Bernstein, Richard Courant, Ernst Zermelo; John Von Neumann e Paul Bernays erano i suoi assistenti, e figure importanti come Emmy Noether, Edmund Landau, Alonzo Church frequentavano lo stesso ambiente.

Oltre a questo ruolo di indirizzo, Hilbert fornì in prima persona significativi contributi ai campi di ricerca più disparati, dalla teoria degli anelli a quella dei numeri, dalla geometria alla fisica matematica e all'analisi funzionale, oltre che alla logica e al problema dei fondamenti.
Gli interessi matematici di Hilbert furono vastissimi, di modo che è difficile trovare un settore della matematica nel quale egli non sia intervenuto con contributi di tale peso ed importanza da indurre in essi profonde trasformazioni tematiche e metodiche.

In virtù della sua conoscenza enciclopedica della matematica, nel 1900, venne incaricato di tenere il discorso di apertura del Congresso internazionale dei ma\-te\-ma\-ti\-ci tenutosi a Parigi. In tale occasione propose alla comunità ma\-te\-ma\-ti\-ca un elenco di ventitré problemi aperti, da lui ritenuti i più rilevanti da affrontare nel secolo incipiente: tale lista ebbe una grossa influenza sulle direttrici della ricerca matematica nel primo Novecento, e i ventitré problemi vennero tutti affrontati in modo sistematico, e molti risolti almeno in parte.

Il lavoro di Hilbert illustra gran parte dei temi e degli interessi della ma\-te\-ma\-ti\-ca del diciannovesimo secolo: l'enfasi per la rappresentazione simbolica e per la caratterizzazione astratta, l'uso di metodi infinitari e non costruttivi e la ricerca di un'unità fondazionale. E proprio la ricerca di questa unità e di nuovi metodi matematici, lo renderà uno degli artefici della insiemizzazione della matematica e della conseguente rivalutazione della teoria degli insiemi sviluppata da Cantor.

In virtù di tutto ciò egli vide nei paradossi della teoria degli insiemi un pro\-ble\-ma drammatico e di cui era necessaria una chiara e precisa soluzione, al fine di restituire alla matematica quella certezza e inoppugnabilità che, a suo giudizio, le era propria. La sua risposta alla crisi dei fondamenti fu la proposizione di una nuova forma di metodo assiomatico: il \emph{metodo assiomatico formale}.
\newpage



\subsection{Il metodo assiomatico formale}

Il metodo assiomatico è quel modo di sviluppare una teoria, che consiste nel fissare certe proposizioni iniziali (dette \emph{assiomi} o \emph{postulati}) e da queste, procedendo per deduzione, ottenere nuove proposizioni. Questo approccio è stato utilizzato fin dall'antichità e l'esempio più illustre è rappresentato dagli \emph{Elementi} di Euclide.

Fino a tutto l'Ottocento, l'idea soggiacente a questo metodo era la seguente: prendendo come assiomi proposizioni la cui verità sia evidente, e prestando attenzione ad usare solo modi di inferenza che preservino la verità, si ottengono proposizioni la cui verità è implicita nella verità degli assiomi, ed è pertanto garantita anche quando essa non è immediatamente evidente. In altri termini, il metodo assiomatico consisteva nel dimostrare la verità di certe proposizioni riducendola alla verità di altre proposizioni prefissate considerate evidenti.

Questa visione degli assiomi come proposizioni ``evidentemente vere'' per\-si\-stet\-te fino alla fine del XIX secolo. Kant, per esempio, riteneva che gli assiomi della geo\-me\-tria euclidea fossero proposizioni vere \emph{a priori}, cioè esprimessero verità inerenti alla nostra modalità di percezione dei fenomeni.

Nell'Ottocento, però, si comprese che negando il quinto postulato di Euclide (il celebre \emph{postulato delle parallele}) non si otteneva alcuna contraddizione, in quanto gli assiomi ottenuti sarebbero risultati veri semplicemente modificando l'interpretazione intuitiva dei termini `punto' e `retta', come dimostrato da Beltrami e da molti altri dopo di lui.\\
Questo importantissimo esempio concreto portò a prendere pienamente coscienza della dipendenza della nozione di verità dall'interpretazione intesa dei concetti coinvolti nella proposizione.

\`{E} in questo contesto che si inserisce nel 1899 la pubblicazione dei \emph{Grundlagen der Geometrie} (\emph{Fondamenti della geometria}) di Hilbert, un'opera che ha avuto una profonda influenza sulla matematica del Novecento.

Hilbert mostra di essere pienamente cosciente del fatto che, sebbene la nozione di verità di una proposizione dipenda dall'interpretazione intesa, è possibile concepire la relazione di consequenzialità logica in modo tale che essa risulti indipendente da tale interpretazione. In altre parole, le inferenze ammissibili devono basarsi esclusivamente sulla forma logica delle relazioni tra concetti, e non devono dipendere in nessun modo dal significato intuitivo attribuito a tali concetti (ciò che Hilbert chiama `intuizione geometrica'). All'interno della teoria, le nozioni si intendono implicitamente definite dagli assiomi, e le proprietà che è lecito usare nelle inferenze sono tutte e sole quelle espresse dagli assiomi.

Procedendo in questo modo, si ottengono dimostrazioni che stabiliscono non più la \emph{verità} della conclusione, ma il fatto che, data una \emph{qualunque} interpretazione che renda veri gli assiomi della teoria, questa deve anche rendere vera la conclusione. Che questo fosse ciò che Hilbert aveva in mente è chiarissimo nella sua celebre affermazione che ``si deve sempre poter dire al posto di `punti, rette, piani', `tavoli, sedie, boccali di birra' ''.

La visione della matematica che emerge da quest'opera è quella del `se... allora': la matematica consiste nell'esplorazione delle conseguenze logiche di certe assunzioni. Se la matematica deve essere assolutamente certa, la validità dei suoi teoremi non può in alcun modo dipendere dalla configurazione contingente del mondo. Il problema della verità delle proposizioni (relativamente ad una data interpretazione) è invece empirico e pertanto giace al di fuori del dominio della matematica.
\bigskip

Il metodo assiomatico era sostenuto da Hilbert come `una procedura ge\-ne\-ra\-le per il pensiero scientifico', applicabile ad ogni teoria la cui costruzione logica possa essere basata su un numero limitato di proposizioni fondamentali.\\
\`{E} in questo senso che Hilbert riconosce alla matematica, identificata con il metodo assiomatico, un valore universale, affermando che:

\emph{nella loro parte teorica [le scienze] si dispiegano direttamente all'interno della matematica. [...] Tutto ciò che può essere oggetto del pensiero scientifico, non appena è maturo per la formazione di una teoria, cade sotto il metodo assiomatico e per suo tramite sotto la matematica. Progredendo verso livelli sempre più profondi di assiomi otteniamo anche illuminazioni sempre più profonde sulla natura del pensiero scientifico e diveniamo sempre più consapevoli dell'unità del nostro sapere. Nel segno del metodo assiomatico la matematica sembra essere chiamata ad un ruolo guida in tutto ciò che è scienza.}
\bigskip

\subsection{Il finitismo}

Attraverso il metodo assiomatico formale, Hilbert ritiene di essere in grado di dare alla matematica una fondazione capace di completa chiarezza e certezza.  La necessità di una fondazione, per Hilbert, non era affatto dettata dal bisogno di ``rinforzare'' i risultati della matematica stessa, quanto dal desiderio di spiegare una volta per tutte cosa sia la matematica e come si origina la sua certezza.

Condizione necessaria affinché si possa affermare la certezza delle proposizioni matematiche è che esse riguardino oggetti che sono completamente conoscibili, dunque finiti e concretamente presentabili, e le cui relazioni rilevanti al discorso siano, diremmo oggi, decidibili in senso intuitivo. Ora, i segni che si usano in teoria dei numeri soddisfano questi requisiti, e viceversa è ra\-gio\-ne\-vo\-le assumere che un insieme di oggetti della tipologia descritta possano essere legittimamente chiamati un \emph{sistema di segni}.

Hilbert nega che oggetti infiniti, quali i numeri reali, siano immediatamente rappresentabili alla nostra conoscenza e quindi ammissibili come oggetto di pensiero contenutistico. Infatti, egli afferma che ``l'infinito non è rea\-liz\-za\-to in nessun luogo; non esiste in natura, né è ammissibile come fondamento del nostro pensiero razionale [\dots] Pertanto, le operazioni con l'infinito devono essere rese certe all'interno del finito''.

Saranno dunque quegli oggetti finiti e concretamente esibibili chiamati \emph{segni} ad essere oggetto di quell'attività che Hilbert chiama \emph{matematica contenutistica}, \emph{finitistica} o \emph{metamatematica}. Nel contesto di un sistema di segni, le singole proposizioni matematiche hanno un valore di verità.
Inoltre tali valori di ve\-ri\-tà sono determinabili con procedure finite mediante un semplice confronto di simboli.
\bigskip

\subsection{La proposta formalista}

Hilbert ritiene perciò che la completa certezza e verificabilità degli enunciati matematici si possano ottenere solo all'interno di un contesto in cui gli oggetti siano rigorosamente finiti e supervisionabili, e le loro relazioni immediatamente verificabili. D'altra parte, egli vuole giustificare \emph{tutta la matematica}, anche quella transfinita, ponendola su basi certe.

Ora, come è possibile conciliare questi due \emph{desiderata}? Certo i numeri reali \emph{non} sono in generale oggetti finitamente presentabili, né le loro relazioni (ad esempio $x<y$) sono direttamente verificabili, cioè decidibili. Dunque come è possibile ottenere assoluto rigore nel campo dell'analisi?

La proposta di Hilbert è di utilizzare il metodo assiomatico.\\
Supponiamo infatti di aver fondato una teoria transfinita, quale ad esempio l'ana\-li\-si, in modo assiomatico: ora, sebbene gli oggetti di cui la teoria stessa parla non siano oggetti finiti, tanto le proposizioni della teoria stessa (in particolare gli assiomi) che le dimostrazioni lo sono.\\
Inoltre, e questa è una delle profondissime intuizioni di Hilbert, i modi di inferenza disponibili nella teoria potranno sempre essere assegnati in modo chiaro come parte della teoria stessa, sottoforma di regole.\\
Pertanto, in un contesto assiomatico di questo tipo, sarà possibile stabilire in modo effettivo se una data sequenza finita di proposizioni costituisca o meno una dimostrazione di una determinata proposizione.

\`{E} dunque possibile sviluppare l'analisi (e analogamente le altre teorie infinitarie) in maniera finitistica prendendo come oggetto del nostro ragionamento non gli ``enti infiniti'' di cui la teoria parla, ma gli enti finiti che costituiscono la teoria stessa, cioè gli assiomi, le dimostrazioni, i teoremi. 

Hilbert si serve dunque dei metodi della logica matematica per tradurre le componenti di una teoria assiomatica in oggetti della matematica finitaria, cioè in segni. In questo modo, specificata una teoria formale (che consiste di assiomi e regole) è possibile verificare effettivamente, cioè con un numero finito di operazioni se una certa figura sia o meno una dimostrazione di una formula.

%Hilbert chiama quindi \emph{matematica} ciò che avviene all'interno di un si\-ste\-ma formale. I sistemi formali stessi, che consistono di oggetti finiti che altro non sono che segni, sono quindi oggetto della matematica finitistica, che viene pertanto detta \emph{metamatematica}.

L'intento di Hilbert non è quello di convincere i matematici a sviluppare le loro teorie all'interno di un sistema formale, ma solo quello di mostrare che la matematica transfinita può essere riformulata all'interno di un tale sistema ed è dunque possibile comprenderla completamente all'interno del finito, senza invocare l'esistenza di oggetti infiniti. In questo modo si ottiene inoltre un metodo che permette di controllare con assoluta certezza la correttezza di un risultato ottenuto anche con metodi transfiniti.

In definitiva, il \emph{programma formalista} prevede la completa formalizzazione dell'aritmetica, spogliando segni logici e matematici del loro significato, attraverso la costruzione di un {\em sistema assiomatico formale}, cioè un sistema costituito di soli segni "`"`senza significato"'"' e regole sintattiche per legare i segni tra loro.\\
In tal modo la metamatematica è un linguaggio con cui si parla della matematica dall'esterno, cioè parlando dei segni che determinano formule o espressioni ma\-te\-ma\-ti\-che: la \emph{matematica}, infatti, tratta delle formule e delle espressioni considerando il loro significato e l'interpretazione che noi gli diamo, mentre la \emph{metamatematica} tratta formule ed espressioni considerandole come stringhe di simboli privi di significato.

Questa operazione di formalizzazione ci permette di affrontare lo studio della matematica in maniera diversa.\\
Infatti, è possibile operare direttamente su queste stringhe di segni con una serie di manipolazioni di simboli prefissate e ricavare, a partire da alcune formule iniziali, delle nuove espressioni, senza preoccuparci di quale sia il significato di ciò che stiamo ottenendo o se si tratti di un'affermazione vera. Questa serie di espressioni potrà poi essere interpretato in una dimostrazione.\\
In questo caso, le formule iniziali vengono interpretate come gli assiomi o le ipotesi da cui si parte per la dimostrazione, le manipolazioni di simboli sono interpretate come le regole logiche di deduzione, mentre la stringa finale a cui si arriva è il teorema che si è dimostrato. 



\section{Il problema della coerenza}

Si è visto come la possibilità di sviluppare teorie di strutture transfinite con metodi finitari sia ottenuta a scapito della capacità della matematica di costruire esplicitamente una qualche rappresentazione degli oggetti del proprio discorso.\\
Hilbert mostra dunque come l'astrazione ci permetta di sviluppare teorie anche in assenza di un dominio del discorso e di procedere secondo quella che chiama una filosofia del `come se', nel senso che si ragiona `come se' oggetti del tipo descritto dagli assiomi fossero dati, prescindendo dalla possibilità concreta di costruire tali oggetti.\\
Attraverso i sistemi formali, la matematica si configura come ``teoria delle forme'' astratte, cioè dei sistemi di connessioni logiche tra concetti.

Questa prospettiva sulla matematica apre due tipi di problemi.
\begin{itemize}
\item Quando dovremmo studiare un certo sistema formale? Cioè, quali sistemi formali sono matematicamente significativi e perché?
\item Quando possiamo studiare un certo sistema formale? Cioè quali sistemi formali rappresentano un sistema di connessioni logiche che sia possibile almeno in principio, cioè che possa essere concepito senza generare contraddizioni?
\end{itemize}

La prima di queste domande non sembra aver mai preoccupato Hilbert e Bernays, ma la mancanza di un'elaborazione su questo tema è stato spesso oggetto di critiche alle posizioni formaliste.

Il secondo problema, quello di garantire una volta per tutte la coerenza di certe teorie -in particolare dell'aritmetica e dell'analisi- è invece considerato da Hilbert e Bernays il problema cruciale della metamatematica.

L'idea di Hilbert è che ogniqualvolta si voglia studiare un certo sistema di connessioni logiche tra concetti ci si può porre in un sistema formale, il che permette di sviluppare una teoria pienamente rigorosa, ma è importante garantire che tale teoria ``stia in piedi'', cioè non si riveli a posteriori insensata anche dal punto di vista meramente combinatorio, permettendo di derivare qualunque conclusione.

Ovviamente -sostiene Bernays- c'è una schiacciante evidenza empirica della coerenza dell'aritmetica e dell'analisi, che risiede nell'ampio uso che si è fatto di tali teorie senza che mai sorgesse l'ombra di una contraddizione. Tuttavia, grazie alla traduzione di tali teorie in sistemi formali, cioè oggetti della ma\-te\-ma\-ti\-ca finistica, l'enunciato di coerenza della teoria diviene una proposizione della metamatematica.

Ora, qualora gli assiomi di una teoria siano soddisfatti da una certa struttura finitisticamente presentabile e le regole di derivazione preservino la verità rispetto a tale struttura, la coerenza della teoria è garantita: infatti, tutti i teoremi de\-ri\-va\-bi\-li nella teoria saranno veri quando interpretati in tale struttura, e poiché una contraddizione non può mai essere vera per definizione, la teoria non può derivare contraddizioni.
Una struttura come quella descritta è detta un \emph{modello finito} della teoria, e ciò che si è detto è che una teoria che ammetta modelli finiti è coerente.

Il problema è che teorie quali l'aritmetica, l'analisi e la teoria degli insiemi non possono per loro stessa natura ammettere modelli finiti. L'unico modo per affrontare il problema della coerenza è quindi quello diretto: esaminare gli assiomi e le regole della teoria e mostrare, con ragionamenti sintattici rigorosamente finitistici, che essi non permettono di derivare la formula $1\neq 1$ (o un'altra qualunque contraddizione fissata).

Certo è possibile dare dimostrazioni di coerenza relativa: ad esempio, nei \emph{Grundlagen der Geometrie} la coerenza degli assiomi della geometria euclidea è dimostrata costruendo un modello per tali assiomi basato sui numeri reali, ed è quindi ridotta alla coerenza dell'analisi.

Tuttavia, affinché il tutto stia in piedi si deve dimostrare in modo diretto la coerenza di almeno una teoria.
Hilbert si propone di dimostrare me\-ta\-ma\-te\-ma\-ti\-ca\-men\-te la coerenza dell'a\-ritme\-ti\-ca formalizzata, utilizzando metodi finitari. Cioè vuole dimostrare l'impossibilità di ottenere formule formalmente contraddittorie dagli assiomi, utilizzando però procedimenti che non fanno uso di un numero infinito di operazioni o proprietà delle formule, e che sono detti perciò \emph{finitari}.

Nel 1900, al secondo Congresso Internazionale dei Matematici, Hilbert presenta una lista di problemi aperti: il secondo di questi è, appunto, la dimostrazione della coerenza dell'aritmetica!
\`{E} così che Hilbert, nel corso degli anni Venti, insieme ai giovani Ackermann e Von Neumann, inizia a lavorare al problema della coerenza dell'aritmetica: qualora fosse riuscito a dimostrarla, ciò avrebbe segnato il trionfo del piano fondazionale formalista, noto come \emph{programma di Hilbert}.
\bigskip

\section{G\"odel}

Kurt G\"{o}del nacque a Br\"unn, in Moravia, il 28 aprile 1906. Conseguita la licenza liceale, si trasferì a Vienna per iniziare i suoi studi universitari, inizialmente con l'intenzione di studiare fisica teorica, ma poi dedicandosi alla matematica e alla filosofia, e concentrandosi infine sulla logica matematica.

Il decennio 1929-1939 fu un periodo di lavoro intenso, che produsse i principali risultati di G\"odel nel campo della logica matematica.\\
Nel 1930 egli cominciò ad approfondire il programma di Hilbert di stabilire con mezzi finitari la non contraddittorietà dei sistemi assiomatici formali per la ma\-te\-ma\-ti\-ca ed  è proprio cercando la soluzione a questo problema che arrivò, invece, a dimostrare che questa dimostrazione di coerenza è impossibile, nella forma in cui la immaginava Hilbert. 

Pubblicò, ancora venticinquenne, il suo risultato nel 1931 su una ri\-vi\-sta scien\-ti\-fi\-ca\footnote{Monatshefte f\"{u}r Mathematik und Physik}, con un articolo dal titolo "`"`Uber formal unentscheidbare Satze der \emph{Principia Mathematica} und verwandter systeme"'"' ("`"`Sulle proposizioni formalmente indecidibili dei \emph{Principia Mathematica} e dei sistemi affini"'"').

Le conclusioni e i caratteri del tutto originali di questi teoremi attirarono presto l'attenzione di molti intellettuali. Uno dei primi a riconoscere il potenziale significato dei risultati di incompletezza e ad incoraggiarlo a proseguire verso un loro più ampio sviluppo fu John von Neumann, che era tre anni più vecchio di G\"odel ma che si era già distinto per i brillanti risultati in teoria degli insiemi, teoria della dimostrazione, analisi e fisica matematica.
\bigskip



\section{Il Formalismo dopo i Teoremi di G\"odel}

Hilbert aveva concepito un ambizioso e articolato piano per mettere al sicuro tutta la matematica esistente, sviluppandola in modo perfettamente su\-per\-vi\-sio\-na\-bi\-le e dimostrandola libera da contraddizioni.

Ma Hilbert si era spinto anche oltre, congetturando che il metodo assiomatico fornisca i mezzi non solo per formulare, ma anche per risolvere ogni problema matematico. Tale pretesa può essere giustificata solo ritenendo che le teorie formali con cui la matematica ha a che fare siano \emph{complete}, cioè che dimostrino sempre o una formula o la sua negazione. Ciò è esplicitamente congetturato da Hilbert e Bernays per quanto riguarda l'aritmetica e l'analisi e sembra ci si aspettasse lo stesso anche per la teoria degli insiemi.

Ma le cose non stanno come Hilbert pensava: nel 1931 G\"odel dimostrò infatti che ogni teoria formale che estenda la teoria standard PA dell'aritmetica non può essere completa, dato che esistono nella teoria delle \emph{proposizioni indecidibili} che il sistema non è in grado né di dimostrare né di refutare.

Ma c'è di più: G\"odel mostrò che ogni proposizione della me\-ta\-ma\-te\-ma\-ti\-ca su un dato sistema formale è equivalente ad una proposizione aritmetica, traducibile in una formula nel linguaggio dell'aritmetica. In particolare, alla proposizione di consistenza (o coerenza) di una teoria $T$ cor\-ri\-spon\-de una formula aritmetica $\mathrm{Con}(T)$.\\ G\"odel dimostrò che ogni teoria $T$ che estenda la teoria standard dell'aritmetica non dimostra la formula $\mathrm{Con}(T)$.

Ne segue l'impossibilità di dimostrare con metodi finitistici la consistenza dell'aritmetica e, \emph{a fortiori}, la consistenza di una teoria all'interno della quale è possibile sviluppare l'aritmetica, quale quella degli insiemi.

Da ciò segue che una dimostrazione di coerenza di un tale sistema deve ne\-ces\-sa\-ria\-men\-te fare ricorso a qualche principio non contenuto in esso.\\
Quindi se si cerca una dimostrazione finitaria della coerenza di tale teoria, si deve estendere l'ambito della matematica finitaria oltre ciò che è esprimibile nell'aritmetica formale e questo effettivamente è stato fatto da Gerhard Gentzen che trovò nel 1936 una dimostrazione di coerenza per l'a\-ritme\-ti\-ca formale che fa uso dell'induzione fino ad un particolare numero ordinale transfinito, chiamato epsilon-zero.\\
Se poi dall'aritmetica si passa all'analisi matematica, cioè ad una teoria formale per i numeri reali, la dimostrazione di coerenza deve attendere il 1970 e richiede l'induzione fino ad un ordinale infinitamente più grande di epsilon-zero. 

Inoltre, come si è dedotto sopra, la teoria ZF non è in grado di dimostrare la propria coerenza, quindi per farlo abbiamo bisogno di un principio non dimostrabile in ZF stessa. Ora, da un lato è del tutto irragionevole ritenere che la matematica finitaria non sia contenuta nella teoria degli insiemi nella quale è di fatto contenuta tutta la matematica d'oggi e dall'altro, fatto ancora più rilevante, è attualmente del tutto inimmaginabile un principio matematico che trascenda la teoria degli insiemi e quindi risulta proprio impossibile allo stato attuale fornire una prova di coerenza di tale teoria.\footnote{Sambin, \emph{Alla ricerca della matematica perduta}.}
\bigskip

Il programma di Hilbert per mettere al sicuro la matematica dai paradossi garantendone la consistenza con mezzi ``al di sopra di ogni sospetto'' era in questo modo distrutto.

Nonostante questo duro colpo alla sua filosofia, il Formalismo si impose nella pratica matematica come fondazione dominante: oggi, gran parte della comunità matematica utilizza una teoria assiomatica, la teoria degli insiemi di Zermelo-Fraenkel con assioma di scelta (ZFC), come fondamento per il proprio lavoro.

Il successo del metodo assiomatico formale è probabilmente da attribuirsi al fatto che esso rimane ad oggi l'unica maniera in cui è possibile sviluppare in modo completamente chiaro e finitistico teorie transfinite, quali la teoria degli insiemi e l'analisi, preservandone tutti i risultati (anzi, favorendo salti nell'infinito sempre più arditi e slegati da ogni possibilità di interpretazione costruttiva).

A prescindere comunque dalla teoria degli insiemi e dal suo attuale successo, il Formalismo ha dato un grosso contributo alla matematica e alla sua filosofia per una ragione più generale.\\
Grazie al metodo assiomatico formale, infatti, fissate le premesse e le regole logiche, il procedimento di deduzione diviene un processo combinatorio e pertanto non controverso e anzi addirittura automatizzabile. In questo modo si garantisce una volta per tutte la possibilità di risolvere ogni disputa sulla correttezza dei teoremi formalizzati, e di ricondurre i disaccordi a una diversa (ma libera) scelta di princìpi (assiomi o regole logiche). 


\include{11_church}

\begin{thebibliography} {}
\bibitem{Tur} \textsc{Turing}, \textsl{On computable numbers, with an
application to the entscheidungproblem}, 1936: abelard.org.
\bibitem{Stnfrd} \textsc{Online Stanford Enciclopedia}: plato.stanford.edu.

\bibitem{CompNr} \textsl{Computable Numbers and the Turing Machine}, 1936:
turing.org.uk.

\bibitem{WTM} \textsc{Jack Copeland}, \textsl{What is a Turing Machine?}, 2000:
alanturing.net.

\bibitem{Cut} \textsc{N. Cutland}, \textsl{Computability. An
  Introduction to Recursive Function Theory}, 1980.

\bibitem{Dis} \textsc{Unipd}, \textsl{Dispense del corso di Logica 2},
  2009.
\bibitem{Wik} \textsc{Wikipedia}, \url{http://en.wikipedia.org}.
\bibitem{TK} \textsc{G. Takeuti}, \textsl{Proof Theory}, 1987.

\bibitem{key-1}{[}Hilbert, 18.. D.HILBERT. Ricerche sui fondamenti
della matematica, a cura di V.Michele Abrusci, Bibliopolis, 1984.

\bibitem{key-4}{[}Sambin, 1987 G. SAMBIN. Alla ricerca della certezza
perduta. 

\bibitem{key-5}{[}Cantini, 1979 A. CANTINI. I fondamenti della matematica,
Loescher, 1979.

\bibitem{key-8}{[}Hintikka, 19.. HINTIKKA. On Godel.

\bibitem{key-1}{[}Avigad, 2001 J.AVIGAD \& E.H.RECH. {}``Clarifyng
the nature of infinite'': the development of metamathematics and
proof theory.

\bibitem{key-1}{[}Zach, .... R.ZACH. Hilbert's program then and now.

\bibitem{Boolos} \textsc{G. Boolos}, \textsl{The Logic of Provability}, Cambridge University Press, 1993.

\bibitem{Visser} \textsc{A. Visser}, \textsl{Aspects of Diagonalization and Provability}, PhD thesis, Utrecht, 1981.

\bibitem{Mendelson} \textsc{Elliot Mendelson}, \textsl{Introduzione alla logica matematica}, 1972

\end{thebibliography}

\end{document}
